\documentclass{article}
\title{bdots}
\date{}

\usepackage{setspace}
\doublespacing

\usepackage[margin=1in]{geometry}
\usepackage{amsmath}
\usepackage{multirow}
\usepackage{hhline}
\usepackage[table]{xcolor}

\newcommand{\xt}{\texttt}% \xt will be replaced with \textit

\usepackage{listings}

\begin{document}

%https://www.namsu.de/Extra/klassen/latex-article-template.html

\maketitle

%\begin{abstract}
abstract
\end{abstract}

\section{Introduction}
intro

\paragraph{Outline}
outline



\begin{abstract}
The Bootstrapped Differences of Timeseries (bdots) was first introduced by Oleson (and others) as a method for controlling type I error in a composite of serially correlated tests of differences between two time series curves in the context of eye tracking data.  This methodology was originally implemented in R by Seedorff 2018. Here, we revist that implementation, both improving the underlying theoretical components and creating a more robust implementation (that word twice) in R. 
\end{abstract}

\section{Introduction}

Idk, introduction stuff

This paper is not intended to serve as a complete use guide to updates in the bdots package. Rather, the purpose is to showcase major changes and improvements to the package, with those seeking a more comprehensive treatment directed to the package vignettes. 

Updates to the bdots package have been such that there is little resemblance to the original. Rather than taking a ``compare and contrast" approach, we will first enumerate the major changes, followed by a general demonstration of the package use. 

\begin{enumerate}
%\item Parameter and nonparametric functions (not going to be true)
\item User defined curves
\item Permit fitting for arbitrary number of groups
\item Updates to bootstrapping algorithm and introduction of permutation test
\item Automatic detection of paired tests based on subject identifier
\item Allows for non-homogenous sampling of data across subjects and groups
\item Introduce formula syntax for bootstrapping difference function
%\item Removal of auto-correlation assumption
\item bdots object inherits from data.table class
\item bdots is now stylized ``bdots"
\end{enumerate}

\paragraph{Bootstrapped differences in time series}

The high level motivation, abstracted from the particulars, is more or less as follows: we are often interested in comparing time series between two or more groups. A full(er) review of previous methods can be found in Seedorff,  though we can limit the scope of interest to specifying that we are interested in ``developing a statistical tool to (1) detect differences in two time series (as the VWP) and (2) to offer a precise characterization of the time window in which a difference occurs \cite{seedorff2018bdots}. 

A typical instantiation of this problem occurs when we have two groups (or experimental conditions, etc.,) in which subjects in each group have an associated time series. It's assumed that each group has some distribution of associated functions in time, and we are interested in identifying windows in time in which these distributions are significantly different.

There are also situations in which we are interested in the difference of differences. I don't remember where or why this is done, some justification for doing this instead of some complicated F test, but I can find that later. Instead, let me offer an example. I'm already not going to like this example, so let me just put it anyways knowing that it will be deleted. Suppose we are interested in understanding how the color of a vehicle differentially impacts performance based on the vehicle type. We know that there is some difference between cars and trucks. Suppose then that we look at the difference between red cars and red trucks and then the difference between blue cars and blue trucks. If color does not mediate this difference, the difference between red cars and trucks should be the same as the difference between blue cars and trucks. This information can be determined if we look then at the difference between the differences. 

The original bdots package was predicated on comparing differences between dense, highly correlated time series by first specifying functional forms and then performing statistical tests on each of the observed time points. With verison 2.0, this is no longer the case, and bdots is able to fit parameteric functions to any type of data observed in time. Along with methodological improvements, we have included more options in determining statistical significance in the differences of curves, utilizing a robust permutation testing framework when the assumptions of autocorrelation do not hold. In addition to methodology, a number of quality-of-life improvements have also been made, greatly simplifying syntax, creating more robust functions, and the addition of a number of useful methods for handling returned objects.

In summary, \texttt{bdots} has transitioned from a package focused exclusively on densly sampled timeseries assuming a limited number of functional forms to a robust framework for identifying time windows of significant difference across a wide breadth of timeseries-adjacent data. 

%talk about how we are no longer just doing time series, as indicated with saccade method. we have now branched to the more general problem of comparing functional forms of observed data.

\section{Methodology and Overview} 

There are major changes in the underlying methodology used in the bdots package, and we will briefly review the current methodology here (without explicit comparisons to the original). For those interested, please see some other article that I have to write.

Broadly, there are two steps to performing an analysis with the bdots pacakge: fitting the curves to observed data and bootstrapping differences between groups. The first step involves specifying an underlying curve, $f$, which may or may not be parametric. Along with the observed data $y$ for each $i$th subject, bdots, via fitting with gnls, returns a set of parameters along with an estimate of their covariance.

\begin{equation}
F: f \times y_i \rightarrow N(\hat{\theta_i}, V_{i}),
\end{equation}

%(Note: for non-parametric functions, the $\theta$ values are associated with splines). \\
where $\theta$ is a length-$p$ vector representing the parameters of the function.


Once fits have been made, we are ready for testing the bootstrapped difference between curves. Once the groups of interest have been specified, two algorithms are implemented: a bootstrapping algorithm is used to determine the distribution of each group of curves, while permutation testing is used to specify regions of statistically significant differences. The algorithm for  bootstrapping for each group is as follows:

\begin{enumerate}
\item For a group of size $n$, select $n$ subjects from the group, \textit{with replacement}
\item For each selected subject, draw a set of parameters from the distribution $\theta_{i}^* \sim N(\hat{\theta}_i, V_i)$. This permits us to account for within subject variability
\item For each of the resampled $\theta_i^*$, find the $b$th bootstrap estimate for the group $\theta_b = \frac1n \sum_{i=1}^n \theta_i^*$
\item Perform this sequence $B$ times
\end{enumerate}


The end results is a $B \times p$ matrix containing a bootstrapped sample of the group distribution for $\theta$. Each row of this matrix is used to create a $1 \times T$ vector representing $f_{\theta}$ evaluted at $T$ time points. This results in a $B \times T$ matrix representing a collection of bootstrapped curves evaluated at each time point, in total representing a bootstrapped distribution of the curves.


%Each of these is used to construct a $B \times T$ matrix ($T$ the number of time points), a collection of bootstrapped curves. Each column of this matrix represents a time point, $t$, from which we can compute the mean and standard deviation of the group curve at that time (wordy). $\Leftarrow$ this also might contain too specific of detail. they don't care that its a matrix. they care for each time point we can construct an estimate of the mean and sd. And really, even this is only from CI and group distribution. There could possibly be an option to skip computing this at all and just do the permutation test.

Next we attend to idenfiying regions in which a statistically significant difference between curves is present. We begin by computing a $t$-statistic of the difference at each time point, 

\begin{equation}
T(t) = \frac{|\overline{f}_1(t) - \overline{f}_2(t)|}{\sqrt{\frac{1}{n_2} \text{Var}(f_1(t)) + \frac{1}{n_2} \text{Var}(f_2(t))}},
\end{equation}

or, in the case of paired groups, 

\begin{equation}
T(t) = \frac{\overline{f}_D(t)}{\sqrt{\frac1n \text{Var}(f_D(t))}}.
\end{equation}

Next, we go about creating a null distribution against which to test our hypothesis that there is no difference between each group at each time point. We do this with permutation testing, and the algorithm is as follows: for two groups, with $n_1$ and $n_2$ subjects in each

\begin{enumerate}
	\item Assign to each subject a label indicating group membership
	\item Randomly shuffle the lables indicating membership, creating two new groups with $n_1$ and $n_2$ subjects in each
	\item Recalculate the $t$-statistic, $T(t)$ and log the maximum
\end{enumerate}

The collection of maximum values for $T(t)$ will serve as the null distribution against which to compare our observed $T(t)$.

This notation was clearly stolen from the FDA book. 

Previous implementations of \texttt{bdots} performed this test by adjusting the nominal $\alpha$ to account for correlation between subsequent tests in time, based on previous work in Oleson. Simluations presented in [cite] suggest that this was not optimial. \texttt{bdots} does still maintain this functionality, and an estimate of the adjusted $\alpha$, along with adjusted $p$-values, can still be determined with the function, \texttt{somefun}.


---

In addition to whatever general uncertainy exists in the section above, we again now need to consider if we want each iteration of our bootstrap to be $f ( \frac1n \sum \theta_{bi})$ or $ \frac1n \sum f_i (\theta_{bi})$

\section{Example Data}

For the example data, I'm going to use the origin/vehicle/class for cars with some made up time series data. The outcome will be fake. I should then be able to tweak parameters and whatnot so that there are some differences when I want there to be. 

Committing to this now. Also committing to having to update bdotsBoot (bboot)

Maybe here choose a dataset and explain what it is, specifying the components of it so that we can use it in an ongoing example. That way (especially in bootstrap formula step) I can just say something like oh yeah Group1 values A and B and Group2 values with C and D. Or whatever. 

\section{Fitting Curves}

The curve fitting process is performed with the \texttt{bfit} function (previously \texttt{bdotsFit}), taking the following arguments: (removing `cor` and numRefits)

\begin{center}
\begin{verbatim}
bfit(data, subject, time, y, group, curveType, cores, ...)
\end{verbatim}
\end{center}

\paragraph{Curve functions} Each of \texttt{subject}, \texttt{time}, \texttt{y}, and \texttt{group} are length one character vectors representing columns of the dataset used in \texttt{data}. New here is \texttt{curveType}, taking as an argument an R call to a particular curve, for example the four parameter logistic, \texttt{logistic()}. This is done to self-contain any additional arguments associated with the fitting curve, for example the concavity of the double Gaussian (\texttt{curveType = doubleGauss(concave = TRUE)}) or the number of knots in a piecewise spline (\texttt{curveType = splines(knots = 5)}). A number of curves are included with the \texttt{bdots} package, including those for the four-parameter logistic, the double Gaussian, an exponential curve, and polynomials of arbitrary degree. A detailed vignette on writing your own curves can be found with \texttt{vignette("bdots")} ($\Leftarrow$ actually it would be vignette(``customCurves", "bdots") or browseVignette(``bdots") to see, but I haven't decided which I want because I don't really like the name customCurves).

Notably, \texttt{bdots} can now fit curves to an arbitrary number of groups, so long as all have the same parametric specification. Fitting a collection of curves to our vehicle data with all of the groups at once with logistic function would look like


\begin{center}
\begin{verbatim}
# Need to change these once I get real data
fit <- bfit(data = Vehicle, subject = "vehicle", 
	 	time = "Time", y = "out", group = c("origin", "vehicle", "color"),
	 			curveType = logistic(), cores, ...)
\end{verbatim}
\end{center}

\paragraph{Fit Codes}

\paragraph{Return object and generics}



The function \texttt{bfit} returns an object of class \texttt{bdotsObj}, inheriting from class \texttt{data.table}. As such, each row of this object (object object object i need a new word) uniquely identifies one permutation of subject group (meaning if a subject in two groups, they get two rows). Included in this row are the subject identifier, group classification, summary statistics regarding the curves, and a nested gnls object. Not sure if this is worth including, most people won't use it and i can put it in the vignette.


Several methods exist for this object, including \texttt{plot}, \texttt{summary}, and \texttt{coef}, returning a matrix of fitted coefficients returned from \texttt{gnls}. One consequence of inheriting from \texttt{data.table}, we are able to utilize data.table syntax. Note, for example, the differences between \texttt{coef(fit)}, \texttt{coef(fit[group == "A",])}, and \texttt{coef(fit[group == "B",])}. That's pretty much it on the neat stuff you can do with this. Time to go to bootstrapping step.

Actually, there is one additional thing here that I might include as an aside for now -- for part 2 of disseration, we are fitting curve to saccades instead of an ``observed" function. In this case, $R^2$ ends up being kind of a dumb/silly metric. Same can be said for auto-correlation (in terms of it being relevant). Both of these things are included in the derivation of the \texttt{fitCode}. Have not yet decided how that will be handled. 

\section{Bootstrapping}

Like the fitting function, the bootstrapping process has been consolidated to a single function, \texttt{bboot} (previously \texttt{bdotsBoot}). As we will detail, the number of options included in the \xt{bboot} function have expanded considerably. A call to \xt{bboot} takes the following form

\begin{center}
\tt bboot(formula, bdObj, Niter, alpha,   padj = "oleson", \\ cores = 0, permutation = FALSE, ...)
\end{center}

content


\subsection{Formula}

In addition to taking as an argument an object returned from \texttt{bfit}, \texttt{bboot} also takes a formula argument, with syntax unique to the \texttt{bdots} package.

The formula argument serves two functions in \xt{bfit}: first, it specifies the collection of curves we wish to investigate the difference between, and second, it determines if we are interested in directly comparing the differences or the difference of differences between curves. 

To begin, let's reintroduce the structure of the groups we have in our dataset. We have foreign and domestic cars and trucks, and each of these vehicles comes in red and blue. Recall too that the different colors of each vehicle are paired observations.

%\begin{center}
%\begin{tabular}{|p{0.6in}|p{0.3in}|p{0.9in}|p{0.9in}|p{0.8in}|} \hline 
%\multicolumn{2}{|c|}{\multirow{2}{*}{$P_\mathrm{batt}$}} &                                                   
%  \multicolumn{3}{|c|}{$P_\mathrm{mot}$ }  \\ \cline{3-5} 
%\multicolumn{2}{|c|}{}   & S & M & H \\ \cline{1-5} 
%\multirow{3}{*}{SOC} & S & S & Z & VS \\ \cline{2-5} 
% & M & M & Z & H \\ \cline{2-5} 
% & H & M & Z & H \\ \hline 
%\end{tabular}
%\end{center}

\begin{center}

\begin{tabular}{|p{0.9in}|p{0.9in}|p{0.9in}|} \hline 
\rowcolor{lightgray} \multicolumn{1}{|c|}{Origin} & \multicolumn{1}{c|}{Vehicle} & \multicolumn{1}{c|}{Color}\\
\hline
\multirow{4}{*}{foreign} & \multirow{2}{*}{car} & red \\
\hhline{~~-}
& & blue \\
\hhline{~--}
& \multirow{2}{*}{truck} & red \\
\hhline{~~-}
& & blue \\
\hline
\multirow{4}{*}{domestic} & \multirow{2}{*}{car} & red \\
\hhline{~~-}
& & blue \\
\hhline{~--}
& \multirow{2}{*}{truck} & red \\
\hhline{~~-}
& & blue \\
\hline
\end{tabular}
\end{center}

Let's begin with a simple case: supposing we want to investigate the difference in outcome between foreign and domestic cars. Notionally, we would write

\begin{center}
\tt y $\sim$ Origin(foreign, domestic).
\end{center}

Note that this involves the grouping variable, \xt{Origin}, with the two values we are interested in comparing, \xt{domestic} and \xt{foreign}. If we wanted to limit our investigation to only foreign and domestic cars, we would do this by including an extra term specifying the group and the desired value. In this case, 

\begin{center}
\tt y $\sim$ Origin(foreign, domestic) + Vehicle(car).
\end{center}
To compare only foreign and domestic \textit{red} cars, we would add an additional term for color:

\begin{center}
\tt y $\sim$ Origin(foreign, domestic) + Vehicle(car) + Color(red).
\end{center}

NOTE THAT THERE DOES NOT (IS NOT) A SIMPLE WAY TO INDICATE WE WISH TO COMPARE FOREIGN RED CARS TO DOMESTIC BLUE TRUCKS.

In each of these cases, we have specifed particular groups or nesting of groups who's outcomes we wish to compare. Alternatively, we may be interested in comparing the \textit{difference of differences}. For example, suppose we suspect that there may be some difference between red and blue vehicles, and that this difference may be different for cars compared to trucks. Whereas previously we were interested in comparing the differences in \xt{y} between groups, we are now interested in comparing the differences of \xt{y} between colors. Consequently, we will include this difference on the left hand side of the formula. This is done using \xt{diffs} syntax as such:

\begin{center}
\tt diffs(y, Color(red, blue)) $\sim$ Vehicle(car, truck)
\end{center}

Similar as to the case before, if we wanted to limit this difference of differences investigation to only include domestic vehicles, we can do so by including an additional term:

\begin{center}
\tt diffs(y, Color(red, blue)) $\sim$ Vehicle(car, truck) + Origin(domestic)
\end{center}

\subsection{Bootstrapping and Permutation}

Once we have determined the groups we are interested in comparing, we are ready to call the \xt{bboot} function. Despite it's name, \xt{bboot} is now able to perform permutation testing on curve differences in addition to bootstrapping. We will discuss each of these briefly in more detail.

\paragraph{Bootstrapping} As in the original iteration of the package, \xt{bdots} seeks to identify differences in curves without a priori specification of a time window. This is done as in (other paper), where curves are bootstrapped to create a sampling distribution, and a collection of $t$-tests is performed at  each observed timepoint. The FWER is controlled against this series of tests by making an adjustment to the nominal $\alpha$ -- the novel contribution of the original \xt{bdots} package was implementing a correction based on an estimate of the autocorrelation of tests (oleson paper). In addition to this adjustment, \xt{bdots} also allows corrections to be made using a number of other methods, including fdr, bonferonni, and I think maybe another. The type of correction made is specified with the \xt{padj} argument in \xt{bfit}.

\paragraph{Permutation} In addition to the bootstrapping algorithm just described, 
\xt{bdots} now also includes the option for running a permutation test to establish a null distribution on the differences between curves. This is done by setting the argument in \xt{bboot(..., permutation = TRUE)}.  \xt{Niter} now refers to the number of permutations performed rather than the number of bootstraps. Additionally, when \xt{permutation = TRUE} is set, the argument to \xt{padj} is ignored. Permutation testing may have less power in some cases (explored in other paper), but is maximally robust and ideal for situations in which the number of assumptions made is limited.

\subsection{Summary and Analysis}

Let's begin first by running \xt{bboot} using bootstrapping to compare the outcome between  domestic cars and trucks

\begin{center}
\tt boot <- bboot(y ~ Vehicle(car, truck) + Origin(domestic), fit)
\end{center}

This returns an object of class 

%This next part is kind of a weird aside, and I'm not sure yet where I want to put it. At least in the context of the visual world paradigm, and likely others, we find ourselves in situations in which two distinct types of difference curves are of interest: the difference between two group curves (say, A and B), and the difference of the difference between four group curves (say, the difference between the difference between condition 1 and 2 within group A and the difference between condition 1 and 2 in group B). 

%As curves for an arbitrary number of groups may be fit at once with the \texttt{bdotsFit} function, a formula syntax has been introduced to specify the differences sought. (possibly a better way to introduce/write this section) 




\section{Extensions? Plots? I don't know!}

Let's do a brief tour of some of the other additions to bdots that probably doesn't warrant its own section for use

\subsection{Non-homogenous sampling}

The \texttt{bdots} package now has support for data with non-homogenous time sampling across subjects or trials. For example, here is data collected comparing tumor growth in rates for xyz

\begin{center}
\texttt{with(tumrdata[subjects 1-4, ], plot(observations as points))}
\end{center}

It is not a problem to fit these groups and perform our bootstrapping analysis either on the union of observed time, or some custom range in between

\begin{center}
\texttt{<bfit, bboot, summary, plot> but for rats}
\end{center}

\texttt{bdots} also allows for repeated observations, as is the case with saccade data from the VWP. Here, an individual subject has 30 trials with saccades taken at the trial level. That is, rather than taking a sequence of observations for each subject, \texttt{bfit} allows for an unordered set with observations and associated time, $\mathcal{S}_i = \{(y_j, t_j)\}$ across $j$ observations. As this relates to the VWP, you can read more about his development in my dope ass other paper called chapter 2.


\subsection{Refitting}

There are sometimes situations in which the fitted function returned by \texttt{bfit} is a poor fit for some of the subjects. This can be evidenced by the \texttt{fitCode} or via a visual inspection of the fitted functions against the observations for each subject.  When this occurs, there are several options available to the user, all of which are provided through the function \texttt{brefit} (previously \texttt{bdotsRefit}). \texttt{brefit} takes the following arguments:

\begin{center}
\texttt{brefit(bdObj, fitCode = 1L, quickRefit = FALSE, numRefits = 2L, paramDT = NULL, ...)}
\end{center}

The first of these arguments, outside of the object itself, is \texttt{fitCode}, indicating  the minimum fit code to be included in the refitting process. This is a convenient way to limit the refitting process to those of a particular quality. The \texttt{quickRefit} option allows the fitter to run automatically, jittering the previous set of parameters for each refitted subject and comparing the new fit to the previous, keeping the better of the two. \texttt{numRefits} indicates how many attempts the fitter should make in doing this. Finally, \texttt{paramDT} allows for a \texttt{data.table} with columns for subject, group identifiers, and parameters to be passed in as a new set of starting parameters. This \texttt{data.table} requires the same format as that returned by \texttt{bdots::coefWriteout}.

When \texttt{quickRefit = FALSE}, the user is put through a series of prompts whereby for each subject to be refit, in addition to being given a series of diagnostics, they have the option to:

\begin{enumerate}
\item Keep the original fit
\item Jitter starting parameters
\item Adjust starting parameters manually
\item Remove AR1 assumptions (come back to this)
\item See original fit metrics again
\item Delete subject
\item Save and exit the refitter
\end{enumerate}

\begin{center}
\texttt{idk show comparison of plots for refitting i guess}
\end{center}

As the menu item suggests, users have the ability to end the manually refitting process early and save where they had left off. This looks like this

\begin{center}
\texttt{refit <- brefit(fit, ...)} \\
\texttt{refit <- brefit(refit, ...) \# pass that shit back in}
\end{center}

A final note should be said regarding the option to delete a subject. As \texttt{bdots} now automatically determines if subjects are paired based on subject identifiers (necessary for  calculations in the bootstrapping step), it is critical that if a subject has a poor fit in on group and must be removed that he or she is also removed from all subsequent groups in order to keep paired status. This can be overwritten with a final prompt in the \texttt{brefit} function before they are removed.

Additionally, this can be done without the refitter with the function \texttt{bdRemove}


\subsection{User created curves}

I know I mentioned this elsewhere, but I might erase it there and move a fuller discussion of it here

\subsection{Correlations}

There are sometimes cases in which we are interested in determining the correlation of a fixed attribute with group outcome responses across time (what such a case may be, I have no idea). This can be done with the \texttt{bcorr} function (previously \texttt{bdotsCorr}), which takes as an argument an object of class \texttt{bdotsObj} as well as a character vector representing a column from the original dataset used in \texttt{bfit}

\begin{center}
\texttt{bcorr(fit, "value", ciBands, method = "pearson")} 
\end{center}

This returns a thing that can be plotted. Idk, it really doesn't seem that important

\subsection{$\alpha$ Adjustment}

This probably last section that needs anything special, and that is an update to the \texttt{p.adjust} function, \texttt{p\_adjust}, identical to \texttt{p.adjust} except that it accepts method \texttt{"oleson"} and takes additional arguments \texttt{rho}, \texttt{df}, and \texttt{cores}. \texttt{rho} determines the autocorrelation estimate for the oleson adjustment while \texttt{df} returns the degrees of freedom used to compute the original vector of t-statistics. If an estimate of \texttt{rho} isn't available, one can be computed on a vector of t-statistics using the \texttt{ar1Solver} function:

\begin{center}
\texttt{t <- diffinv(rnorm(999))} \\
\texttt{rho <- ar1Solver(t)}
\end{center}


\subsection{Other}

Misc whatever. Good plots to show include (in general)

\begin{enumerate}
\item plots of fits
\item plots of bootstrap with ci
\item difference curve, also with sig sections highlighted
\item Correlation with fixed variables across time (\texttt{bdotsCorr})
\end{enumerate}

\section{Discussion}
I'm not really sure what to include in the discussion. We don't need to compare it to other approaches for analyzing this data, as that's aleady been done. I can point to full methodology paper to see improvements in CI coverage and difference detection/power. Reporting should be fairly simple, an $\alpha$ is given which is used to set the treshold for permutation tests -- nothing else needs to be done. The previous bdots package suggested reporting quality of individual fits that made up the bootstrapped curves, though that was based on $R^2$ and AR(1) status. The former of these is problem specific, the latter now irrelevant.

raff raff raff raff raff


Improvements made to the \texttt{bdots} package have drastically improved the ease of use of the package and the scope of the types of  problems it is able to address. The consolidation of major components into two functions has also streamlined use. Quality of life improvements include multiple group fitting, formula syntax for bootstraps, tractable return objects, and others. Generics have also been good. The package is also now statistically correct. It has extended the types of data that can be accomodated, including heterogenous observations across time and the ability to construct user-specified curves. it really is a pretty neat package.

\end{document}






