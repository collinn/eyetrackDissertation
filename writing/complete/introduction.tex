

This dissertation is made up of three chapters, each addressing some aspect of eye tracking in the Visual World Paradigm (VWP) or \xt{bdots}, the R software created to accompany the analysis. Each of these chapters is written as a separate manuscript. As such, there will be some overlap and redundancy between chapters, though this is intentional as they are intended to stand alone. Although they are presented necessarily in a linear fashion, they can be read in any order.

The first chapter addresses major changes in the \xt{bdots} package, an implementation of the ``bootstrapped differences in time series" methodology first introduced in \citet{oleson2017detecting}. The package offers a user-friendly way for fitting parametric functions to subject-specific time series data along with a bootstrapping procedure for identifying temporal differences between experimental groups. Major updates in this reiterated version include a simplification to the user interface, increased functionality, and general quality-of-life improvements. Included also is a use-case example performing an analysis with non-VWP data: an analysis of tumor growth in mice across a number of experimental treatment groups. This chapter also includes important changes to the underlying methodology, including a change to the way the original bootstrapping algorithm is constructed, as well as the introduction of a permutation test to control the family-wise error rate. These methods are included both for completeness for new users and to demonstrate important changes for existing users. Justification for these changes is provided in Chapter 4. 

The second chapter addresses the Visual World Paradigm itself. It begins with a general review of the VWP and eye tracking, and how these are used in relation to one another to assess lexical activation in time. This chapter examines critical issues in the current ``proportion of fixation" method, presents a more comprehensive model for relating the underlying cognitive mechanisms of interest with the observed physiological behavior, and introduces a novel ``look onset" method for using data in the VWP. We justify our approach both with a theoretical argument as to why this method would be superior for the recovery of the underlying cognitive mechanisms and verify the veracity of our claims with a number of simulations. We conclude this chapter by suggesting a number of ways this new method can be incorporated into current theories of lexical activation.

Finally, the last chapter of this dissertation addresses the underlying methodology presented in the motivating methodological paper for the \xt{bdots} package \citep{oleson2017detecting}. Specifically, this paper introduced a novel correction to highly correlated test statistics constructed via bootstrapping from densely sampled time series data. We identify a critical issue in the underlying assumptions as they relate to empirically observed data and the impacts on the family-wise error rate when these assumptions do not hold. Presented in this chapter is a modification of the original bootstrapping algorithm that still uses the modified Bonferonni correction based on the autocorrelation of the test statistics in time, as well as the introduction of a permutation test for identifying temporal differences. We demonstrate that both the modified bootstrap and the permutation test maintain a FWER close to the nominal rate across a number of differing assumptions and conclude with a comparison of power for each of the methods discussed. The results of Chapter 4 provide justification for the underlying changes in Chapter 2. 

All of the code used in this dissertation is available at github.com/collinn.


