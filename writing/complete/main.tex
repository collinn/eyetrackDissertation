\documentclass[12pt]{report}

%% Thesis style file
\usepackage{uithesis}

%% Other packages
%% These are optional, depending on what is in your thesis.
%% However, assuming that you have equations, figures, and tables,
%% I would recommend including all of them.
%% This may require installation of missing packages
\usepackage{amsmath}
\usepackage{amssymb}
\usepackage{amsthm}
\usepackage{graphicx}
\graphicspath{{fig/}}
\usepackage{natbib}
\usepackage{floatrow}
\usepackage{caption}


\usepackage{multirow}
\usepackage{hhline}
%\usepackage{graphicx}
\usepackage{fancyvrb}
\usepackage{float}
\usepackage[table]{xcolor}
\usepackage{xcolor}
\usepackage{lscape} 
%\usepackage{datetime}
%
\usepackage{color}
%\providecommand{\pb}[1]{\textcolor{red}{#1}}
%\providecommand{\cn}[1]{\textcolor{blue}{#1}}

\providecommand{\pb}[1]{}
\providecommand{\cn}[1]{}              
\providecommand{\cmt}[1]{\textcolor{purple}{#1}}

%\usepackage{setspace}
%\doublespacing
%
\usepackage[margin=1in]{geometry}
\usepackage{amsmath}
\usepackage{graphicx}
\usepackage{datetime}
\usepackage{float}
\usepackage{txfonts}


\usepackage{pdfpages}

\DeclareMathOperator*{\argmax}{arg\!\max}
\DeclareMathOperator*{\argmin}{arg\!\min}

\usepackage{subfigure}
\graphicspath{{img/}}

%\bibliographystyle{plain}
\bibliographystyle{unsrtnat}
\newcommand{\xt}{\texttt}% \xt will be replaced with \textit

%\usepackage{listings}

\begin{document}

% This file specifies information specific to your thesis, such as your
% title, advisor, dedication, etc.


% To remove optional components, comment out the line
\abtitlepgtrue
\abstractpgtrue
\titlepgtrue
\copyrighttrue %(optional)
%\signaturepagetrue %remove signature page
\acktrue %(optional)
\tablecontentstrue
\tablespagetrue
\figurespagetrue

\title{What You See is What You Get: A Closer Look at Bias in the Visual World Paradigm}
\author{Collin Nolte}
\advisor{Professor Patrick Breheny}
\dept{Biostatistics}
\submitdate{May 2023}
\supervisor{Patrick Breheny}
\membera{Jacob Oleson}
\memberb{Bob McMurray}
\memberc{Grant Brown}
\memberd{Kristi Hendrickson}


\newcommand{\abstextwithesis}
{ % main abstract
In 1995 the Visual World Paradigm was first introduced as an experimental paradigm relating eye-tracking data to lexical activation. This was done by measuring the location of a participant's visual fixation in real time in response to a spoken word. Shortly following this was the introduction of the ``proportion of fixations" method, whereby indicators of fixations in the VWP were aggregated across trials at dense time slices, creating an ostensible curve demonstrating the proportion of trials in which participants were fixated on particular objects at each time. In 2017, methods were introduced to make temporal analysis of these curves tractable via bootstrapping and a novel $\alpha$ correction to counteract the multiple comparison problems implicit in densely sampled time series. 

This dissertation improves upon the field in multiple ways. First, we reintroduce the \xt{bdots} package with a dramatically simplified user interface. Most prominent among these changes include the ability for users to create and fit parametric functions independent the \xt{bdots} software as well as the introduction of permutation testing for identifying temporal differences between groups. We also identify and correct methodological issues found in the original bootstrapping algorithm that, when unaccounted for, potentially lead to family-wise error rates of over 90\%. Both this correction and the newly introduced permutation test demonstrate quality maintenance of the FWER across a robust collection of underlying assumptions without significant losses in power. 

And finally, we propose a new generative model that links eye mechanics to lexical activation, along with a novel ``look onset" method that seeks to replace the proportion of fixations method. We demonstrate asymptotic consistency between our generative model and the look onset method, both with regards to the recovery of subject-specific activation curves, as well as with the identification of systematic temporal differences between groups. We conclude by demonstrating this utility with data collected from prior studies as well as by providing a number of avenues for future research. 
}



 
\newcommand{\acknowledgement}
{
\cmt{acknowledgement}
}

\newcommand{\pubabstextwithesis}
{ % public abstract
In 1995 the Visual World Paradigm was first introduced as an experimental paradigm relating eye-tracking data to lexical activation. This was done by measuring the location of a participant's visual fixation in real time in response to a spoken word. Shortly following this was the introduction of the ``proportion of fixations" method, whereby indicators of fixations in the VWP were aggregated across trials at dense time slices, creating an ostensible curve to help visualize and quantify the time course of lexical activation. In 2017, methods were introduced to make temporal analysis of these curves tractable via bootstrapping and a novel $\alpha$ correction to counteract the multiple comparison problems implicit in densely sampled time series. 

This dissertation contributes to the field in multiple ways. First, we reintroduce the \xt{bdots} package that significantly enhances researchers' ability to draw conclusions from VWP studies. Accompanying this are drastic changes to the underlying methodology to accommodate a far more robust set of assumptions regarding data generation. And finally, we introduce a generative model for relating eye mechanics to lexical activation alongside a novel look onset method that seeks to replace the proportion of fixation method currently in use.
}

\beforepreface
\afterpreface






















\Chapter{Introduction}


Last compiled: \today \  at \currenttime

This dissertation is made up of three chapters, each addressing some aspect of eye tracking in the Visual World Paradigm (VWP) or \xt{bdots}, the R software created to accompany the analysis. Each of these chapters is also written as a separate manuscript. As such, there will be some overlap and redundancy between chapters, though this is intentional as they are intended to stand alone. Although they are presented necessarily in a linear fashion, they can be read in any order.

The first chapter addresses major changes in the \xt{bdots} package, an implementation of the ``bootstrapped differences in time series" methodology first introduced in \citet{oleson2017detecting}. The package offers a user-friendly way for fitting parametric functions to subject-specific time series data along with a bootstrapping procedure for identifying temporal differences between experimental groups. Major updates in this reiterated version include a simplification to the user interface, increased functionality, and general quality-of-life improvements. Included also is a use-case example performing an analysis with non-VWP data: an analysis of tumor growth in mice across a number of experimental treatment groups. This chapter also includes important changes to the underlying methodology, including a change to the way the original bootstrapping algorithm is constructed, as well as the introduction of a permutation test to control the family wise error rate. These methods are included both for completeness for new users and to demonstrate important changes for existing users. Justification for these changes is provided in Chapter 4. 

The second chapter addresses the Visual World Paradigm itself. It begins with a general review of the VWP, eye tracking, and how these are used in relation to one another to assess lexical activation in time. This chapter examines critical issues in the current ``proportion of fixation" method, presents a more comprehensive model for relating the underlying cognitive mechanisms of interest with the observed physiological behavior, and introduces a novel ``look onset" method for using data in the VWP. We justify our approach both with a theoretical argument as to why this method would be superior for the recovery of the underlying cognitive mechanisms and verify the veracity of our claims with a number of simulations. We conclude this chapter by suggesting a number of ways this new method can be incorporated into current theories of lexical activation.

Finally, the third chapter of this dissertation addresses the underlying methodology presented in the the motivating methodological paper for the \xt{bdots} package \citep{oleson2017detecting}. Specifically, this paper introduced a novel correction to highly correlated test statistics constructed via bootstrapping from densely sampled time series data. We identify a critical issue in the underlying assumptions as they relate to empirically observed data and the impacts on the family-wise error rate when these assumptions do not hold. Presented in this chapter is a modification of the original bootstrapping algorithm that still uses the modified Bonferonni correction based on the autocorrelation of the test statistics in time, as well as the introduction of a permutation test for identifying temporal differences. We demonstrate that both the modified bootstrap and the permutation test maintain a FWER close to the nominal rate across a number of differing assumptions and conclude with a comparison of power for each of the methods discussed. The results of Chapter 4 provide justification for the underlying changes in Chapter 2. 

All of the code used in this dissertation is available at github.com/collinn.




\newpage

\Chapter{Reintroduction of the \xt{bdots} Package}


%\begin{abstract}
abstract
\end{abstract}

\section{Introduction}
intro

\paragraph{Outline}
outline



\begin{abstract}
The Bootstrapped Differences of Timeseries (bdots) was first introduced by Oleson (and others) as a method for controlling type I error in a composite of serially correlated tests of differences between two time series curves in the context of eye tracking data.  This methodology was originally implemented in R by Seedorff 2018. Here, we revist the original package, both expanding the underlying theoretical components and creating a more robust implementation.
\end{abstract}


\section{Introduction}

i hate introductions we will do this part last

In 2017, Oleson et al. introduced a method for detecting time-specific differences in the trajectory of outcomes between experimental groups. Particularly in the case of a densely sampled time series, the construction of evaluating differences at each point in time results in a series of highly correlated test statistics expanding the family-wise error rate, accommodated with an adjustment to the nominal alpha based on this autocorrelation. This was followed up with in 2018 with the introduction of the \xt{bdots} package to CRAN \cite{seedorff2018bdots}. Here, we introduce the second version of \texttt{bdots}, an update to the package that broadly expands the capabilities of the original. 

This manuscript is not intended to serve as a complete guide for using the \xt{bdots} package. Instead, the purpose is to showcase major changes and improvements, with those seeking a more comprehensive treatment directed to the package vignettes. Rather than taking a ``compare and contrast" approach, we will first enumerate the major changes, followed by a general demonstration of the package use:

\begin{singlespace}
\begin{enumerate}
\item Major changes to underlying methodology with implications for prior users of the package
\item Simplified user interface
\item Introduction of user defined curves
\item Permit fitting for arbitrary number of groups
\item Automatic detection of paired tests based on subject identifier
\item Allows for non-homogeneous sampling of data across subjects and groups
\item Introduce formula syntax for bootstrapping difference function
\item Interactive refitting process
\end{enumerate}
\end{singlespace}

First, a quick review of framing the problem at hand.

We start by clearly delineating the type of problem that \xt{bdots} has been created to solve.

\paragraph{Bootstrapped differences in time series}

We begin by assuming two or more experimental groups in which a subject response is measured over time. This may include the growth of tumors in  mice or the change in the proportion of fixations over time in the context of the VWP. In either case, we assume that each of the subjects in the groups being considered has observed data of the following form:

\begin{equation}
y_{it} = f_{\theta_i}(t) + \epsilon_{it} 
\end{equation}
OR (as was written in the original)
\begin{equation}
y_{it} = f(\theta_{it}) + \epsilon_{it} 
\end{equation}

where $f$ represents a functional mean structure, while the error structure of $\epsilon_{it}$ is open to be either IID or possess an AR(1) structure. At present, \xt{bdots} requires that each of the subjects being compared have the same parametric function $f$, this is not strictly necessary and future directions of the package include accommodating non-parametric functions. While each of the subjects are required to be of the same parameteric form $f_{\theta}$, each differs in their instance of their own subject-specific parameters, $\theta_i$.

The collection of subjects' $\theta$  within a group constitutes a group distribution -- bootstrapping parameters from this distribution and evaluating functions $f$ at these values gives us an estimate of the distribution of functions. As these functions are \textit{in time}, this in turn gives a representation of temporal changes in group characteristics. It is precisely the identification of how these temporal changes differ between groups that \xt{bdots} seeks to [do?]

This differs from the original iteration of \xt{bdots} (or Oleson 2017) in a critical way, however. Idk how to say this, really, but the original assumed that there was no variability between subject parameters (which I am calling the homogeneity of means assumption), in which $\theta_i = \theta_j$ for all $i,j$ in an experimental group. The primary consequence of this was the absence of any need to estimate the variability within a group, taking into consideration only subject-specific variability (don't want to go into a lot of details because I kind of treat this mathematically in the next section). Anyways, see Chapter 3 for all the deets on this. Finally, a full review of the context in which it was introduced is available in Seedorff et. al., 2018 \cite{seedorff2018bdots}.

[transition paragraph]

I really need to decide how much detail to go into here. I don't want to repeat everything from chapter 3, and even though this doesn't have to be a stand alone paper now, it will be and when it is this will be pretty damn important to convey, especially to those who used bdots previously.

[also note: however we decide to deal with dissemination of this information, and while I think 2.0.0 should definitely be released with all of this, I almost htink we should address the fitcode awkwardness before this is presented to the public]

Alrightly, on to taking a looksee at how bdots works


\section{Methodology and Overview} 

A standard analysis using \xt{bdots} consists of two steps: fitting the observed data to a specified parametric function, $f_\theta$, and then using the observed variability to construct estimates of the distributions of each groups' curves.  Here, we briefly detail how this is implemented in practice and introduce the new methodologies in \xt{bdots v2}. A more comprehensive treatment of these new methods, along with their justifications, is offered in Chapter 3. 



\subsection{Establishing subject-level curves}

We begin with the assumption that for subject $i$, the observed data is of the form

\begin{equation}\label{eq:mean_structure}
y_{it} = f_{\theta_i}(t) + \epsilon_{it},
\end{equation}
where $\epsilon_{it}$ can be specified to be either independent or have an AR(1) auto-correlation structure. Each subject is fit in \xt{bdots} via \xt{gnls}, returning an estimated set of parameters and their associated standard errors. Assuming large sample normality, we are able to construct a sampling distribution for each subject, accounting for within-subject variability. This gives us for each subject a distribution

\begin{equation}
\hat{\theta}_i \sim N(\theta_i, s_i^2).
\end{equation}

\subsection{Estimating Group Distributions}\label{sec:group_dist}

Once sampling distributions are created for each subject, we are prepared to begin estimating group distributions. We assume that the mean parameters for each subject come from a group level distribution, where for each subject $i$,

\begin{equation}
\theta_i \sim N(\mu_{\theta}, V_{\theta}).
\end{equation}

This is notably in contrast to the original implementation of \xt{bdots}; there, the authors proceeded with an assumption of homogeneity of means, with $\theta_i = \theta_j$ for each $i, j$. A more thorough investigation of this, along with justification for the current method, is presented in Chapter 3.

This in hand, we go about estimating the group distributions according to the following procedure: (not in love with the notation here).

\begin{singlespace}
\begin{enumerate}
\vspace{-3mm}
\item For a group of size $n$, select $n$ subjects from the group \textit{with replacement}. This allows us to construct an estimate of $V_{\theta}$.
\item For each selected subject $i$, draw a set of parameters from the distribution $\theta_{i}^* \sim N(\hat{\theta}_i, s_i^2)$. This gives us an accounting of the observed within-subject variability
\item For each resampled $\theta_i^*$, find the $b$th bootstrap estimate for the group,.
\begin{equation}
\theta_b^* = \frac1n \sum_{i=1}^n \theta_i^*, \quad \text{where} \quad \theta_b^* \sim N  \left(\mu_{\theta}, \frac1n V_{\theta} + \frac{1}{n^2}\sum s_i^2 \right).
\end{equation}
\item Perform steps (1.)-(3.) $B$ times, using each $\theta_b^*$ to construct a distribution of population curves, $f_{\theta}(t)$.
\end{enumerate}
\end{singlespace}

The final population curves from (4) can be used to create estimates of the mean response and an associated standard deviation at each time point for each of the groups bootstrapped. These estimates are used both for plotting and in the construction of confidence intervals. They also can be, but do not necessarily have to be, used to construct a test statistic, which is the topic of our next section.

\subsection{Hypothesis testing for statistically significant differences}

We now turn our attention to the primary goal of an analysis in \xt{bdots}, identifying time windows in which the distribution of curves of two groups differ significantly. A problem unique to the ones address by \xt{bdots} is that of multiple testing; and especially in densely sampled time series, we must account for multiple testing while controlling the family-wise error rate (FWER). Version 2 of \xt{bdots} provides two ways with which this can be accomplished.

\subsubsection{Oleson Adjustment}

Just as in the original iteration of \xt{bdots}, we are able to construct test statistics from the bootstrapped estimates described in the previous section. These test statistics $T(t)$ can be written as 

\begin{equation}\label{eq:test_statistic}
T(t) = \frac{(\overline{f}_{1}(t) - \overline{f}_{2}(t))}{\sqrt{\frac{1}{n_1} \text{Var}(f_1(t)) + \frac{1}{n_2} \text{Var}(f_2(t))}},
\end{equation}
where [$\dots$] I actually don't like that notation at all, but I can't use $\overline{p}_{1t}$ and $s_{1t}^2$, really, because I already used $s_i^2$ for within-subject variance. Whatever, pin in this for now, pretend I carefully described what all of the components of this test statistic is for now.

To account for the multiple testing problem with autocorrelated test statics, we construct a nominal significance level $\alpha^*$ to produce the desired effective FWER $\alpha$. For now I won't give any more details, it basically finds some autocorrelation parameter and uses this to create a new alpha.  We then use this to determine which $T(t) < z{1 -\alpha^*/2}$, giving us our significant regions. neato.

\subsubsection{Permutation testing}

Alternatively, \xt{bdots} provides a modified permutation test for controlling the FWER without any additional assumptions on the autocorrelated status of the errors or test statistics. 

We begin by creating test statistics at each time point, similar to Equation~\ref{eq:test_statistic}. Using the fitted parameters $\hat{\theta}_i$ for each subject $i$, we construct subject-specific curves $f_{\hat{\theta}_i}$ and use \textit{these} to construct population mean and standard deviations at each time point, giving population curves and standard deviations [$\dots$] (i still don't have notation i really like for this, especially in contrast to the bootstrapping step). We use these to create the observed test statistic

\begin{equation}
T_p(t) = [\dots] 
\end{equation}
(this is basically the same formula as Equation 6 but with absolute value and having $\overline{f}$ mean different things. whatever notation is used)

When then going about using permutations to construct a null distribution against which to compare the observed statics. We do so with the following algorithm:

\begin{enumerate}
\item Assign to each subject a label indicating group membership
\item Randomly shuffle the labels assigned in (1.), creating two new groups 
\item Recalculate the test statistic $T_p(t)$, recording the maximum value
\item Repeat (2.)-(3.) $P$ times
\end{enumerate}

The collection of maximum values for $T(t)$ will serve as the null distribution against which to compare our observed $T(t)$. Regions in which the observed $t$ statistic are beyond the specified $\alpha$ in the null distribution are then considered significant.

\subsubsection{Odds and Ends}

Both of the methods presented are able to accommodate paired assumptions with minor adjustments to their algorithms. In the case of the bootstrap, we simply must take care to ensure that for each iteration, the collection of subjects sampled in one group with replacement are also sampled in the other, ensuring that each bootstrapped estimate comes from the same distribution. Likewise, paired testing is implemented through permutation testing by modifying the shuffling process so that each subject has one set of observations in each of the permuted groups.

Finally, it is of note that other adjustments for FWER are offered here as were in the original implementation of \xt{bdots}, including all of the adjustments present in the \xt{p.adjust} function from the \xt{stats} R package. I say this for completeness, but idk that anybody cares. Okie dokie, then, thems the methods! On to an example analysis.


\section{Example Analysis}

In this next section we are going to review a worked example of a typical use of the \xt{bdots} package. We will use as our illustration a study (source?) comparing tumor growth for the 451LuBr cell line in mice data with repeated measures in five treatment groups.

\begin{singlespace}
\begin{figure}
\centering
\begin{BVerbatim}
> head(mouse, n = 10)
      Volume Day Treatment ID
 1:   47.432   0         A  1
 2:   98.315   5         A  1
 3:  593.028  15         A  1
 4:  565.000  19         A  1
 5: 1041.880  26         A  1
 6: 1555.200  30         A  1
 7:   36.000   0         B  2
 8:   34.222   4         B  2
 9:   45.600  10         B  2
10:   87.500  16         B  2
\end{BVerbatim}
\caption{Illustration of Mouse data}
\label{fig:mouse_head}
\end{figure}
\end{singlespace}

Note that in Figure~\ref{fig:mouse_head}, the \xt{Day} variable contains different values between two mice (indicated by \xt{ID}). This reflects a new feature of \xt{bdots}, which is the ability to fit and analyze subjects with non-homogenous time samples. Details on how to adjust how non-homogenous times are handled in a later section (they're not -- I'm not sure what to do abou this yet when bootstrapping functions that are FAR outside of their observed range. I almost think the default should be a mini-max approach, creating an arbitrary range of values in the intersection of the range of all of the observed data, but with ability to specify time range directly in bootstrap function).

There are two primary functions in the \xt{bdots} package: one for fitting the observed data to a parametric function and another for estimating group distributions and identifying time windows where they differ significantly. The first of these, \xt{bfit}, is addressed in the next section.


\subsection{Curve Fitting}

The curve fitting process is performed with the \texttt{bfit} function (previously \texttt{bdotsFit}), taking the following arguments:


\begin{figure}[h!]
\centering
\begin{BVerbatim}
bfit(data, subject, time, y, group, curveType,, ...)
\end{BVerbatim}
\caption{Main arguments to \xt{bfit}, though see \xt{help(bfit)} for additional options}
\end{figure}


(need to add/talk about an AR1 argument that fits data with this assumption or not. Really this AR1 thing is a mess. I think for the dissertation, outside of new methodology we just pretend that chapter 3 doesn't exist)

The \xt{data} argument takes the name of the dataset being used, which should be stored in long format. \xt{subject} is the subject identifier column in the data and should be passed as a character. It is important to note here that the identification of paired data is done automatically; in determining if two experimental groups are paired, \xt{bdots} checks that the intersection of subjects in each of the groups are identical with the subjects in each of the groups individually. The \xt{time} and \xt{y} arguments are column names of the time variable and outcome, respectively. Similarly, \xt{group} takes as an argument a character vector of each of the group columns that are meant to be fit, accommodating the fact that \xt{bdots} is now able to fit an arbitrary number of groups at once, provided that the outcomes in each group adopt the same parametric form. This brings us to the \xt{curveType} argument, which is addressed in the next section.


\paragraph{Curve functions} Whereas the previous iteration of \xt{bdots} had a separate fitting function for each parametric form (i.e., \xt{logistic.fit} for fitting data to a four-parameter logistic), we are now able to specify the curves we wish to fit independent of the fitting function \xt{bfit}. This is done with the \xt{curveType} argument. Unlike the previous arguments which took either a \xt{data.frame} or character  vector, \xt{curveType} takes as an argument a function call, for example, \xt{logistic()}. The motivation for this is detailed elsewhere (the appendix, maybe?), but in short, it allows the user to pass additional arguments to further specify the curve. For example, among the parametric functions included in \xt{bdots} is now the \xt{polynomial} function, taking as an additional argument the number of degrees we wish to use. The fit the observed data with a five parameter polynomial in \xt{bfit}, one would then pass the argument \xt{curveType = polynomial(degree = 5)}. Literal magic takes care of the rest. Curve functions currently included in \xt{bdots} include \xt{logistic()}, \xt{doubleGauss()}, \xt{expCurve()}, and \xt{polynomial()}. \xt{bfit} can also accept user-created curves; detailed vignettes for writing your own can be found with \xt{vignette("bdots")}. (maybe i'll show going from logistic to logistic with distribution for custom functions)

We fit the mouse data to an exponential curve with \xt{expCurve()} and using the column names found in Figure~\ref{fig:mouse_head}:

\begin{singlespace}
\begin{figure}[H]
\centering
\begin{BVerbatim}
mouse_fit <- bfit(data = mouse, subject = "ID", time = "Day", 
                  y = "Volume", group = "Treatment", curveType = expCurve())
\end{BVerbatim}
%\caption{Fitting mouse data with \xt{bfit} -- should I caption these?}
\label{fig:bfit_example}
\end{figure}
\end{singlespace}


\paragraph{Return object and generics}


The function \texttt{bfit} returns an object of class \texttt{bdotsObj}, inheriting from class \texttt{data.table}. As such, each row uniquely identifies one permutation of subject and group values. Included in this row are the subject identifier, group classification, summary statistics regarding the curves, and a nested \xt{gnls} object. 

\begin{singlespace}
\begin{figure}[H]
\centering
\begin{BVerbatim}
> class(mouse_fit)
[1] "bdotsObj"   "data.table" "data.frame"

> head(mouse_fit)
   ID Treatment        fit      R2   AR1 fitCode
1:  1         A <gnls[18]> 0.97349 FALSE       3
2:  2         B <gnls[18]> 0.83620 FALSE       4
3:  3         E <gnls[18]> 0.96249 FALSE       3
4:  4         C <gnls[18]> 0.96720 FALSE       3
5:  5         D <gnls[18]> 0.76156 FALSE       5
6:  7         B <gnls[18]> 0.96361 FALSE       3
\end{BVerbatim}
\caption{A \xt{bfit} object inheriting from \xt{data.frame}}
\label{fig:bdotsObj}
\end{figure}
\end{singlespace}

The number of columns will depend on the total number of groups specified, with the subject and group identifiers always being the first columns. Following this is the \xt{fit} column, which contains the fitted object returned from \xt{gnls}, as well as \xt{R2} indicating the $R^2$ statistic. The \xt{AR1} column indicates whether or not the observed data was able to be fit with an AR(1) error assumption. Finally, there is the \xt{fitCode} column, which we will describe in more detail shortly.

Several methods exist for this object, including \texttt{plot}, \texttt{summary}, and \texttt{coef}, returning a matrix of fitted coefficients obtained from \texttt{gnls}. 


\paragraph{Fit Codes}\label{sec:fitcode}

The \xt{bdots} package was originally introduced to address a very narrow scope of problems, and the \xt{fitCode} designation is an artifact of this original intent. Specifically, it assumed that all of the observed data was of the form given in Equation~\ref{eq:mean_structure} where the observed time series was dense and the errors were autocorrelated. Autocorrelated errors can be specified in the \xt{gnls} package (used internally by \xt{bdots}) when generating subject fits, though there were times when the fitter would be incapable of converging on  a solution. In that instance, the autocorrelation assumption was dropped and constructing a fit was reattempted.

$R^2$ proved a reliable metric for this kind of data, and preference was given to fits with an autocorrelated error structure over those without. From this, the hierarchy given in Table~\ref{tab:fit_codes} was born. \xt{fitCode} is a numeric summary statistic ranked from 0 to 6 detailing information about the quality of the fitted curve, constructed with the following pseudo-code:

\begin{singlespace}
\begin{figure}[H]
\centering
\begin{BVerbatim}
  AR1 <- # boolean, determines AR1 status of fit
  fitCode <- 3L*(!AR1) + 1L*(R2 < 0.95)*(R2 > 0.8) + 2L*(R2 < 0.8)
\end{BVerbatim}
%\caption{Illustration of Mouse data}
\end{figure}
\end{singlespace}

A fit code of 6 indicates that \xt{gnls} was unable to successfully fit the 
subject's data. 

\xt{bdots} today stands to accommodate a far broader range of data for which the original \xt{fitCode} standard may no longer be reliable. The presence of autocorrelation cannot always be assumed, and users may opt for a metric other than $R^2$ for assessing the quality of the fits. Even the assessments of fits on a discretized scale may be something of only passing interest. Even then, however, this is how the current implementation of \xt{bdots} categorizes the quality of its fits, with the creation of greater flexibility in this regard being a large priority for future directions. Outside of general summary information, the largest impact of this system is in the refitting process, which organizes the fits by \xt{fitCode}. There is still flexibility in how this is handled, though we will reserve that for the relevant section. (i hate this whole section).

\begin{singlespace}
\begin{table}[H]
\centering
\def\arraystretch{1.5}
\begin{tabular}{|c|c|c|}
\hline
\xt{fitCode} & AR(1) & $R^2$ \\
\hline
0 & TRUE & $R^2 > 0.95$ \\
1 & TRUE & $0.8 < R2 < 0.95$ \\
2 & TRUE & $ R^2 <0.8$ \\
3 & FALSE & $R^2 >0.95$ \\
4 & FALSE & $0.8 < R2 < 0.95$ \\
5 & FALSE &$ R^2 <0.8$  \\
6 & NA & NA \\
\hline
\end{tabular}
\caption{fit codes, though less relevant for other types of data so idk really what to do about it. nothing for the dissertation, at least}
\label{tab:fit_codes}
\end{table}
\end{singlespace}


\paragraph{Plots and summaries}

Users are able to quickly summarize the quality of the fits with the \xt{summary} method now provided. 

\begin{singlespace}
\begin{figure}[H]
\centering
\begin{BVerbatim}
> summary(mouse_fit)

bdotsFit Summary

Curve Type: expCurve 
Formula: Volume ~ x0 * exp(Day * k) 
Time Range: (0, 106) [31 points]

Treatment: A 
Num Obs:  10 
Parameter Values: 
        x0          k 
172.232953   0.056843 
########################################
############### FITS ###################
########################################
AR1,       0.95 <= R2        -- 2 
AR1,       0.80 < R2 <= 0.95 -- 1 
AR1,       R2 < 0.8          -- 0 
Non-AR1,   0.95 <= R2        -- 0 
Non-AR1,   0.8 < R2 <= 0.95  -- 3 
Non-AR1,   R2 < 0.8          -- 4 
No Fit                       -- 0 

[...]

All Fits 
Num Obs:  42 
Parameter Values: 
        x0          k 
102.487118   0.053662 
########################################
############### FITS ###################
########################################
AR1,       0.95 <= R2        -- 4 
AR1,       0.80 < R2 <= 0.95 -- 2 
AR1,       R2 < 0.8          -- 0 
Non-AR1,   0.95 <= R2        -- 9 
Non-AR1,   0.8 < R2 <= 0.95  -- 16 
Non-AR1,   R2 < 0.8          -- 11 
No Fit                       -- 0 
\end{BVerbatim}
\caption{Abridged output from the summary function (missing summary information for groups B-E. Note that this includes data on the formula used, the quality of fits and mean parameter estimates by group, and a summary of all fits combined}
\end{figure}
\end{singlespace}

It is also recommended that users visually inspect the quality of fits for their subjects, which includes a plot of both the observed and fit data. There are a number of options available in \xt{?plot.bdotsObj} (they can call it with just plot but you look for help with this idk), including the option to fit the plots in base R rather than \xt{ggplot2}. This is especially helpful when looking to quickly assess the quality of its (rather than reporting) because \xt{ggplot2} is notoriously slow, and especially when you have 400 subjects. I haven't \textit{actually} done this yet but will shortly. Anyways, check out Figure~\ref{fig:plot_fits} for a plot of the first four fitted subjects.


\begin{singlespace}
\begin{figure}[H]
\centering
\begin{BVerbatim}
plot(mouse_fit[1:4, ])
\end{BVerbatim}
\end{figure}
\end{singlespace}

\begin{figure}[H]
\centering
\includegraphics[width=0.9\textwidth]{img/mouse_fit.pdf}
\caption{Plot of \xt{mouse\_fit}, using \xt{data.table} syntax to subset to only the first four observations}
\label{fig:plot_fits}
\end{figure}

\subsection{Bootstrapping}

Once fits have been made, we are ready to begin estimating the group distributions. This is done with the bootstrapping function, \xt{bboot}. The number of options included in the \xt{bboot} function have expanded to include a new formula syntax for specifying the analysis of interest as well as to include options for permutation testing. A call to \xt{bboot} takes the following form

\begin{singlespace}
\begin{figure}[H]
\centering
\begin{BVerbatim}
bboot(formula, bdObj, alpha, permutation = TRUE, padj = "oleson", ...)
\end{BVerbatim}
\caption{should i caption this? It feels naked without and it feels too ritzy with}
\end{figure}
\end{singlespace}

The \xt{formula} argument is new to version 2 of \xt{bdots} and will be discussed in the next section. As for the remaining arguments, \xt{bdObj} is simply the object returned from \xt{bfit} that we wish to investigate, and \xt{B} serves the dual role of indicating the number of bootstraps/permutations we wish to perform. \xt{alpha} is the rate at which we wish to control the FWER. \xt{permutation} and \xt{padj} work in contrast to one another; when \xt{permutation = TRUE} (the default?), the argument to \xt{padj} is ignored. Otherwise, \xt{padj} indicates the method to be used in adjusting the nominal \xt{alpha} to control the FWER. By default, \xt{padj = "oleson"}. Finally, as previously mentioned, there is no longer a need to specify if the groups are paired, and \xt{bboot} determines this automatically based on the subject identifiers in each of the groups.


\subsubsection{Formula}

As the \xt{bfit} function is now able to create fits for an arbitrary number of groups at once, we rely on a formula syntax in \xt{bboot} to specify precisely which groups differences we wish to compare. Let\xt{y} designate the outcome variable indicated in the \xt{bfit} function and let \xt{group} be one of the group column names to which our functions were fit. Further, let \xt{val1} and \xt{val2} be values of two of the groups in that same column. The general syntax for the \xt{bboot} function takes the following form:

\begin{singlespace}
\begin{figure}[H]
\centering
\begin{BVerbatim}
y ~ group(val1, val2)
\end{BVerbatim}
%\caption{Illustration of Mouse data}
\end{figure}
\end{singlespace}

Note that this is an \textit{expression} in R and is written without quotation marks as in a character vector. To give a more concrete example, suppose we wished to compare the difference in tumor growth curves for A and B from the \xt{Treatment} column in our mouse data (Figure~\ref{fig:mouse_head}). We would do so with the following syntax:

\begin{singlespace}
\begin{figure}[H]
\centering
\begin{BVerbatim}
Volume ~ Treatment(A, B)
\end{BVerbatim}
%\caption{Illustration of Mouse data}
\end{figure}
\end{singlespace}

There are two special cases to consider when writing this syntax. The first is the situation that arises in the case of multiple or nested groups, the second when a difference of difference analysis is conducted. Details on both of these cases are handled in the appendix. 



\subsubsection{Summary and Analysis}

[what gets a paragraph, what gets a subsubsection?  compare this with \textit{\textbf{subsection}} for fitting]

Let's begin first by running \xt{bboot} using bootstrapping to compare the difference in tumor growth between treatment groups A and E in our mouse data using permutations to test for regions of significant difference. 

\begin{singlespace}
\begin{figure}[H]
\centering
\begin{BVerbatim}
mouse_boot <- bboot(Volume ~ Treatment(A, E), bdObj = mouse_fit, permutation = TRUE)
\end{BVerbatim}
%\caption{Illustration of Mouse data}
\end{figure}
\end{singlespace}


This returns an object of class \xt{bdotsBootObj}. A summary method is included to display relevant information:

\begin{singlespace}
\begin{figure}[H]
\centering
\begin{BVerbatim}
> summary(mouse_boot)

bdotsBoot Summary

Curve Type: expCurve 
Formula: Volume ~ x0 * exp(Day * k) 
Time Range: (0, 59) [21 points]

Difference of difference: FALSE 
Paired t-test: FALSE 
Difference: Treatment 

FWER adjust method: Permutation 
Alpha: 0.05 
Significant Intervals:
     [,1] [,2]
[1,]   15   32
\end{BVerbatim}
%\caption{Illustration of Mouse data}
\end{figure}
\end{singlespace}

There are a few components of the summary that are worth identifying when reporting the results. In particular, note the time range provided, an indicator of if the test was paired, and which groups were being considered (noticing now it only has \xt{Treatment}, not \xt{A} or \xt{E}). The last section of the summary indicates the method used, an adjusted \xt{alphastar} if \xt{padj} was used, and then a matrix of regions identified as being significantly different. This matrix is \xt{NULL} is no differences were identified at the specified alpha; otherwise there is one row included for each disjointed region of significant difference.

In addition to the provided summary output, a \xt{plot} method is available, with a list of additional options included in \xt{help(plot.bdotsBootObj)}.

\begin{figure}[H]
\centering
\begin{BVerbatim}
plot(mouse_boot)
\end{BVerbatim}

\includegraphics{img/mouse_boot_plot.pdf}
\caption{There are some obvious issues with time for non-homogenous samples, namely, what do we use for bootstrapping? It will be quick fix, whatever we decide, but I don't think ``union of all observed times" is going to work. Here, I artificially cut it back to only 0-60}
\end{figure}

Depending on user needs, these plots can be recreated both without confidence bands or without the additional difference curve

\begin{figure}[H]
\centering
\begin{BVerbatim}
plot(mouse_boot, ciBands = FALSE, plotDiffs = FALSE)

\end{BVerbatim}

\includegraphics{img/mouse_boot_plot_extra.pdf}
\caption{I think I'm going to actually not include this because who cares. I specified that they can find more options with the help function.}
\end{figure}


\section{Extras}

Let's do a brief tour of some of the other additions to bdots that probably doesn't warrant its own section for use




\subsection{Refitting}

[``Better intro; I have an idea" - Patrick Breheny 1/18/31, what does better intro mean?]

There are sometimes situations in which the fitted function returned by \texttt{bfit} is a poor fit. This is largely a consequence of the sensitivity of the \xt{nlme::gnls} function used in \xt{bfit} for fitting the non-linear curves. Sensible starting parameters are computed as a part of the curve fitting functions (i.e., \xt{logistic()}, but see the vignettes for more details), though these can often be improved upon. The quality of the fit can be evidenced by the \texttt{fitCode} or via a visual inspection of the fitted functions against the observations for each subject.  When this occurs, there are several options available to the user, all of which are provided through the function \texttt{brefit} (previously \texttt{bdotsRefit}). \texttt{brefit} takes the following arguments:

\begin{singlespace}
\begin{figure}[H]
\centering
\begin{BVerbatim}
brefit(bdObj, fitCode = 1L, subset = NULL, quickRefit = FALSE, paramDT = NULL)
\end{BVerbatim}
%\caption{Illustration of Mouse data}
\end{figure}
\end{singlespace}

The first of these arguments outside of the \xt{bdObj} is \xt{fitCode}, indicating the minimum fit code to be included in the refitting process. As discussed in Section~\ref{sec:fitcode}, this can be sub-optimal. To add flexibility to which subjects are fit there is now the \xt{subset} argument taking either a logical expression or collection of indices that would be used to subset an object of class \xt{data.table} (am I explaining this clearly?) or a numeric vector with indices that the user wishes to refit.

To assist with the refitting process is the argument \xt{quickRefit}. When set to \xt{TRUE}, \xt{brefit} will take the average coefficients of accepted fits within a group and use those as new starting parameters for poor fits. The new fits will be retained if they have a larger $R^2$ value by default (and really the only option and \textit{technically} this isn't actually true yet but will be). When set to \xt{FALSE}, the user will be guided through a set of prompts to refit each of the curves manually.

Finally, the \texttt{paramDT} argument allows for a \xt{data.table} with columns for subject, group identifiers, and parameters to be passed in as a new set of starting parameters. This \xt{data.table} requires the same format as that returned by \xt{bdots::coefWriteout}. The use of this functionality is covered in more detail in the \xt{bdots} vignettes and is a useful way for reproducing a \xt{bdotsObj} from a plain text file. 

When \texttt{quickRefit = FALSE}, the user is put through a series of prompts along with a series of diagnostics for each of the subjects to be refit:


\begin{singlespace}
\begin{figure}[H]
\centering
\begin{BVerbatim}
Subject: 11
R2: 0.837
AR1: FALSE
rho: 0.9
fitCode: 4

 Model Parameters:
       x0         k 
53.186497  0.051749 

Actions:
1) Keep original fit
2) Jitter parameters
3) Adjust starting parameters manually
4) Remove AR1 assumption
5) See original fit metrics
6) Delete subject
99) Save and exit refitter
Choose (1-6):
\end{BVerbatim}
%\caption{Illustration of Mouse data}
\end{figure}
\end{singlespace}


Along with this is given a plot of the original fit, side-by-side with the suggested alternative, Figure~\ref{fig:refit_plot}. 

\begin{figure}[H]
\centering
\includegraphics{img/mouse_refit_plot.pdf}
% new pars x0=50, k = 0.06
\caption{before and after refit: man i am good at picking new parameters}
\label{fig:refit_plot}
\end{figure}

As the menu item suggests, users have the ability to end the manually refitting process early and save where they had left off. To retain previously refit items and start again at a later time, pass the first refitted object back into the refitter as such:

\begin{singlespace}
\begin{figure}[H]
\centering
\begin{BVerbatim}
refit <- brefit(fit, ...)
refit <- brefit(refit, ...) # pass in the refitted object
\end{BVerbatim}
%\caption{Illustration of Mouse data}
\end{figure}
\end{singlespace}



A final note should be said regarding the option to delete a subject. As \xt{bdots} now automatically determines if subjects are paired based on subject identifiers (necessary for  calculations in significance testing), it is critical that if a subject has a poor fit in one group and must be removed that he or she is also removed from all additional groups in order to retain paired status. This can be overwritten with a final prompt in the \texttt{brefit} function before they are removed. The removal of subjects can also be done with the ancillary function, \texttt{bdRemove}, useful for removing subjects without undergoing the entire refitting process. See \xt{help(bdRemove)} for details.


\subsection{User created curves}

Continue to ponder if this worth elaborating on here or appendix (or at all)

\subsection{Correlations}

There are sometimes cases in which we are interested in determining the correlation of a fixed attribute with group outcome responses across time . This can be done with the \texttt{bcorr} function (previously \texttt{bdotsCorr}), which takes as an argument an object of class \texttt{bdotsObj} as well as a character vector representing a column from the original dataset used in \texttt{bfit}

\begin{center}
\xt{bcorr(fit, "value", ciBands, method = "pearson")} 
\end{center}

And I can't give an example of this with the mouse data atm because it doesn't gracefully handle nonhomogenous time samples.

This returns a thing that can be plotted. Idk, I'll see what bob wants me to say about this because truly im like so whatever

\subsection{$\alpha$ Adjustment}

Finally, we consider an extension to the \texttt{p.adjust} function, \texttt{p\_adjust}, identical to \texttt{p.adjust} except that it accepts method \texttt{"oleson"} and takes additional arguments \texttt{rho}, \texttt{df}, and \texttt{cores}. \texttt{rho} determines the autocorrelation estimate for the oleson adjustment while \texttt{df} returns the degrees of freedom used to compute the original vector of t-statistics. If an estimate of \texttt{rho} isn't available, one can be computed on a vector of t-statistics using the \texttt{ar1Solver} function in \xt{bdots}:

%\begin{center}
%\texttt{t <- diffinv(rnorm(100))} \\
%\texttt{rho <- ar1Solver(t)} \\
%\texttt{unadj\_p <- pt(t, df = 10)} \\
%\texttt{adj\_p <- p\_adjust(unadj\_p, method = "oleson", df = 10, rho = rho, alpha = 0.05)}
%\end{center}


\begin{singlespace}
\begin{figure}[H]
\centering
\begin{BVerbatim}
t       <- diffinv(rnorm(5))
rho     <- ar1Solver(t)
unadj_p <- pt(t, df = 10)
adj_p   <- p_adjust(unadj_p, method = "oleson", 
                    df = 10, rho = rho, alpha = 0.05)
\end{BVerbatim}
%\caption{Illustration of Mouse data}
\end{figure}
\end{singlespace}

Doing so returns both adjusted p-values, which can be compared against the specified alpha (in this case, $0.05$). Alternatively the result will also print an alphastar, a nominal alpha at which one can compare the original p-values:

\begin{singlespace}
\begin{figure}[H]
\centering
\begin{BVerbatim}
> unadjp
[1] 0.5000000 0.0849965 0.0381715 0.1601033 0.0247453 0.0013016
> adjp
[1] 0.9201915 0.1564261 0.0702501 0.2946514 0.0455408 0.0023954
attr(,"alphastar")
[1] 0.027168
\end{BVerbatim}
%\caption{Illustration of Mouse data}
\end{figure}
\end{singlespace}

Here, for example, we see that the last two positions of \xt{unadjp} have values less than \xt{alphastar}, identifying them as significant; alternatively, we see these same two indices in \xt{adjp} significant when compared to \xt{alpha = 0.05}

\section{Discussion}

[short and sweet? I hate unnecessary words]

The original implementation of \xt{bdots} set out to address a very narrow set of problems. Previous solutions beget new opportunities, however, and it is in this space that the second iteration of \xt{bdots} has sought to expand. Since then, the interface between user and application has been significantly revamped, creating a intuitive, reproducible workflow that is able to quickly and simply address a broader range of problems. The underlying methodology has also been improved and expanded upon, offering better control of the family-wise error rate.

While significant improvements have been made, there is room for further expansion. The most obvious of these is the need to include support for non-parametric functions, the utility of which cannot be overstated. Not only would this alleviate the need for the researcher to specify in advance a functional form for the data, it would implicitly accommodate more heterogeneity of functional forms within a group. Along with this, the current implementation is also limited in the quality-of-fit statistics used in the fitting steps to assess performance. $R^2$ and the presence of autocorrelation are relevant to only a subset of the types of data that can be fit, and allowing users more flexibility in specifying this metric is an active goal for future work. In all, future directions of this package will be primarily focused on user interface, non-parametric functions, and greater flexibility in fit metrics (this last sentence kind of sucks).



\appendix

\section*{Appendix A - custom curves}

From an R programming perspective, this is perhaps the most novel and interesting portion of the new package update. Worked use-case examples are included in the pacakge vignettes, so here we will limit discussion to the theoretical considerations when implementing it since it's actually pretty neat (I think). 



\section*{Appendix X} Copy and paste hard data example here \\

We will illustrate use of the updated \xt{bdots} package with a worked example, using an artificial dataset to help detail some of the newer aspects of the package. The dataset will consist of outcomes for a collection of vehicles, consisting of eight distinct groups. These groups will be nested in order of vehicle origin (foreign or domesetic), vehicle class (car or truck), and vehicle color (red or blue). Further, vehicles of different color but within the same origin and class groups will be considered paired observations. A table detailing the relationship of the groups is shown here:



The outcome here is simply \xt{y} due to a lack of creativity, but the functional form assumed (and used in data generation) follows the four parameter logistic, 

\begin{equation}
f_{\theta}(t) = b + \frac{p-b}{1 + \exp \left( \frac{4s}{p-b} (x-t) \right)},
\end{equation}
where $b$, $p$, $s$, and $x$ represent the baseline, peak, slope, and crossover points, respectively



The formula argument serves two functions in \xt{bboot}: first, it specifies the collection of curves we wish to investigate the difference between, and second, it determines if we are interested in directly comparing the differences or the difference of differences between curves. 

To begin, let's reintroduce the structure of the groups we have in our dataset. Recall that we have foreign and domestic cars and trucks, and each of these vehicles comes in red and blue. Recall also that the different colors of each vehicle are considered paired observations.

\begin{table}
\centering
\def\arraystretch{1.5}
\begin{tabular}{|p{0.9in}|p{0.9in}|p{0.9in}|} \hline 
\rowcolor{lightgray} \multicolumn{1}{|c|}{Origin} & \multicolumn{1}{c|}{Class} & \multicolumn{1}{c|}{Color}\\
\hline
\multirow{4}{*}{foreign} & \multirow{2}{*}{car} & red \\
\hhline{~~-}
& & blue \\
\hhline{~--}
& \multirow{2}{*}{truck} & red \\
\hhline{~~-}
& & blue \\
\hline
\multirow{4}{*}{domestic} & \multirow{2}{*}{car} & red \\
\hhline{~~-}
& & blue \\
\hhline{~--}
& \multirow{2}{*}{truck} & red \\
\hhline{~~-}
& & blue \\
\hline
\end{tabular}
\caption{table of stuff}
\label{tab:group_table}
\end{table}


Beginning with a simple case, suppose we want to investigate the difference in outcome between foreign and domestic vehicles. Notionally, we would write

\begin{center}
\tt y $\sim$ Origin(foreign, domestic).
\end{center}


Note that this involves the grouping variable, \xt{Origin}, with the two values we are interested in comparing, \xt{domestic} and \xt{foreign}. With this specification, the distribution of functions considered in \xt{domestic} include all red and blue domestic cars and trucks.


If we wanted to limit our investigation to only foreign and domestic \textit{trucks}, we would do this by including an extra term specifying the group and the desired value. In this case, 

\begin{center}
\tt y $\sim$ Origin(foreign, domestic) + Class(truck).
\end{center}
To compare only foreign and domestic \textit{red} trucks, we would add an additional term for color:

\begin{center}
\tt y $\sim$ Origin(foreign, domestic) + Class(truck) + Color(red).
\end{center}

There are also instances in which we might be considered in the interaction of two groups. Although there is no native way to handle interactions in \xt{bdots}, this can be done indirectly through the difference of differences (McMurray et al 2019, though truthfully I still don't understand why). To illustrate, suppose we are interested in understanding how the color of the vehicle differentially impacts outcome based on the vehicle class. In such a case, we might look at the difference in outcome between red cars and red trucks, and then again the difference between blue cars and blue trucks. Any difference between these two differences would give information regarding the differential impact of color between each of the two classes. This is done in \xt{bdots} using the \xt{diffs} synatx in the formula:

%\textbf{FOUND SOURCE FOR THIS:} \textit{McMurray, Klein-Packard, Tomblin 2019, real time mechanics..pg 7}


%[I think I'm going to move this paragraph to a different section since its kinda out of nowhere] Although there is no native way to handle interactions between groups in \xt{bdots}, this can be done indirectly through the differences of differences (McMurray et al 2019, though truthfully I still don't understand why). To illustrate, suppose we are interested in understanding how the color of a vehicle differentially impacts performance based on the vehicle type such as cars and trucks. We might then look at the difference between red cars and red trucks and then the difference between blue cars and blue trucks. If color does not mediate this difference, the difference between red cars and trucks should be the same as the difference between blue cars and trucks. If color does differentially impact performance between cars and trucks, this will be evident when considering the difference between the differences.


%[this section needs rewritten still its not correct]
%In each of these cases, we have specifed particular groups or nesting of groups who's outcomes we wish to compare. Alternatively, we may be interested in comparing the \textit{difference of differences}. For example, suppose we suspect that there may be some difference between red and blue vehicles, and that this difference may be different for cars compared to trucks. Whereas previously we were interested in comparing the differences in \xt{y} between origin, we are now interested in comparing the differences of \xt{y} between colors. Consequently, we will include this difference on the left hand side of the formula as our new outcome. This is done using \xt{diffs} syntax as such:

\begin{center}
\tt diffs(y, Class(car, truck)) $\sim$ Color(red, blue)
\end{center}

Here, the \textit{outcome} that we are considering is the difference between vehicle classes, with the outcome of interest being color. This is helpful in remembering which term goes on the LHS of the formula. 

Similar as to the case before, if we wanted to limit this difference of differences investigation to only include domestic vehicles, we can do so by including an additional term:

\begin{center}
\tt diffs(y, Class(car, truck)) $\sim$ Color(red, blue) + Origin(domestic).
\end{center}

The formula syntax was originally contrived to make comparisons within groups or within nested groups. Conceivably, however, one could be interested in making the comparison between domestic red trucks and foreign blue cars. Doing so requires a bit of a work around. Examples detailing how one might go about doing this are included in appendix B. 

\section*{Appendix B - Fitting non-nested groups}

(currently just copy pasted from the body of document, not edited so no need to really review)

First, there would be some function of sorts, something like \xt{makeUniqueGroups} which would create a new group column with each permutation of previous groups being given a unique identifier. Doing this on the vehicle example would look something like \xt{fit <- makeuniquewhatever} resulting in the following grouping structure (for example) (and maybe you could specify group name and values who knows, kinda like factor this is just a working thought example)

\begin{center}

\begin{tabular}{|p{0.9in}|p{0.9in}|p{0.9in}|p{0.5in}|} \hline 
\rowcolor{lightgray} \multicolumn{1}{|c|}{Origin} & \multicolumn{1}{c|}{Class} & \multicolumn{1}{c|}{Color} & \multicolumn{1}{c|}{bgroup}\\
\hline
\multirow{4}{*}{foreign} & \multirow{2}{*}{car} & red & A\\
\hhline{~~--}
& & blue & B \\
\hhline{~---}
& \multirow{2}{*}{truck} & red & C\\
\hhline{~~--}
& & blue & D\\
\hline
\multirow{4}{*}{domestic} & \multirow{2}{*}{car} & red & E \\
\hhline{~~--}
& & blue & F\\
\hhline{~---}
& \multirow{2}{*}{truck} & red & G\\
\hhline{~~--}
& & blue & H\\
\hline
\end{tabular}
\end{center}

To then investigate differences in outcome between a foreign red car and a domestic blue truck would simply then be

\begin{center}
\tt y $\sim$ bgroup(A, H)
\end{center}

yeah not like sexy or anything but whatever it would work.


\section{Appendix 2}


\begin{singlespace}
\begin{figure}[H]
\centering
\begin{BVerbatim}
> logistic
function (dat, y, time, params = NULL, ...) 
{
    logisticPars <- function(dat, y, time, ...) {
        time <- dat[[time]]
        y <- dat[[y]]
        if (var(y) == 0) {
            return(NULL)
        }
        mini <- min(y)
        peak <- max(y)
        r <- (peak - mini)
        cross <- time[which.min(abs(0.5 * r - y))]
        q75 <- 0.75 * r + mini
        q25 <- 0.25 * r + mini
        time75 <- time[which.min(abs(q75 - y))]
        time25 <- time[which.min(abs(q25 - y))]
        tr <- time75 - time25
        tr <- ifelse(tr == 0, median(time), tr)
        slope <- (q75 - q25)/tr
        return(c(mini = mini, peak = peak, slope = slope, cross = cross))
    }
    if (is.null(params)) {
        params <- logisticPars(dat, y, time)
    }
    else {
        if (length(params) != 4) 
            stop("logistic requires 4 parameters be specified for refitting")
        if (!all(names(params) %in% c("mini", "peak", "slope", 
            "cross"))) {
            stop("logistic parameters for refitting must be correctly labeled")
        }
    }
    if (is.null(params)) {
        return(NULL)
    }
    y <- str2lang(y)
    time <- str2lang(time)
    ff <- bquote(.(y) ~ mini + (peak - mini)/(1 + exp(4 * slope * 
        (cross - (.(time)))/(peak - mini))))
    attr(ff, "parnames") <- names(params)
    return(list(formula = ff, params = params))
}
<bytecode: 0x5591d0f862c0>
<environment: namespace:bdots>
\end{BVerbatim}
%\caption{Illustration of Mouse data}
\end{figure}
\end{singlespace}




\begin{singlespace}
\begin{figure}[H]
\centering
\begin{BVerbatim}
function (dat, y, time, params = NULL, startSamp = 8, ...) {
    logisticPars <- function(dat, y, time, startSamp, ...) {
        time <- dat[[time]]
        y <- dat[[y]]
        if (var(y) == 0) {
            return(NULL)
        }
        spars <- data.table(param = c("mini", "peak", "slope", "cross"), 
                    mean = c(0.115, 0.885, 0.0016, 765), 
                    sd = c(0.12, 0.12, 0.00075, 85), 
                    min = c(0, 0.5, 0.0009, 300), 
                    max = c(0.3, 1, 0.01, 1100))
        fn <- function(p, t) {
            p[1] + (p[2] - p[1])/(1 + exp(4 * p[3] * ((p[4] - t)/(p[2] - p[1]))))
        }
        tryPars <- vector("list", length = startSamp)
        for (i in seq_len(startSamp)) {
            maxFix <- Inf
            while (maxFix > 1 | maxFix < 0.6) {
                tryPars[[i]] <- Inf
                while (any(spars[, tryPars[[i]] <= min | tryPars[[i]] >= max])) {
                  tryPars[[i]] <- spars[, rnorm(length(tryPars[[i]])) * sd + mean]
                }
                maxFix <- max(fn(tryPars[[i]], time))
            }
        }
        r2 <- vector("numeric", length = startSamp)
        for (i in seq_len(startSamp)) {
            yhat <- fn(tryPars[[i]], time)
            r2[i] <- mean((y - yhat)^2)
        }
        finalPars <- tryPars[[which.min(r2)]]
        names(finalPars) <- c("mini", "peak", "slope", "cross")
        return(finalPars)
    }
    if (is.null(params)) {
        params <- logisticPars(dat, y, time, startSamp)
    }
    # Was var(y) == 0?
    if (is.null(params)) {
        return(NULL)
    }
    y <- str2lang(y)
    time <- str2lang(time)
    ff <- bquote(.(y) ~ mini + (peak - mini)/(1 + exp(4 * slope * 
        (cross - (.(time)))/(peak - mini))))
    attr(ff, "parnames") <- names(params)
    return(list(formula = ff, params = params))
}
<bytecode: 0x5591d39da1f0>
<environment: namespace:bdots>
\end{BVerbatim}
%\caption{Illustration of Mouse data}
\end{figure}
\end{singlespace}









\newpage

\Chapter{The Look Onset Method}

\section{Introduction}

\cn{[or, you know what, create a new introduction here and let what follows be the first part of next body sentence]}

Spoken words create analog signals that are processed by the brain in real time. That is, as spoken word unfolds, a collection of candidate words are considered until the target word is recognized. The degree to which a particular candidate word is being considered is known as activation. An important part of this process involves not only correctly identifying the word but also eliminating competitors. For example, we might consider a discrete unfolding of the word ``elephant" as ``el-e-phant". At the onset of ``el", a listener may activate a cohort of potential resolutions such as ``elephant", ``electricity", or ``elder", all of which may be considered competitors. With the subsequent ``el-e", words consistent with the received signal, such as ``elephant" and ``electricity" remain active competitors, while incompatible words, such as ``elder", are eliminated. Such is a rough description of this process, continuing until the ambiguity is resolved and a single word remains.

%[don't like this next section]


\cn{[start more broadly, there are a number of ways to do this, we use activation]}

Our interest is in measuring the degree of activation of a target, relative to competitors. Activation, however, is not measured directly, and we instead rely on what can be observed with physiological behavior. And though there are a number of relevant indices (Spivey mouse trials), we concern ourselves here with eye tracking data collected in the context of the Visual World Paradigm (VWP) \cite{tanenhaus1995integration}, an experimental model in which a participant's eye movements are tracked as the respond to spoken language. In a typical VWP experiment, participants are placed in the presence of visual objects (typically presented on a computer screen) and asked to select one in response to spoken language. The location of fixations are measured in real time, with the proportion of fixations towards any potential targeted aggregated across trials

[...]

Recently, researchers have begun to reexamine some of the underlying assumptions associated with the VWP, calling into question the validity or interpretation of current methods. We present here a brief history of word recognition in the context of the VWP, along with an examination of contemporary concerns. We address some of these concerns directly, presenting an alternate method for relating eye-tracking data to lexical activation.


This section needs work but it mostly covers the gist of what I am trying to convey, namely we are about to go from history $\rightarrow$ current state of the world $\rightarrow$ proposal and comparison $\rightarrow$ results.




\paragraph{Visual World Paradigm} The Visual World Paradigm (VWP) was first introduced in 1995, making the initial link between the mental processes associated with language comprehension and eye movements \cite{tanenhaus1995integration}. A typical experiment in the VWP involves situating a subject in front of a ``visual world", commonly a computer screen today, and asking them to identify and select an object corresponding to a spoken word. The initiation of eye movements and subsequent fixations are recorded as this process unfolds, with the location of the participants' eyes serving as a proxy for which words or images are being considered. This association was first demonstrated by comparing how the mean time to initiate an eye movement to the correct object was mediated by the presence of phonological competitors (``candy" and ``candle", sharing auditory signal at word onset) and situations containing syntactic ambiguity (``Put the apple on the towel in the box" and ``Put the apple \textit{that's} on the towel in the box" in ambiguous scenarios with one or more apples). It is by comparing the trajectory of these mechanics across trials in the presence of auditory or semantic competitors that researchers have used the VWP in their investigation of spoken word recognition.




\paragraph{Proportion of fixation} It was against simulated TRACE data that Allopenna (1998) found a tractable way of analyzing eye tracking data. By coding the period of a fixation as a 0 or 1 for each referent and taking the average of fixations towards a referent at each time point, Allopenna was able to create a ``fixation proportion" curve that largely reflected the shape and competitive dynamics of word activation suggested by TRACE, both for the target object, as well as competitors. This also served to establish a simple linking hypothesis, specifically, ``We made the general assumption that the probability of initiating an eye movement to fixate on a target object $o$ at time $t$ is a direct function of the probability that $o$ is the target given the speech input and where the probability of fixating $o$ is determined by the activation level of its lexical entry relative to the activation of other potential targets." Further of note is what this linking hypothesis does not include, namely:


\begin{enumerate}
\item No assumption that scanning patterns in and of themselves reveal underlying cognitive processes
\item No assumption that the fixation location at time $t$ necessarily reveals where attention is directed (only probabilistically related to attention)
\end{enumerate}


Other assumptions included here include that language processing proceeds independent of vision, and that visual objects are not automatically activated. Or, more succinctly, it assumes that fixation proportions over time provide an essentially direct index of lexical activation, whereby the probability of fixating an object increases as the likelihood that it has been referred to increases.


While other linking hypotheses have been presented (Magnuson 2019) \cite{Magnuson2019}, that there is \textit{some} link between the function of fixation proportions and activation has guided the last 25 years of VWP research.



\paragraph{Parametric Methods and Individual Curves} While there have most certainly been advancements to the use of the VWP for speech perception and recognition (and expanded into related domains, such as sentence processing and characterizing language disorders (according to Bob)), we  limit ourselves here to one in particular. In 2010, McMurray et al expanded the domain of the VWP by introducing emphasis on individual differences in participant activation curves. Two aspects of this paper are relevant here. First, although they were not the first to introduce non-linear functions to be fit to observed data, they did introduce a number of important parametric functions in use today, namely the four (or five) parameter logistic and the double-gauss (asymmetrical gauss), the primary benefit being that the parameters of these functions are interpretable, that is, they ``describe readily observable properties." Second, which I suppose was also introduced by Mirman (2008) \cite{Mirman2008} to some degree (though I have not read it yet, just pulling from Bob) is specifying individual subject curves across participants. This has been critical in that:

\begin{singlespace}
\begin{enumerate}
\vspace{-3mm}
\item The parameters of the functions describe interpretable properties
\item This made the idea of distributions of parameters for a particular group a relevant construct
\end{enumerate}
\end{singlespace}

Though not stated directly (given it predates bdots by 8 years), this also served as the impetus for investigating group differences in word activation through the use of bootstrapped differences in time series \cite{oleson2017detecting} and the subsequent development of the \xt{bdots} software in R for analyzing such differences. (A history of exploring differences in group curves can be found in \cite{seedorff2018bdots}).

This brings us to the current day, where the state of things is such that VWP data is widely used to measure word recognition by collecting data on individual subjects and fitting to them non-linear parametric curves with interpretable parameters. Context in hand, we are now able to introduce some of the main characters of our story, specifically how data in the VWP is understood and used. 



\begin{figure}[h]
\centering
\includegraphics[scale=0.4]{logistic_label.png}
\caption{An illustration of the four-parameter logistic and its associated parameters, introduced as a parametric function for fixations to target objects in McMurray 2010. Can describe the parameters in detail, but should also have the formula itself somewhere to be referenced. (Equation~\ref{eq:logistic})}
\label{fig:logistic_definition}
\end{figure}



\section{Analysis with VWP Data}

The following section goes into more detail on the specifics


\subsection{Anatomy of Eye Mechanics \cn{[this section needs new name]}}

In the context of eye tracking data and word recognition, there are a few mechanics with which we are concerned. The first of these is activation. Even with the immediacy and (fullness? some word they use to describe dense time series here being better than yes/no response), what we observe with any eye movement is not a direct readout of the underlying activation.  Rather, there is a period of latency between the decision to launch an eye movement and the physiological response, a period known as oculomotor delay. And finally, there are they physical mechanics of the eye movements themselves, the saccade and the fixation which, together, make up a ``look". We will briefly address each of these in the reverse of the order in which they were introduced.



\paragraph{Saccades and fixations:} Rather than acting in a continuous sweeping motion as our perceived vision might suggest, our eyes themselves move about in a series of short, ballistic movements, followed by brief periods of stagnation. These periods of movement and stagnation, respectively, are the saccades and fixations. 

Saccades are short, ballistic movements lasting between 20ms-60ms, during which time we are effectively blind. Once in motion, saccades are unable to change trajectory from their intended destination. Following this movement is a period known as a fixation, itself made up of a necessary refraction period (during which time the eye is incapable of movement) followed by a period of voluntary fixation which may include planning time for deciding the destination of the next eye movement; the duration of fixations are typically (some length). It will be convenient to follow previous convention and consider a saccade followed by its adjacent fixation as a single concept called a ``look" \cite{mcmurray2002look}. We take particular care here to note that the beginning of a look, or ``look onset", starts the instance that a previous look ends or, said another way, the instant an eye movement is launched. A visual description of these is provided in Figure~\ref{fig:sac_fix_look}.



\begin{figure}[H]
\centering
\includegraphics[scale=0.25]{sac_fix_look.png}
\caption{redo this image to match anataomy of look image, also for size}
\label{fig:sac_fix_look}
\end{figure}

\paragraph{Oculomotor delay:} While the physiological responses are what we can measure, they are not themselves what we are interested in. Rather, we are interested in determining word activation, itself governing the cognitive mechanism facilitating movements in the eye. Between the decision to launch an eye movement (a cognitive mechanism governed by the activation, next section) and the movement itself is a period known as oculmotor delay. It is typically estimated to take around 200ms to plan and launch an eye movement \cite{viviani1990time}, and this is usually accounted for by subtracting 200ms from any observed behavior. As oculomotor delay is only ``roughly" estimated to be around 200ms, we suggest that accounting for randomness will be critical in correctly recovering the the cognitive mechanism of interest or at very least in identifying possible sources of bias or error. How this phenomenon relates to saccades and fixations is demonstrated in Figure~\ref{fig:sac_fix_look_om}.


\begin{figure}[H]
\centering
\includegraphics[scale=0.25]{om_delay2.png}
\caption{redo this image}
\label{fig:sac_fix_look_om}
\end{figure}


\begin{figure}
\centering
\includegraphics[scale=0.5]{labeled_full_diagram.png}
\caption{I want to do this figure again but differently. have saccade be  two bars matching anatomy of look, include refractory period of fixation, noting that that and saccade are identical, followed by period of time of voluntary fixation (theoretically relevant) followed by next CM}
\label{fig:full_diagram_looks}
\end{figure}

\begin{figure}
\centering
\includegraphics[scale=0.5]{what_is_a_look.pdf}
\label{fig:whats_in_a_look}
\caption{Alternative to Figure~\ref{fig:full_diagram_looks} with diagram of saccade/fixation more consistent with Figure~\ref{fig:anatomy_of_look}. There may be a bit more than is needed here, especially with the pink and blue (and maybe even subsequent CM). Although I don't discuss their use in detail, the pink/blue/second CM highlight the fact that \textit{even when} we argue that using proportion of fixation method indirectly captures strength of activation via length of duration, the assumption of linearity is incorrect. For the sake of argument, assuming that the refractory period is exactly 100ms and the length of oculomotor delay is 200ms, a fixation of length 500ms may seem to be 25\% larger than a fixation lasting 400ms, but in terms of the \textit{intentional} fixation period, it is twice as large. This serves both to demonstrate another issue with the proportion method while also paving the way for incorporating fixation length in the future. Still, this may not all be necessary here, or presented in a cleaner way} 
\end{figure}

\subsection{Activation}

The concept of activation, as it relates to the discussion here, arises from the metaphor in which word perception is made up of a network made up of hierarchical levels (letter, phoneme, word, etc.,) acting as an interactive process unfolding in time \cite{McClelland1981}. Under this \textit{interactive activation model}, greater activation is associated with a greater excitatory action for a network node (specifically here, a word) resulting from consistency with the received auditory signal. The interactive activation model allows for both excitatory and inhibiting activations, resulting in the ``competing" activation curves being modeled in the VWP. [maybe also address happens continuously, hence the continuous mapping model of lexical activation]

Here tie in idea of activation, though need to be more concise about what we mean. \cn{Good source for framework being (McClelland and Rumelhard 1981? Rumelhart and McClelland 81 and 82, and mcclelland/elman 86 with trace). They seem to all mention the ``interactive activation framework" which may be worthwhile to elaborate on further. For now, assume that we have adequately stated \textit{what it is}.}

While a number of experimental methods are used as real-time indices of lexical access (\cite{Spivey2005}, others), we concern ourselves here with the use of eyetracking as it relates to activation as first suggested by \cite{allopenna1998tracking}. Whereas the initial treatment of eyetracking data made no attempt identify or model subject-specific trends, more recent work has made strides in making subject analysis more tractable. Specifically, we adopt the idea that each participant's results can be fit to non-linear functions who's parameters describe clinically relevant properties \cite{mcmurray2010individual}. We will denote this activation function $f$ with parameters $\theta$ as a function in time, giving $f(t|\theta)$

For example, the four parameter logistic function in Figure~\ref{fig:logistic_definition} is often used to model fixations to the Target object in the VWP with functional form

\begin{equation} \label{eq:logistic}
f(t|\theta) = \frac{p-b}{1 + \exp \left(\frac{4s}{\text{p}-b} (x - t) \right)} + b.
\end{equation}

Similarly, a six parameter asymmetric Gaussian function, often used for fixations to competitors, is given as

\begin{equation} \label{eq:dg}
f(t|\theta) = \begin{cases}
\exp \left( \frac{(t - \mu)^2}{-2\sigma_1^2} \right) (p - b_1) + b_1 \quad \text{if } t \leq \mu \\
\exp \left( \frac{(t - \mu)^2}{-2\sigma_2^2} \right) (p - b_2) + b_2 \quad \text{if } t > \mu
\end{cases}
\end{equation}

(I didn't make a nice graph/label for this). 

While both functions are commonly used in the VWP for modeling eye fixations, for simplicity we will limit the primary focus of our discussion to fixations to the Target with the four parameter logistic, though ultimately our argument is agnostic to the modeling function used, parametric or otherwise. Discussion related to the asymmetric Gaussian is treated in the appendix.


\subsection{VWP data}


We now consider how the aforementioned mechanics relate to the visual world paradigm. In a typical instantiation of the VWP, a participant is asked to complete a series of trials, during each of which they are presented with a number of competing images on screen (typically four). A verbal cue is given, and the participants are asked to select the image corresponding to the spoken word. All the while, participants are wearing (generally) a head-mounted eye tracking system recording where on screen they were fixated. 

An individual trial of the VWP may be short, lasting anywhere from 1000ms to 2500ms before the correct image is selected. Prior to selecting the correct image, the participant's eyes scan the environment, considering images as potential candidates to the spoken word. As this process unfolds, a snapshot of the eye is taken at a series of discrete steps (typically every 4ms) indicating where on the screen the participant is fixated. A single trial of the VWP typically contains no more than four to eight total ``looks" before the correct image is clicked, resulting in a paucity of data in any given trial.

[relevant quote][``We find that eye movements to objects in the workspace are closely time-locked to referring expression s in the unfolding speech stream, providing a sensitive and nondistruptive measure of spoken language comprehension during continuous speech" \cite{allopenna1998tracking}]

To be clear, eye trackers themselves only record $x$ and $y$ coordinates of the eye at any given time, and it is only after the fact that ``psychophysical" attributes are mapped onto the data (saccades, fixations, blinks, etc.,). We adopt the strategy of prior work in discussing eye tracking data in terms of their physiological mapping, as this will be crucial in constructing a physiologically relevant understanding of the problem at hand \cite{mcmurray2002look}.

[``Default interpretation is that greater fixation proportions indicate greater activation in the underlying processing system" \cite{Magnuson2019}]


To create a visual summary of this process aggregated over all of the trials, a la Allopena, a ``proportion of fixations" curve is created, aggregating at each discrete time point the average of indicators of whether or not a participant is fixated on a particular image. A resulting curve is created for each of the competing categories (target, cohort, rhyme, unrelated), creating an empirical estimate of the activation curve, $f(t|\theta)$. See Figure~\ref{fig:bob_diagram_full}. For any subject $i = 1, \dots, n$, across times $t = 0, \dots, T$ and trials $j = 1, \dots, J$, a construction  of this curves can be expressed as:


\begin{equation}\label{eq:sum_proportions}
y_{it} = \frac1J \sum z_{ijt}
\end{equation}
where $z_{ijt}$ is an indicator $\{0, 1\}$  towards a particular object in trial $j$ at time $t$ and such that we have an empirical estimate of the activation curve,
\begin{equation}\label{eq:empir_to_activation}
f(t | \theta_i) \equiv y_{it}.
\end{equation}



For our discussions here, we will call this the proportion of fixation method (though we not that in actuality it is the proportion of \textit{trials} in which a fixation occurs at each time point).


\begin{figure}[H]
\centering
\includegraphics[scale=0.45]{bob_vwp_full.png}
\caption{Stole this from Bob (who apparently stole it from richard aslin), plan on making my own}
\label{fig:bob_diagram_full}
\end{figure}


As each individual trial is only made up of a few ballistic movements, the aggregation across trials allows for these otherwise discrete measurements to more closely represent a continuous curve. Curve fitting methods, such as those employed by \xt{bdots}, are then used to construct estimates of function parameters fitted to this curve. Figure~\ref{fig:bob_diagram_full} provides an illustration.

\section{Bias/Look Onset (better name for section)} 


Having given due consideration to the state of things are they are, we find ourselves in a time of moral reflection, reexamining the underlying relationship between lexical activation (the mechanism of interest) and the physiological behavior we are able to observe (here, specifically eye-tracking). This is referred to in the literature as the linking hypothesis. And while there are a number of competing hypothesis, they each share a collection of implicit assumptions relating what is observed to what is being studied.

The simplest version of a linking hypothesis in the context of the VWP is the ``general assumption that the probability of initiating an eye movement to fixate on a target object $o$ at time $t$ is a direct function of the probability that $o$ is the target given the speech input and where the probability of fixating $o$ is determined by the activation level of its lexical entry relative to the activations of other potential targets (i.e., the other visible objects)" \cite{allopenna1998tracking}. It is from this assumption that we justify the relation in Equation~\ref{eq:empir_to_activation}. To a degree, this assumption is shared by most linking hypothesis in that the probabilistic nature of the proportions of fixations is assumed to be related in time to the strength of the underlying activation. Primary differences in linking hypotheses tend to revolve around the particulars of the mechanics involved, including the duration of fixations, eye scanning behavior, the impacts of priming, or the relation between visual  processing acting in conjunction with lexical activation. We make no statement as to the merits of each, though see \cite{Magnuson2019} for a review.

We consider a particular meta contribution to this debate presented by McMurray in which he probed the relationship between the observed dynamics in the fixations and the underlying dynamics of activation under a variety of assumptions \cite{mcmurray2022m}. In short, he showed that curves reconstructed using the standard proportion of fixations analysis in the VWP were poor estimates of the underlying system, with the magnitude of bias increasing with the complexity of the mechanisms involved. Though this made few claims as to what the underlying mechanics may be, it did demonstrate the inherent difficulty in relating observable behavior to the underlying cognitive process.

An important contribution made there, however, and one that we adopt here is an explicit definition of the underlying activation function. Given the relation in Equation~\ref{eq:empir_to_activation}, it is reasonable to assume that the underlying activation of any of the objects with the VWP could be modeled with a nonlinear function $f(t|\theta)$. The goal of a VWP analysis, then, is the recovery of this underlying activation function.

From this assumption we propose an alternative model of the relation between the underlying activation and the observed behavior, with a careful delineation of the psycho-physical components of a look in conjunction with its generating behavior. In particular, we consider the cognitive mechanism associated with initiating an eye movement, which is probabilistically associated with lexical activation, the delay between this and the onset of its associated look, and finally how the different components of the look are related to fundamentally different mechanisms. From this and what we ultimately argue is that observed bias in the recovery of the activation curve under the proportion of fixations method can be partitioned into two distinct components:

[i would like to maybe go into more detail here or have a picture]

%\begin{singlespace}
\begin{enumerate}
\item The first source of bias, which is the primary emphasis of my proposal, is what I call the ``added observation" bias. This involves the fact that in  a standard analysis of VWP data, the entire duration of a fixation is indicated with a $\{0,1\}$  at any time, $t$, without having observed any behavior associated with the initiation of an eye movement at that time. In other words, by using the entire fixation, we are both obscuring data relevant to the mechanism of interest (onset) while also conflating it with data generated by a fundamentally different mechanism [unpack this a bit more here].
\item The second source of bias is ``delayed observation bias". This bias arises from the fact that an eye movement launched at some time $t$ was planned at some time prior. This is primarily a consequence of the oculomotor delay
\end{enumerate}
%\end{singlespace}

The first source of bias, the ``added observation" bias, arises singularly from the fact that the destination of a look, which is observed at look onset, has a fundamentally \textit{different} generating mechanism than what determines the duration of a look, never minding such mechanics as the duration of a saccade or the refractory period of a fixation. Nonetheless, a standard analysis of VWP data does not differentiate between the initial onset and the period of subsequent fixation: both are recorded as either $0$ or $1$ according to it's location. A look onset at time $t$ is probabilistically determined by by its lexical activation $f(t|\theta)$ whereas the period of fixation is governed by a separate mechanism altogether. Treating the subsequent fixation as indistinguishable has the effect of not only ``adding" observations to the data, but adding observations that necessarily biased. The result is a distorted estimation of the underlying activation. A depiction of this phenomenon is given in Figure~\ref{fig:folly_of_fixation}. 


\begin{figure}[H]
    \centering
    \subfigure[]{\includegraphics[width=0.45\textwidth]{logistic_a.pdf}} 
    \subfigure[]{\includegraphics[width=0.45\textwidth]{logistic_b.pdf}} 
    \subfigure[]{\includegraphics[width=0.45\textwidth]{logistic_c.pdf}}
    \caption{ \textbf{(a.)} Example of a nonlinear activation curve $f(t|\theta)$ \textbf{(b.)} At some time, $t$, a saccade is launched with its destination probabilistically determined by $f(t|\theta)$ \textbf{(c.)} For a look persisting over $n$ time points, $t+1, \dots, t+n$, we are recording ``observed" data, adding to the proportion of fixations at each time but without having gathered any additional observed data at $f(t+1 | \theta), \dots,f(t+n | \theta)$, thus inflating (or in the case of a monotonically increasing function like the logistic, deflating) the true probability. }
\label{fig:folly_of_fixation}
\end{figure}

The second source of bias is the ``delayed observation" bias. It is well established in the literature that the time it takes to plan and launch a saccade is around 200ms \cite{viviani1990time}, which is typically accounted for by subtracting 200ms from the observed data. There are two aspects of this that are worth considering further. First, if the mean duration of this oculomotor delay is not 200ms, bias will be observed as the difference between the true time and the 200ms adjustment. And although not bias in the technical sense, there has been no accounting for what effect randomness in this delay has on the error in the recovery of the underlying activation. It will be worthwhile in investigating this as the potential magnitude will determine if this delay is worth considering in any more detail in future research.

---

While we present no immediate solution to the effects of randomness in the delayed observation bias, we argue that the added observation bias can be rectified by using \textit{only} the times observed with look onset in the recovery of the underlying dynamics. We call this the the ``look onset" method, which we explain in more detail.

We argue further that the added observation bias can be rectified by using \textit{only} the times observed with the look onset in the recovery of the underlying dynamics. We call this the ``look onset" method in contrast to the ``proportion of fixation" method. And while we do not present an immediate solution to the problem of the delayed observation bias, we do demonstrate the effects on estimation error in the presence of oculomotor delay even when the correct mean is accounted for.


\paragraph{Look Onset Method:} The look onset method differ in the proportion of fixation method only in determining which observed data should be considered relevant in the estimation of lexical activation. A particularly compelling argument to made in favor of the look onset method, a corollary of the added observation bias, is that it has a readily defensible mathematical description. A saccade launched at time $t$ (marking the onset of a look) is assumed to be probabilistically determined by its lexical activation (relative to competitors) at time $t$, giving us

\begin{equation} \label{eq:saccade_dist}
s_t \sim Bin[f(t| \theta)]
\end{equation}

(it may be that $l_t$ for look onset is better notation, but my concern is that it doesn't capture the ``onset" nature that we are concerned with and may instead suggest the entire saccade + fixation).

The utility of this is evident when tasked with stating the distribution of $y_t$ in Equation~\ref{eq:sum_proportions} as it relates to $f(t|\theta)$. And where given the overlap of fixations within a particular trial, it is unclear what relation $y_t$ may have to $y_{t+1}$. 

[maybe statement along the lines of we make no assumptions as to the processing of visual stimuli, with probabilities of fixations independent of previous fixations? consistent with other ``bare bones" linking hypotheses]

Two further comments are made about this method here. First, in anticipation of the observation that the look onset method discards relevant information regarding the strength of activation by not implicitly including the length of a fixation (source), we acknowledge this and reserve further comment for the discussion. Second, given the difference in structure of the observed data, we confirm that the current iteration of \xt{bdots} is capable of fitting nonlinear curves to data both under the proportion of fixation and look onset methods, removing any technical difficulties in the adoption of this method.



\section{Simulations}

Simulations were conducted to replicate the mechanics of a look combined with oculomotor delay as detailed in Figure~\ref{fig:anatomy_of_look}. This section only address Target fixations with a four parameter logistic as given in Equation~\ref{eq:logistic}; simulations according to looks to competitors is treated in the appendix. We will begin by describing the process of simulating a single subject.

\begin{figure}[H]
\centering
\includegraphics[width=\textwidth]{anatomy_of_look.pdf}
\caption{Anatomy of a look -- a key thing to discuss somewhere is the OM delay, refractory period, and planning time. The latter two go in $\gamma$. Worth noting also that while we do need to be able to control for $\rho$, \textit{information} regarding strength of consideration will be in $\gamma$ less the refractory period}
\label{fig:anatomy_of_look}
\end{figure}


First, each subject randomly draws a set of parameters $\theta_i$ from an empirically determined distribution based on normal hearing participants in the VWP \cite{FarrisTrimble2014} to construct a subject specific generating curve, $f(\cdot | \theta_i)$, where $f$ here is assumed to be the logistic given in Equation~\ref{eq:logistic}.   It is according to this function that the decision to initiate a look at time $t$ will subsequently direct itself to the Target with probability $f(t|\theta_i)$. We then go about simulating trials according to the following method: At some time $t_0$, a subject initiates a look. This look persists for at least a duration of $\gamma$, drawn from a gamma distribution with mean and standard deviation independent of time and previous fixations. At time $t_0+\gamma$, the subject determines the location of its next look, with the next look being directed towards the target with probability $f(t+\gamma | \theta_i)$. The decision to initiate a look is followed by a period of oculomotor delay, $\rho$, during which time the subject remains fixated in the current location. Finally, at time $t_0 + \gamma + \rho$, the subject ends the look initiated at $t_0$ and immediately begins its next look to the location determined at time $t_0 + \gamma$. For the look onset method, the only data recorded are the times of a look onset and their location: in this case, at times $t_0$ and $t_0 + \gamma + \rho$. By contrast, the proportion of fixation method records the object of fixation at 4ms intervals for the entire period of length $\gamma + \rho$. A single trial begins at $t = 0$ and continues constructing looks as described until the total duration of looks exceeds 2000ms. Each subject undergoes 300 trials, and 1,000 subjects are included in each simulation.

Three total simulations were performed to investigate the biases identified in the previous section, each differing only in the random distribution of the oculomotor delay parameter, $\rho$. In the first simulation, we set $\rho = 0$ to remove any oculomotor delay. In this scenario, a look initiated at time $t$ by subject $i$ will be directed towards the target with probability $f(t|\theta_i)$. Doing so removes any potential bias from delayed observation and allows us to identify the effects of the added observation bias in isolation. In the remaining simulations we probe the effects of randomness in oculomotor delay, investigating what effect uncertainty may have in our recovery of the generating function. We do this assigning $\rho$ to follow either a normal or Weibull distribution, each with a mean value of 200ms. As is standard in a VWP analysis, we subtracted 200ms from each observed point prior to fitting the data. A consequence of this is that in these simulations, the bias itself is accurately accounted for by subtracting the correct mean, with the resulting error in the curve fitting process the result of the inherent variability. This does not detract from the argument being made, however, and any true bias in the mean of the oculomotor delay would asymptotically result in a horizontal shift of the observed data according to the direction and magnitude of the bias.

The simulations are performed in R, with the simulation code available on the author's Github page (link?). Simulated data was fit to the four parameter logistic function using \xt{bdots v2.0.0}.

As all of the data could not be individually inspected prior to being included in the analysis, subjects were excluded from consideration if fitted parameters from either the look onset method or the proportion of fixation method resulted in a peak less than the slope, or if the crossover or slope were negative. In the settings in which there was no delay, normally distributed delay, or Weibull distributed delay, 981, 973, and 981 subjects were retained, respectively.

[In all of the histograms I removed the top and bottom 1\% of observations, true outliers which obscured the ranges of the histograms, relevant particularly in comparing proportion method  par bias with and without delay]

\subsection{No Delay}


\begin{figure}[H]
\centering
    \subfigure[]{\includegraphics[width=0.9\textwidth]{no_delay_par_bias_onset.pdf}} 
    \subfigure[]{\includegraphics[width=0.9\textwidth]{no_delay_par_bias_proportion.pdf}} 
\caption{Parameter bias for with oculomotor delay. }
\label{fig:par_bias_no_delay}
\end{figure}

\begin{figure}[H]
\centering
\includegraphics[width=0.9\textwidth]{rep_curves_no_delay.pdf}
\caption{Representative curves with no oculomotor delay}
\label{fig:rep_curves_no_delay}
\end{figure}



\subsection{Normal Delay}

\begin{figure}[H]
\centering
    \subfigure[]{\includegraphics[width=0.9\textwidth]{normal_delay_par_bias_onset.pdf}} 
    \subfigure[]{\includegraphics[width=0.9\textwidth]{normal_delay_par_bias_proportion.pdf}} 
\caption{Parameter bias with normal OM delay}
\label{fig:par_bias_normal_delay}
\end{figure}

\begin{figure}[H]
\centering
\includegraphics[width=0.9\textwidth]{rep_curves_normal_delay.pdf}
\caption{Representative curves with normal oculomotor delay}
\label{fig:rep_curves_normal_delay}
\end{figure}

\subsection{Weibull Delay}


\begin{figure}[H]
\centering

    \subfigure[]{\includegraphics[width=0.9\textwidth]{weibull_delay_par_bias_onset.pdf}} 
    \subfigure[]{\includegraphics[width=0.9\textwidth]{weibull_delay_par_bias_proportion.pdf}} 
\caption{Parameter bias with Weibull oculomotor delay}
\label{fig:par_bias_weibull_delay}
\end{figure}

\begin{figure}[H]
\centering
\includegraphics[width=0.9\textwidth]{rep_curves_weibull_delay.pdf}
\caption{Representative curves with Weibull oculist delay}
\label{fig:rep_curves_weibull_delay}
\end{figure}





\subsection{Results}

[I will go back and comment on results in each of the scenarios with broader discussion of implications in each, though normal and Weibull will be similar]

In addition to the visual summaries, we present in Table~\ref{tab:mise_sims} a summary of the mean integrated squared error (MISE) between the generating and recovered curves using each of the methods

We begin by noting the  magnitude of difference between the look onset method and the proportion of fixation method in the case of $\rho = 0$, or No Delay, demonstrating the amount of bias introduced in the proportion method. This alone demonstrates how critical of an issue the added observation bias is in the recovery of the underlying activation.

To assess the effects of randomness in the oculomotor delay, it seems prudent to limit the comparisons within each method. Considering first the look onset method, we see that as the degree of variability increases, so does the difficulty in correctly recovering the underlying curve. It is important to note that these magnitudes are meant to be relative rather than absolute: the particular values observed are a function of the relationship between the generating $\gamma$ distribution and that of $\rho$. Nonetheless, this does suggest a need to further investigate ways to control for the added uncertainty. To quickly comment on the apparently ``flipped" MISE for the proportion of fixation method as it relates to the normal and Weibull distributed oculomotor delay, it would seem as if the skew of the Weibull distribution acted in such a way as to actually offset some of the observed added observation bias and seems more an artifact of the simulation conditions rather than an inherent statement relating OM bias to the proportion of fixation method in general.

% latex table generated in R 4.2.2 by xtable 1.8-4 package
% Wed Feb  8 15:10:31 2023
\begin{table}[H]
\centering
\begin{tabular}{llrrr}
  \hline
Method & Delay & 1st Qu. & Median & 3rd Qu. \\ 
  \hline
Look Onset & No Delay & 0.17 & 0.32 & 0.56 \\ 
  Look Onset & Normal Delay & 0.37 & 0.71 & 1.24 \\ 
  Look Onset & Weibull Delay & 1.05 & 2.16 & 4.23 \\ 
  Proportion & No Delay & 8.21 & 11.33 & 16.01 \\ 
  Proportion & Normal Delay & 22.90 & 30.65 & 39.37 \\ 
  Proportion & Weibull Delay & 15.27 & 24.75 & 38.14 \\ 
   \hline
\end{tabular}
\caption{Summary of mean integrated squared error across simulations}
\label{tab:mise_sims}
\end{table}

The proportion of fixation method seems no longer tenable in the recovery the underlying lexical activation, given both the magnitude of differences presented in Table~\ref{tab:mise_sims} as well as the theoretical arguments made and illustrated in Figure~\ref{fig:folly_of_fixation}. And given that data collected via the look onset method can be adequately fit in the newest version of the \xt{bdots} software, there appears to be little to argue against its adoption.

[this next part maybe belongs elsewhere, mainly exists to demonstrate that despite appearances, OM delay still worth further investigation]

Outside of a demonstration of its existence and potential consequence, little more has been said about addressing the delayed observation bias. Further, the consequences of the delayed observation (under the assumption that the mean value is correctly accounted for) seem almost trivial in comparison to the differences between it and the added observation bias. That being said, we believe there are still critical reasons for considering its significance.

As mentioned earlier, the particular values observed in these simulations are both a function of the relationship between the distribution generating $\gamma$ and that of $\rho$. However, they are also a function of the generating function itself. In particular, we draw attention to the degree of total variation $f$ over the interval $[a,b]$, defined as 

\begin{equation}
V(f) = \underset{\mathcal{P}}{\sup} \sum_{i=0}^{n_p-1} \left|f(t_{i+1} - f(t_i) \right|,
\end{equation}

where $\mathcal{P} = \{P = \{t_0, \dots, t_{n_p}\} \}$ is the set of all possible partitions of $[a,b]$. Despite appearances, this is a relatively straightforward metric in the case of monotone functions such as the logistic, where the total variation is simply $|f(b) - f(a)|$. To illustrate the relevance of this, consider a hypothetical situation in which the underlying activation we are wishing to recover is a constant function, $f(t) = c$, where the probability of fixating on a target is independent of time. In such a situation, a delayed observation would be of no issue; despite changes in time $t$, the probability $c$ remains unchanged. In contrast, consider a second hypothetical situation in which activation is defined exponentially, $f(t) = \exp(t)$. In this case, the impact of delayed observation depends drastically on time, when the delay in observation in the range of small values of $t$ have a drastically smaller impact than delayed observations when $t$ is large ($\exp(1) - \exp(0) = 1.7183$ while $\exp(11) - \exp(10) = 37848$, despite both cases having $\Delta t = 1$).

In short, these hypothetical situations detail how the magnitude of total variation can have differential effects on the delay in observation. Now consider again the logistic function in Figure~\ref{fig:logistic_definition} and imagine its domain partitioned into three equally sized portions. Both the first and third, near the asymptotes, have relatively low total variation, resulting in a relatively benign effect from oculomotor delay. In contrast, the middle third contains nearly all of the variation of the function, indicating the delayed observation here will have a disproportionate impact on the successful recovery of the function. Given the clinical relevance of the both the slope and crossover parameters, as well as acknowledging the impact that these have on the overall shape of the function, we demonstrate a need in a accounting for this delay precisely where it will impact function recovery the most. This, of course, is not unique to the logistic, with the effects of the delayed observation bias compounded in the asymmetric Gaussian functions (appendix) which has a more complicated variation structure and, accordingly, more difficulty in recovering the generating curves.


\section{Discussion}

This section needs to be tightened quite a bit.

Through our investigation, we have presented a physiologically grounded model relating eye tracking data to underlying lexical access by placing emphasis singularly on the first instance of a look, discarding information on the rest of the look entirely. 

Of course, we know that longer fixations are relevant to the strength of consideration. One of the primary benefits of the proportion method is that it indirectly captures the duration of fixations, with longer times being associated with stronger activation. This is important when differentiating fixations associated with searching patterns (i.e., what images exist on screen?) against those associated with consideration (is this the image I've just heard?). However, implicit in the proportion of fixations method is a crucially overlooked assumption of a linear relationship between the fixation length and the activation. That is, insofar as the construction of the fixation curve is considered, a fixation persisting at 20ms after look onset (and well within the refraction period in which no new information regarding the cognitive mechanism or voluntary fixation could be obtained (see figure i haven't made yet)) is considered identical to a fixation persisting at 500ms after onset: both are undifferentiated in being recorded as either a $0$ or $1$. More likely it seems this would be more of an exponential relationship, with longer fixations offering increasingly more evidence of lexical activation. By separating the moment of onset from the look itself, we free ourselves to construct far more nuanced models. Put more succinctly, it is not that the look onset method considers the length of fixation irrelevant; rather, it makes no statement about it at all. How these mechanisms could be combined should be the focus of future research.

Another utility that comes to mind when considering the look onset method is how much sparser of a dataset is created. Consider, for example, fitting a mixed model to the four parameter logistic modeling individual trial data for each subject all together. Such models with proportion of fixation data typically infeasible with cumbersome autocorrelative structures found within a trial (from Bob's comment, need to find source). The present method, being a much sparser dataset, may be far more computationally tractable. 


The arguments presented here has hoped to satisfy two goals, agnostic to the linking hypothesis or functions ultimately decided upon. Foremost is the recognition that the mechanisms governing a look onset and the looks themselves follow separate mechanisms, and treating them as such requires fewer assumptions while still retaining the utility in fitting the same nonlinear curves to the observed data.

Second to this, we have put a name to two important sources of potential bias in the recovering of activation curves, generalizable beyond the particulars of the presented simulations. The first is the added observation bias, speaking again to the utility in separating the relevant mechanisms making up a look. And second is the delayed observation bias, bringing into focus the importance of reevaluating how we address bias and variability in the oculomotor delay.


In short, what we have hoped to accomplish here is not to drastically change the original assumptions presented in Allopenna (1996, but rather to qualify them in statistically sound ways. And really, that is pretty much it. Concluding sentence.


As a not really conclusion, I am sometimes left to wonder to what degree the proportion of fixation method was a  ``local minimum" in the pursuit of utilizing eye-tracking data. The proportion of fixations created an ostensible curve, prompting McMurray to establish theoretically grounded non-linear functions to model them. These, in turn, where shown to be suitable functions with which to model saccade data over a period of trials. Had saccades lent themselves so naturally to visualization as the proportion of fixations, perhaps that is where we may have started.


\section{Appendices}

Omitting relation of this method to TRACE as it seems mostly unnecessary to the present argument 


%\section*{Appendix D -- TRACE}

[Moving everything TRACE related here because it turns out we don't actually need it]



\paragraph{TRACE } How speech is perceived and understood has been a subject of much debate for a significant portion of psycholinguistics' history. Starting in the 1980s and persisting today, many researchers subscribe to what is known as the connectionist model of speech perception. Briefly, this model posits that speech perception is best understood as a hierarchical dynamical system in which aspects of the model are either self reinforcing or self inhibiting with feed-forward and feedback mechanisms. For example, hearing the phoneme \textbackslash h\textbackslash  \ as in ``hit" will ``feed-foward", cognitively activating words that begin with the \textbackslash h\textbackslash \  sound. These activated words then ``feedback" to the phoneme letter, inhibiting activation for competing phonemes such as \textbackslash b\textbackslash \ or \textbackslash t\textbackslash. In 1986, McClelland and Elman introduced the TRACE\footnote{TRACE doesn't stand for anything -- the name is a reference to ``the trace", a network structure for dynamically processing things in memory} model implementing theoretical considerations into a computer model \cite{elman1985speech}. Maybe useful here to discuss activation, sigmoidal shape, etc., 


\subsection{TRACE -- section}

I have a few issues with this section, and as I have fleshed out my reading and understanding of things, my intention with this section has changed. Originally, my hope was to show that non-linear functions fit with empirical saccade data would be a better match to what is predicted by TRACE than what is found using fixation data. This, however, seems to be the wrong thing to do. There is an apparently magnificent number of ways with which to transform TRACE activation data to probabilities of fixation, Neverminding the fact that the saccade curve is a fundamentally \textit{different} concept/mechanism than the proportion of fixations, calling into question the value of a direct comparison (though I'm not sure that this is really much of an issue, as none of these transformations had mechanics uniquely specific to the properties of fixations).

This general idea is related to an observation made earlier by Allopenna and friends, 

``It is important to note that although the TRACE simulations provided good fits to the behavioral data, the results should be taken as evidence in support of the class of continuous mapping models, rather than support for particular architectural assumptions made by TRACE."

Of course, this finds us in a bit of a circular loop -- Allopenna suggested that the consistency of evidence was used to support the assumptions of TRACE, and here we are suggesting TRACE as evidence for the saccade method. What seems more appropriate, then, is to demonstrate that there \textit{exist} transformations of TRACE data in which \textit{either} the saccade or fixation methods creates a better fit (as measured by $R^2$ or MISE). As such, it makes less sense to use TRACE to show which is ``better", but rather to use TRACE to demonstrate that what is estimated with the saccade method continues to be consistent with the continuous mapping model of lexical activation. Having established theoretical consistency, my argument for the saccade method will rest on what was presented in the previous section, namely the separation of saccades and fixations, and the problems illustrated with the added observation bias. 

As it is, I have a few sections here addressing high level concerns. How it should be precisely organized is up for debate, but the work itself should be largely finished.

Finally, I will note here without much other detail -- the entirety of this next section rests with the empirical and simulated data from McMurray 2010. For the empirical data, only N/TD subjects were used and for TRACE, only the 14 simulations with the default hyper-parameters from jTRACE. The specifics of data processing (less what I mention in relevant sections here) can be cast to the appendices

\subsection{On fitting saccade data}


I will go into more detail later on the precise transformations that I did to arrive at the empirical data from the raw data.  One key thing to note, however, is that rather than using the separate saccade data in the Access DB, I finished cleaning the fixation data and then transformed this to saccade data based on the start time of subsequent fixations. This helped address ambiguities that resulted from deciding which saccades to be included. For example, if we neglected to include any saccades that began before the onset of the audio signal, the first saccades recorded (generally) had a probability of fixating on the target of about 0.25, resulting in an empirical curve with a base parameter much closer to 0.2. Additionally, this first saccade often occured some time after onset, leading to virtually no observed data near $t = 0$. This was addressed by artificially setting the first saccade to have occurred at $t = 0$ with its direction set to match that of the matching fixation at the same time. After making this adjustment, the baseline of the saccade curve matched nearly that of the fixation curve, making the saccade curve more closely match the shape of the fixation curve.

Slightly more of an indulgence was how to treat the end of the saccade curves at each trial. Necessarily in all cases, following the last saccade, recorded fixations were constant until the end of the trial, drastically increasing the ``added observation" bias and resulting in a fixation curve with a peak much closer to 1.

On one hand, this could perhaps have been dealt with by addressing response times and making the appropriate adjustments. Or, far more simply (and with fewer researcher degrees of freedom), I simply added one last saccade to the end of each trial with it's location being that of the last fixation (typically the target object). The rest of the analysis does not depend on this decision in any fashion, and as the results are functionally the same, I elected to use the saccade curve with the inflated data. This most closely matches our expectation of the relation between the fixation and saccade curves and addresses (to some degree) the asymptotic behavior of the saccades which would otherwise be uncollected. A demonstration of these differences is given in Figure~\ref{fig:saccade_inflate}.

\begin{figure}
\centering
\includegraphics{sac_inflate_compare.pdf}
\caption{this is what happens when i inflate saccade with additional saccade at last endpoint}
\label{fig:saccade_inflate}
\end{figure}

\subsection{Transforming TRACE data}


My primary concern with the TRACE data is I am seemingly unable to reconcile it visually with what is presented in the 2010 SLI paper. Specifically, I never achieve a baseline near 0. I tried manipulating the temperature of the luce choice rule (LCR) with both constant factors and sigmoidal shapes with differing parameters; I also tried playing with some of the parameters from the scaling factor function.


Referring back to an email we exchanged 12/14/2022, you (you being theBob) gave me a list of adjustments to make to the scaling factor, including swapping the activation and crossover, as well as expanding the exponential term to include the entire denominator. I did this and confirmed that, as you had, the function goes from 0.0002 at maxact=-0.2 to .739 at maxact = .55. The issue, though, is that this is performed \textit{after} luce choice rule implemented. In that situation, the minimum activation observed is 0.25 rather than -0.2. This made me think that perhaps some other permutation of transformations would result in a curve starting closer to 0 and peaking nearer to .75 (for example, scaling the raw TRACE activations). In my collection of attempts, I never found anything to quite correct for this. It may be a bit much, but I have included plots of the TRACE data related to the target at different points in the transformation process to see how it changes. Maybe something in that will ring at bell. These are included in  Figure~\ref{fig:shades_of_trace}. 


An interesting aside, though -- if we do not make the adjustment to the saccade data where we anchor asymptotic behavior at 0/1, we get a saccade curve bearing less relation to the fixation curve, but with much higher agreement with the set of TRACE curves, in particular with regards to the baseline point and peak. Presumably with some tweaking, it could be made to match even more closely. This phenomenon is illustrated in Figure~\ref{fig:unadjusted_saccade_against_trace}

\begin{figure}[H]
  \centering
  \includegraphics{unadjusted_sac_w_trace.pdf}
  \caption{Plot illustrating how the unadjusted saccade method (without anchoring at asymptotes) both matches more closely with the TRACE predictions (particularly near the baseline) while also taking on a far different shape than the fixation curve. This is in contrast to the Princess Bride simulations in which the distortion was minimal and the saccade curve appeared to be more of a horizontal shift}
  \label{fig:unadjusted_saccade_against_trace}
\end{figure}


\
\begin{figure}[H]
    \centering
    \subfigure[]{\includegraphics[width=0.45\textwidth]{TRACE_test/raw_trace.pdf}} 
    \subfigure[]{\includegraphics[width=0.45\textwidth]{TRACE_test/luce_choice.pdf}} 
    \subfigure[]{\includegraphics[width=0.45\textwidth]{TRACE_test/scaling_factor.pdf}}
    \subfigure[]{\includegraphics[width=0.45\textwidth]{TRACE_test/scaling_times_luce.pdf}}
    \subfigure[]{\includegraphics[width=0.45\textwidth]{TRACE_test/skipping_luce.pdf}}
    \begin{singlespace}
    \caption{(a) This is simply the raw TRACE data across the 14 simulations with standard parameters. (b) Transformation of TRACE activation using LCR with sigmoidal temperature. (c) Scaling factor function built on max activations \textit{after} performing LCR, using peak/baseline values from target object. (d) TRACE activations following LCR transformation and multiplying by scaling factor. This is what I have been using as model prediction of fixations, though note the baseline value being near 0.15. (e) Perhaps unnecessary, this is simply investigating TRACE activation by the scaling factor but without first conducting LCR. Note that none of these seem to have both the correct baseline and peak values} \end{singlespace}
\label{fig:shades_of_trace}
\end{figure}


\subsection{Comparisons}

Here is where I would suggest the consistency of the models. What I show here is a moderated version of this, namely I show that there are two transformations of TRACE (changing the parameters of the sigmoidal function for Luce choice rule, or the temperature directing the rate at which competitors are weeded out) that each match one or the other of the fixation/saccade curves better. As such, neither is superior in any sense, but both are in the realm of consistency.


\begin{figure}[H]
    \centering
    \subfigure[]{\includegraphics[width=0.45\textwidth]{sac_fix_trace_1.pdf}} 
    \subfigure[]{\includegraphics[width=0.45\textwidth]{sac_fix_trace_2.pdf}} 

    \caption{Examples of different temperatures used in LCR and how this effects TRACE activation. In (a), this leads to greater consistency with the saccade curve; in (b), with the fixation curve. This is evidenced also by RMS error values}
\label{fig:shades_of_trace2}
\end{figure}

%\begin{figure}[H]
%\centering
%\includegraphics{sac_fix_trace_compare.pdf}
%\caption{this is just the mean value of the curve parameters. also only includes NH subjects. I feel like confidence intervals would make this chart look messy so might just instead include table of mean integrated squared error using approxfun since trace only has 108 data points and need a function to integrate in R}
%\end{figure}

Presented in Table~\ref{tab:mise_trace} is a summary of the RMS error of the 1000s simulations using both the saccade and fixation methods against two instantiations of TRACE. As we see and corresponding to (a) in Figure~\ref{fig:shades_of_trace2} we have better agreement between the saccade method and trace predictions; this relation is flipped for case (b). 


% latex table generated in R 4.2.2 by xtable 1.8-4 package
% Wed Jan 18 18:41:24 2023
\begin{table}[ht]
\centering
\begin{tabular}{rllrrrrrr}
  \hline
 & Method & TRACE & Min. & 1st Qu. & Median & Mean & 3rd Qu. & Max. \\ 
  \hline
1 & Proportion of Fixation & TRACE1 & 0.1148 & 0.1743 & 0.2181 & 0.2226 & 0.2583 & 0.4407 \\ 
  2 & Look Onset & TRACE1 & 0.0749 & 0.1051 & 0.1396 & 0.1449 & 0.1655 & 0.2933 \\ 
  3 & Proportion of Fixation & TRACE2 & 0.0991 & 0.1270 & 0.1529 & 0.1606 & 0.1712 & 0.3875 \\ 
  4 & Look Onset & TRACE2 & 0.0957 & 0.1404 & 0.1830 & 0.1879 & 0.2275 & 0.3734 \\ 
   \hline
\end{tabular}
\caption{Summary of RMS of two transformations of TRACE against saccade and fixation method}
\label{tab:mise_trace}
\end{table}

As an aside, this also lends itself to the idea of having a \textit{distribution} of curves associated with lexical activation rather than pursuing point estimation.  In some sense, this allows a natural way to account for the observed variability in experimental conditions without having to attempt to model it. Not sure if this is an idea worth elaborating on.



\section{Recovery of Individual Curves -- Asymmetric Gaussian}

Presented here are the results of the simulations for the recovery of subject-specific curves generated with the asymmetric Gaussian function, the parametric function typically associated with looks to competitors in the VWP. As in the section fitted with the logistic function, simulations include settings in which there is no oculomotor delay, as well as delay that is normally and Weibull distributed. Again, as all fits could not be individual examined, an automated criterion was used to determine which fits were considered  adequate. Here, this stipulated that the estimated sigma parameters be positive and that the height parameter be larger than either of the base parameters. The number of fits retained for the no delay, normal delay, and Weibull delay were  855, 786, and 816, respectively.


\subsection{Results}

As might be expected, the more complicated mean structure provided by the asymmetric Gaussian led to a generalized increase in the difficulty of recovery for both the look onset and proportion methods, relative to those generated with a logistic mean structure. However, we do still find that in the case of no delay, given in Figure~\ref{fig:dg_rep_curves_no_delay}, that the recovery of individual parameters is still unbiased and, as with the logistic, the location parameter (here, mu), is right shifted.

The results for the median integrated squared error are given in Table~\ref{tab:dg_mise_sims}. We again see results similar to those with the logistic in that the look onset method outperforms the proportion of fixation method in all cases. 

\begin{figure}[H]
\centering
\includegraphics[width=0.9\textwidth]{dg_rep_and_diff_no_delay.pdf}
\caption{Summary of simulation results in the recovery of subject-specific curves generated by the asymmetric Gauss with no oculomotor delay}
\label{fig:dg_rep_curves_no_delay}
\end{figure}

\begin{figure}[H]
\centering
\includegraphics[width=0.9\textwidth]{dg_rep_and_diff_no_delay.pdf}
\caption{Summary of simulation results in the recovery of subject-specific curves generated by the asymmetric Gauss with normally distributed oculomotor delay}
\label{fig:dg_rep_curves_normal_delay}
\end{figure}


\begin{figure}[H]
\centering
\includegraphics[width=0.9\textwidth]{dg_rep_and_diff_weibull_delay.pdf}
\caption{Summary of simulation results in the recovery of subject-specific curves generated by the asymmetric Gauss with Weibull distributed oculomotor delay}
\label{fig:dg_rep_curves_weibull_delay}
\end{figure}




\begin{table}[ht]
\centering
\begin{tabular}{llrrr}
  \hline
Curve & Delay & 1st Qu. & Median & 3rd Qu. \\ 
  \hline
Look Onset & No Delay & 0.22 & 0.36 & 0.63 \\ 
  Look Onset & Normal Delay & 0.38 & 0.70 & 1.15 \\ 
  Look Onset & Weibull Delay & 0.52 & 0.84 & 1.39 \\ 
  Proportion & No Delay & 0.75 & 1.29 & 2.08 \\ 
  Proportion & Normal Delay & 1.38 & 2.44 & 3.96 \\ 
  Proportion & Weibull Delay & 1.00 & 1.98 & 3.43 \\ 
   \hline
\end{tabular}
\caption{Median integrated squared error for recovery of individual curves generated with asymmetric Gaussian}
\label{tab:dg_mise_sims}
\end{table}

\subsection{$R^2$ instead of MISE for Recovery of Individual Curves}

Here, we provide an alternative summary of the recovery of subject specific curves fit with both the logistic and asymmetric Gauss. 


\subsubsection{Logistic}

\begin{table}[H]
\centering
\begin{tabular}{llrrr}
  \hline
Curve & Delay & 1st Qu. & Median & 3rd Qu. \\ 
  \hline
Look Onset & No Delay & 1.00 & 1.00 & 1.00 \\ 
  Look Onset & Normal Delay & 0.99 & 1.00 & 1.00 \\ 
  Look Onset & Weibull Delay & 0.98 & 0.99 & 0.99 \\ 
  Proportion & No Delay & 0.92 & 0.94 & 0.95 \\ 
  Proportion & Normal Delay & 0.80 & 0.83 & 0.86 \\ 
  Proportion & Weibull Delay & 0.80 & 0.86 & 0.91 \\ 
   \hline
\end{tabular}
\caption{$R^2$ for Logistic}
\label{tab:r2_logistic_sims}
\end{table}

\subsubsection{Asymmetric Gaussian}

\begin{table}[H]
\centering
\begin{tabular}{llrrr}
  \hline
Curve & Delay & 1st Qu. & Median & 3rd Qu. \\ 
  \hline
Look Onset & No Delay & 0.80 & 0.91 & 0.95 \\ 
  Look Onset & Normal Delay & 0.63 & 0.82 & 0.91 \\ 
  Look Onset & Weibull Delay & 0.57 & 0.77 & 0.87 \\ 
  Proportion & No Delay & 0.48 & 0.65 & 0.75 \\ 
  Proportion & Normal Delay & 0.10 & 0.33 & 0.52 \\ 
  Proportion & Weibull Delay & 0.20 & 0.46 & 0.64 \\ 
   \hline
\end{tabular}
\caption{$R^2$ for Asymmetric Gaussian}
\label{tab:r2_dg_sims}
\end{table}


%\bibliography{../bib/dissertation}



\newpage

\Chapter{Methodological Changes in the Bootstrapped Differences of Time Series}

\input{method_b.tex}

\Chapter{CRAN Vignettes}

Included here are the collection of vignettes that are included with the \xt{bdots} packages. In order, these are:

\begin{enumerate}
\item \textbf{\xt{bdots}} This is the primary vignette used to introduce new users to the package. It includes information on fitting parametric functions to subject data, a brief tutorial on refitting, and a description of the bootstrapping process as well as the \xt{bdots} formula syntax
\item \textbf{Refitting with Saved Parameters} This vignette illustrates how to save the fitted coefficients from the fitting step (or after refitting) and how to load them back into R to recreate a fitted \xt{bdotsObj} object. This has been primarily used by users who have created subject-specific parameters in other software (i.e., MATLAB) who wish to import them to \xt{bdots}
\item \textbf{Correlations} This vignette details the \xt{bdotsCor} function to find the correlation of a fixed value (i.e., vocabulary scores or IQ tests) with the group fitted curves at each time point
\item \textbf{User Curve Functions} This vignette offers detailed instructions for users who wish to create and import their own custom parametric curve functions
\end{enumerate}



\includepdf[pages=-]{vignettes/bdots.pdf}
\includepdf[pages=-]{vignettes/refitCoef.pdf}
\includepdf[pages=-]{vignettes/correlations.pdf}
\includepdf[pages=-]{vignettes/customCurves.pdf}

\bibliography{../bib/dissertation}

\end{document}






