% Options for packages loaded elsewhere
\PassOptionsToPackage{unicode}{hyperref}
\PassOptionsToPackage{hyphens}{url}
%
\documentclass[
]{article}
\usepackage{amsmath,amssymb}
\usepackage{lmodern}
\usepackage{iftex}
\ifPDFTeX
  \usepackage[T1]{fontenc}
  \usepackage[utf8]{inputenc}
  \usepackage{textcomp} % provide euro and other symbols
\else % if luatex or xetex
  \usepackage{unicode-math}
  \defaultfontfeatures{Scale=MatchLowercase}
  \defaultfontfeatures[\rmfamily]{Ligatures=TeX,Scale=1}
\fi
% Use upquote if available, for straight quotes in verbatim environments
\IfFileExists{upquote.sty}{\usepackage{upquote}}{}
\IfFileExists{microtype.sty}{% use microtype if available
  \usepackage[]{microtype}
  \UseMicrotypeSet[protrusion]{basicmath} % disable protrusion for tt fonts
}{}
\makeatletter
\@ifundefined{KOMAClassName}{% if non-KOMA class
  \IfFileExists{parskip.sty}{%
    \usepackage{parskip}
  }{% else
    \setlength{\parindent}{0pt}
    \setlength{\parskip}{6pt plus 2pt minus 1pt}}
}{% if KOMA class
  \KOMAoptions{parskip=half}}
\makeatother
\usepackage{xcolor}
\usepackage[margin=1in]{geometry}
\usepackage{color}
\usepackage{fancyvrb}
\newcommand{\VerbBar}{|}
\newcommand{\VERB}{\Verb[commandchars=\\\{\}]}
\DefineVerbatimEnvironment{Highlighting}{Verbatim}{commandchars=\\\{\}}
% Add ',fontsize=\small' for more characters per line
\usepackage{framed}
\definecolor{shadecolor}{RGB}{248,248,248}
\newenvironment{Shaded}{\begin{snugshade}}{\end{snugshade}}
\newcommand{\AlertTok}[1]{\textcolor[rgb]{0.94,0.16,0.16}{#1}}
\newcommand{\AnnotationTok}[1]{\textcolor[rgb]{0.56,0.35,0.01}{\textbf{\textit{#1}}}}
\newcommand{\AttributeTok}[1]{\textcolor[rgb]{0.77,0.63,0.00}{#1}}
\newcommand{\BaseNTok}[1]{\textcolor[rgb]{0.00,0.00,0.81}{#1}}
\newcommand{\BuiltInTok}[1]{#1}
\newcommand{\CharTok}[1]{\textcolor[rgb]{0.31,0.60,0.02}{#1}}
\newcommand{\CommentTok}[1]{\textcolor[rgb]{0.56,0.35,0.01}{\textit{#1}}}
\newcommand{\CommentVarTok}[1]{\textcolor[rgb]{0.56,0.35,0.01}{\textbf{\textit{#1}}}}
\newcommand{\ConstantTok}[1]{\textcolor[rgb]{0.00,0.00,0.00}{#1}}
\newcommand{\ControlFlowTok}[1]{\textcolor[rgb]{0.13,0.29,0.53}{\textbf{#1}}}
\newcommand{\DataTypeTok}[1]{\textcolor[rgb]{0.13,0.29,0.53}{#1}}
\newcommand{\DecValTok}[1]{\textcolor[rgb]{0.00,0.00,0.81}{#1}}
\newcommand{\DocumentationTok}[1]{\textcolor[rgb]{0.56,0.35,0.01}{\textbf{\textit{#1}}}}
\newcommand{\ErrorTok}[1]{\textcolor[rgb]{0.64,0.00,0.00}{\textbf{#1}}}
\newcommand{\ExtensionTok}[1]{#1}
\newcommand{\FloatTok}[1]{\textcolor[rgb]{0.00,0.00,0.81}{#1}}
\newcommand{\FunctionTok}[1]{\textcolor[rgb]{0.00,0.00,0.00}{#1}}
\newcommand{\ImportTok}[1]{#1}
\newcommand{\InformationTok}[1]{\textcolor[rgb]{0.56,0.35,0.01}{\textbf{\textit{#1}}}}
\newcommand{\KeywordTok}[1]{\textcolor[rgb]{0.13,0.29,0.53}{\textbf{#1}}}
\newcommand{\NormalTok}[1]{#1}
\newcommand{\OperatorTok}[1]{\textcolor[rgb]{0.81,0.36,0.00}{\textbf{#1}}}
\newcommand{\OtherTok}[1]{\textcolor[rgb]{0.56,0.35,0.01}{#1}}
\newcommand{\PreprocessorTok}[1]{\textcolor[rgb]{0.56,0.35,0.01}{\textit{#1}}}
\newcommand{\RegionMarkerTok}[1]{#1}
\newcommand{\SpecialCharTok}[1]{\textcolor[rgb]{0.00,0.00,0.00}{#1}}
\newcommand{\SpecialStringTok}[1]{\textcolor[rgb]{0.31,0.60,0.02}{#1}}
\newcommand{\StringTok}[1]{\textcolor[rgb]{0.31,0.60,0.02}{#1}}
\newcommand{\VariableTok}[1]{\textcolor[rgb]{0.00,0.00,0.00}{#1}}
\newcommand{\VerbatimStringTok}[1]{\textcolor[rgb]{0.31,0.60,0.02}{#1}}
\newcommand{\WarningTok}[1]{\textcolor[rgb]{0.56,0.35,0.01}{\textbf{\textit{#1}}}}
\usepackage{longtable,booktabs,array}
\usepackage{calc} % for calculating minipage widths
% Correct order of tables after \paragraph or \subparagraph
\usepackage{etoolbox}
\makeatletter
\patchcmd\longtable{\par}{\if@noskipsec\mbox{}\fi\par}{}{}
\makeatother
% Allow footnotes in longtable head/foot
\IfFileExists{footnotehyper.sty}{\usepackage{footnotehyper}}{\usepackage{footnote}}
\makesavenoteenv{longtable}
\usepackage{graphicx}
\makeatletter
\def\maxwidth{\ifdim\Gin@nat@width>\linewidth\linewidth\else\Gin@nat@width\fi}
\def\maxheight{\ifdim\Gin@nat@height>\textheight\textheight\else\Gin@nat@height\fi}
\makeatother
% Scale images if necessary, so that they will not overflow the page
% margins by default, and it is still possible to overwrite the defaults
% using explicit options in \includegraphics[width, height, ...]{}
\setkeys{Gin}{width=\maxwidth,height=\maxheight,keepaspectratio}
% Set default figure placement to htbp
\makeatletter
\def\fps@figure{htbp}
\makeatother
\setlength{\emergencystretch}{3em} % prevent overfull lines
\providecommand{\tightlist}{%
  \setlength{\itemsep}{0pt}\setlength{\parskip}{0pt}}
\setcounter{secnumdepth}{-\maxdimen} % remove section numbering
\ifLuaTeX
  \usepackage{selnolig}  % disable illegal ligatures
\fi
\IfFileExists{bookmark.sty}{\usepackage{bookmark}}{\usepackage{hyperref}}
\IfFileExists{xurl.sty}{\usepackage{xurl}}{} % add URL line breaks if available
\urlstyle{same} % disable monospaced font for URLs
\hypersetup{
  pdftitle={bdots},
  hidelinks,
  pdfcreator={LaTeX via pandoc}}

\title{bdots}
\author{}
\date{\vspace{-2.5em}}

\begin{document}
\maketitle

\begin{Shaded}
\begin{Highlighting}[]
\FunctionTok{library}\NormalTok{(bdots)}
\CommentTok{\#\textgreater{} Loading required package: data.table}
\CommentTok{\#\textgreater{} }
\CommentTok{\#\textgreater{} Attaching package: \textquotesingle{}bdots\textquotesingle{}}
\CommentTok{\#\textgreater{} The following object is masked from \textquotesingle{}package:data.table\textquotesingle{}:}
\CommentTok{\#\textgreater{} }
\CommentTok{\#\textgreater{}     rbindlist}
\end{Highlighting}
\end{Shaded}

\hypertarget{overview}{%
\subsection{Overview}\label{overview}}

This vignette walks through the use of the bdots package for analyzing
the bootstrapped differences of time series data. The general workflow
will follow three steps:

\begin{enumerate}
\def\labelenumi{\arabic{enumi}.}
\item
  \begin{description}
  \tightlist
  \item[Curve Fitting]
  During this step, we define the type of curve that will be used to fit
  our data along with variables to be used in the analysis
  \end{description}
\item
  \begin{description}
  \tightlist
  \item[Curve Refitting]
  Often, some of the curves returned from the first step have room for
  improvement. This step allows the user to either quickly attempting
  refitting a subset of the curves from step one or to manually make
  adjustments themselves
  \end{description}
\item
  \begin{description}
  \tightlist
  \item[Bootstrap]
  Having an adequate collection of curves, this function determines the
  bootstrapped difference, along with computing an adjusted alpha to
  account for AR1 correlation
  \end{description}
\end{enumerate}

This process is represented with three main functions,
\texttt{bdotsFit\ -\textgreater{}\ bdotsRefit\ -\textgreater{}\ bdotsBoot}

This package is under active development. The most recent version can be
installed with \texttt{devtools::install\_github("collinn/bdots")}.

\hypertarget{fitting-step}{%
\subsection{Fitting Step}\label{fitting-step}}

For our example, we are going to be using eye tracking data from normal
hearing individuals and those with cochlear implants using data from the
Visual Word Paradigm (VWP).

\begin{Shaded}
\begin{Highlighting}[]
\FunctionTok{head}\NormalTok{(cohort\_unrelated)}
\CommentTok{\#\textgreater{}    Subject Time DB\_cond Fixations LookType Group}
\CommentTok{\#\textgreater{} 1:       1    0      50  0.011364   Cohort    50}
\CommentTok{\#\textgreater{} 2:       1    4      50  0.011364   Cohort    50}
\CommentTok{\#\textgreater{} 3:       1    8      50  0.011364   Cohort    50}
\CommentTok{\#\textgreater{} 4:       1   12      50  0.011364   Cohort    50}
\CommentTok{\#\textgreater{} 5:       1   16      50  0.022727   Cohort    50}
\CommentTok{\#\textgreater{} 6:       1   20      50  0.022727   Cohort    50}
\end{Highlighting}
\end{Shaded}

The \texttt{bdotsFit} function will create a curve for each unique
permutation of \texttt{subject}/\texttt{group} variables. Here, we will
let \texttt{LookType} and \texttt{DB\_cond} be our grouping variables,
though we may include as many as we wish (or only a single group
assuming that it has multiple values). See \texttt{?bdotsFit} for
argument information.

\begin{Shaded}
\begin{Highlighting}[]
\NormalTok{fit }\OtherTok{\textless{}{-}} \FunctionTok{bdotsFit}\NormalTok{(}\AttributeTok{data =}\NormalTok{ cohort\_unrelated,}
                \AttributeTok{subject =} \StringTok{"Subject"}\NormalTok{,}
                \AttributeTok{time =} \StringTok{"Time"}\NormalTok{,}
                \AttributeTok{y =} \StringTok{"Fixations"}\NormalTok{,}
                \AttributeTok{group =} \FunctionTok{c}\NormalTok{(}\StringTok{"DB\_cond"}\NormalTok{, }\StringTok{"LookType"}\NormalTok{),}
                \AttributeTok{curveType =} \FunctionTok{doubleGauss}\NormalTok{(}\AttributeTok{concave =} \ConstantTok{TRUE}\NormalTok{),}
                \AttributeTok{cores =} \DecValTok{2}\NormalTok{)}
\end{Highlighting}
\end{Shaded}

A key thing to note here is the argument for \texttt{curveType} is
passed as a function call with arguments that further specify the curve.
Currently within the \texttt{bdots} package, the available curves are
\texttt{doubleGauss(concave\ =\ TRUE/FALSE),} \texttt{logistic()} (no
arguments), and \texttt{polynomial(degree\ =\ n)}. While more curves
will be added going forward, users can also specify their own curves, as
shown \href{customCurves.html}{here}.

The \texttt{bdotsFit} function returns an object of class
\texttt{bdotsObj}, which inherits from \texttt{data.table.} As such,
this object can be manipulated and explored with standard
\texttt{data.table} syntax. In addition to the subject and the grouping
columns, we also have a \texttt{fit} column, containing the fit from the
\texttt{gnls} package, a value for \texttt{R2}, a boolean indicating
\texttt{AR1} status, and a final column for \texttt{fitCode.} The fit
code is a numeric quantity representing the quality of the fit as such:

\begin{longtable}[]{@{}ccc@{}}
\toprule()
fitCode & AR1 & R2 \\
\midrule()
\endhead
0 & TRUE & R2 \textgreater{} 0.95 \\
1 & TRUE & 0.8 \textless{} R2 \textless{} 0.95 \\
2 & TRUE & R2 \textless{} 0.8 \\
3 & FALSE & R2 \textgreater{} 0.95 \\
4 & FALSE & 0.8 \textless{} R2 \textless{} 0.95 \\
5 & FALSE & R2 \textless{} 0.8 \\
6 & NA & NA \\
\bottomrule()
\end{longtable}

A \texttt{fitCode} of 6 indicates that a fit was not able to be made.

In addition to \texttt{plot} and \texttt{summary} functions, we also
have a method to return a matrix of coefficients from the model fits.
Because of the \texttt{data.table} syntax, we can examine subsets of
this object as well

\begin{Shaded}
\begin{Highlighting}[]
\FunctionTok{head}\NormalTok{(}\FunctionTok{coef}\NormalTok{(fit))}
\CommentTok{\#\textgreater{}          mu      ht   sig1   sig2      base1    base2}
\CommentTok{\#\textgreater{} [1,] 429.76 0.19860 159.89 314.64  0.0097098 0.033761}
\CommentTok{\#\textgreater{} [2,] 634.93 0.26350 303.81 215.38 {-}0.0206361 0.028924}
\CommentTok{\#\textgreater{} [3,] 647.07 0.25438 518.96 255.99 {-}0.2130875 0.013682}
\CommentTok{\#\textgreater{} [4,] 723.05 0.25821 392.95 252.94 {-}0.0548262 0.031973}
\CommentTok{\#\textgreater{} [5,] 501.48 0.22477 500.85 158.42 {-}0.3316790 0.025227}
\CommentTok{\#\textgreater{} [6,] 460.72 0.30677 382.73 166.08 {-}0.2433086 0.039922}

\FunctionTok{head}\NormalTok{(}\FunctionTok{coef}\NormalTok{(fit[DB\_cond }\SpecialCharTok{==} \DecValTok{50}\NormalTok{, ]))}
\CommentTok{\#\textgreater{}          mu      ht   sig1   sig2      base1     base2}
\CommentTok{\#\textgreater{} [1,] 429.76 0.19860 159.89 314.64  0.0097098 0.0337611}
\CommentTok{\#\textgreater{} [2,] 647.07 0.25438 518.96 255.99 {-}0.2130875 0.0136820}
\CommentTok{\#\textgreater{} [3,] 501.48 0.22477 500.85 158.42 {-}0.3316790 0.0252268}
\CommentTok{\#\textgreater{} [4,] 521.68 0.24838 270.74 209.39 {-}0.0385777 0.1045933}
\CommentTok{\#\textgreater{} [5,] 553.19 0.22727 207.44 226.72 {-}0.0101197 0.0286633}
\CommentTok{\#\textgreater{} [6,] 615.90 0.15877 286.21 392.57 {-}0.0105632 0.0076619}
\end{Highlighting}
\end{Shaded}

The plots for this object will compare the observed data with the fitted
curve. Here is an example of the first four:

\begin{Shaded}
\begin{Highlighting}[]
\FunctionTok{plot}\NormalTok{(fit[}\DecValTok{1}\SpecialCharTok{:}\DecValTok{4}\NormalTok{, ])}
\end{Highlighting}
\end{Shaded}

\begin{center}\includegraphics{bdots_files/figure-latex/unnamed-chunk-6-1} \end{center}

\hypertarget{refitting-step}{%
\subsection{Refitting Step}\label{refitting-step}}

Depending on the curve type and the nature of the data, we might find
that a collection of our fits aren't very good, which may impact the
quality of the bootstrapping step. Using the \texttt{bdotsRefit}
function, users have the option to either quickly attempt to
automatically refit specified curves or to manually review each one and
offer alternative starting parameters. The \texttt{fitCode} argument
provides a lower bound for the fit codes to attempt refitting. The
default is \texttt{fitCode\ =\ 1}, indicating that we wish to attempt
refitting all curves that did not have \texttt{fitCode\ ==\ 0}. The
object returned is the same as that returned by \texttt{bdotsFit}.

\begin{Shaded}
\begin{Highlighting}[]
\DocumentationTok{\#\# Quickly auto{-}refit (not run)}
\NormalTok{refit }\OtherTok{\textless{}{-}} \FunctionTok{bdotsRefit}\NormalTok{(fit, }\AttributeTok{fitCode =}\NormalTok{ 1L, }\AttributeTok{quickRefit =} \ConstantTok{TRUE}\NormalTok{)}

\DocumentationTok{\#\# Manual refit (not run)}
\NormalTok{refit }\OtherTok{\textless{}{-}} \FunctionTok{bdotsRefit}\NormalTok{(fit, }\AttributeTok{fitCode =}\NormalTok{ 1L)}
\end{Highlighting}
\end{Shaded}

For whatever reason, there are some data will will not submit nicely to
a curve of the specfied type. One can quickly remove all observations
with a fit code equal to or greater than the one provided in
\texttt{bdRemove}

\begin{Shaded}
\begin{Highlighting}[]
\FunctionTok{table}\NormalTok{(fit}\SpecialCharTok{$}\NormalTok{fitCode)}
\CommentTok{\#\textgreater{} }
\CommentTok{\#\textgreater{}  0  1  3  4  5 }
\CommentTok{\#\textgreater{} 18 14  1  1  2}

\DocumentationTok{\#\# Remove all failed curve fits}
\NormalTok{refit }\OtherTok{\textless{}{-}} \FunctionTok{bdRemove}\NormalTok{(fit, }\AttributeTok{fitCode =}\NormalTok{ 6L)}

\FunctionTok{table}\NormalTok{(refit}\SpecialCharTok{$}\NormalTok{fitCode)}
\CommentTok{\#\textgreater{} }
\CommentTok{\#\textgreater{}  0  1  3  4  5 }
\CommentTok{\#\textgreater{} 18 14  1  1  2}
\end{Highlighting}
\end{Shaded}

There is an additional option, \texttt{removePairs} which is
\texttt{TRUE} by default. This indicates that if an observation is
removed, all observations for the same subject should also be removed,
regardless of fit. This ensures that all subjects have their
corresponding pairs in the bootstrapping function for the use of the
paired t-test. If the data are not paired, this can be set to
\texttt{FALSE}.

\hypertarget{bootstrap}{%
\subsection{Bootstrap}\label{bootstrap}}

The final step is the bootstrapping process, performed with
\texttt{bdotsBoot}. First, let's examine the set of curves that we have
available from the first step

\begin{enumerate}
\def\labelenumi{\arabic{enumi}.}
\item
  \begin{description}
  \tightlist
  \item[Difference of Curves]
  Here, we are interested specifically in the difference between two
  fitted curves. For our example case here, this may be the difference
  between curves for \texttt{DB\_cond\ ==\ 50} and
  \texttt{DB\_cond\ ==\ 65} nested within either the \texttt{Cohort} or
  \texttt{Unrelated\_Cohort} \texttt{LookTypes} (but not both).
  \end{description}
\item
  \begin{description}
  \tightlist
  \item[Difference of Difference Curves]
  In this case, we are considering the difference of two difference
  curves similar to the one found above. For example, we may denote the
  difference between \texttt{DB\_cond} \texttt{50} and \texttt{65}
  within the \texttt{Corhort} group as \(\text{diff}_{\text{Cohort}}\)
  and the differences between \texttt{DB\_cond} \texttt{50} and
  \texttt{65} within \texttt{Unrelated\_Corhort} as
  \(\text{diff}_{\text{Unrelated_Cohort}}\). The difference of
  difference function will then return an analysis of
  \(\text{diff}_{\text{Cohort}}\) -
  \(\text{diff}_{\text{Unrelated_Cohort}}\)
  \end{description}
\end{enumerate}

We can express the type of curve that we wish to fit with a modified
formula syntax. It's helpful to read as ``the difference of LHS between
elements of RHS''

For the first type, we have

\begin{Shaded}
\begin{Highlighting}[]
\DocumentationTok{\#\# Only one grouping variable in dataset, take bootstrapped difference}
\NormalTok{Outcome }\SpecialCharTok{\textasciitilde{}} \FunctionTok{Group1}\NormalTok{(value1, value2)}

\DocumentationTok{\#\# More than one grouping variable in difference, must specify unique value}
\NormalTok{Outcome }\SpecialCharTok{\textasciitilde{}} \FunctionTok{Group1}\NormalTok{(value1, value2) }\SpecialCharTok{+} \FunctionTok{Group2}\NormalTok{(value3)}
\end{Highlighting}
\end{Shaded}

That is, we might read this as ``difference of Outcome for value1 and
value2 within Group1.''

With our working example, we would find the difference of
\texttt{DB\_cond\ ==\ 50} and \texttt{DB\_cond\ ==\ 65} within
\texttt{LookType\ ==\ "Cohort"} with

\begin{Shaded}
\begin{Highlighting}[]
\DocumentationTok{\#\# Must add LookType(Cohort) to specify}
\NormalTok{Fixations }\SpecialCharTok{\textasciitilde{}} \FunctionTok{DB\_cond}\NormalTok{(}\DecValTok{50}\NormalTok{, }\DecValTok{65}\NormalTok{) }\SpecialCharTok{+} \FunctionTok{LookType}\NormalTok{(Cohort)}
\end{Highlighting}
\end{Shaded}

For this second type of curve, we specify an ``inner difference'' to be
the difference of groups for which we are taking the difference of. The
syntax for this case uses a \texttt{diffs} function in the formula:

\begin{Shaded}
\begin{Highlighting}[]
\DocumentationTok{\#\# Difference of difference. Here, outer difference is Group1, inner is Group2}
\FunctionTok{diffs}\NormalTok{(Outcome, }\FunctionTok{Group2}\NormalTok{(value3, value4)) }\SpecialCharTok{\textasciitilde{}} \FunctionTok{Group1}\NormalTok{(value1, value2)}

\DocumentationTok{\#\# Same as above if three or more grouping variables}
\FunctionTok{diffs}\NormalTok{(Outcome, }\FunctionTok{Group2}\NormalTok{(value3, value4)) }\SpecialCharTok{\textasciitilde{}} \FunctionTok{Group1}\NormalTok{(value1, value2) }\SpecialCharTok{+} \FunctionTok{Group3}\NormalTok{(value5)}
\end{Highlighting}
\end{Shaded}

For the example illustrated in (2) above, the difference
\(\text{diff}_{50} - \text{diff}_{65}\) represents our inner difference,
each nested within one of the values for \texttt{LookType.} The ``outer
difference'' is then difference of these between \texttt{LookTypes}. The
syntax here would be

\begin{Shaded}
\begin{Highlighting}[]
\FunctionTok{diffs}\NormalTok{(Fixations, }\FunctionTok{DB\_cond}\NormalTok{(}\DecValTok{50}\NormalTok{, }\DecValTok{65}\NormalTok{)) }\SpecialCharTok{\textasciitilde{}} \FunctionTok{LookType}\NormalTok{(Cohort, Unrelated\_Cohort)}
\end{Highlighting}
\end{Shaded}

Here, we show a fit for each

\begin{Shaded}
\begin{Highlighting}[]
\NormalTok{boot1 }\OtherTok{\textless{}{-}} \FunctionTok{bdotsBoot}\NormalTok{(}\AttributeTok{formula =}\NormalTok{ Fixation }\SpecialCharTok{\textasciitilde{}} \FunctionTok{DB\_cond}\NormalTok{(}\DecValTok{50}\NormalTok{, }\DecValTok{65}\NormalTok{) }\SpecialCharTok{+} \FunctionTok{LookType}\NormalTok{(Cohort),}
                   \AttributeTok{bdObj =}\NormalTok{ refit,}
                   \AttributeTok{Niter =} \DecValTok{1000}\NormalTok{,}
                   \AttributeTok{alpha =} \FloatTok{0.05}\NormalTok{,}
                   \AttributeTok{padj =} \StringTok{"oleson"}\NormalTok{,}
                   \AttributeTok{cores =} \DecValTok{2}\NormalTok{)}

\NormalTok{boot2 }\OtherTok{\textless{}{-}} \FunctionTok{bdotsBoot}\NormalTok{(}\AttributeTok{formula =} \FunctionTok{diffs}\NormalTok{(Fixation, }\FunctionTok{LookType}\NormalTok{(Cohort, Unrelated\_Cohort)) }\SpecialCharTok{\textasciitilde{}} \FunctionTok{DB\_cond}\NormalTok{(}\DecValTok{50}\NormalTok{, }\DecValTok{65}\NormalTok{),}
                   \AttributeTok{bdObj =}\NormalTok{ refit,}
                   \AttributeTok{Niter =} \DecValTok{1000}\NormalTok{,}
                   \AttributeTok{alpha =} \FloatTok{0.05}\NormalTok{,}
                   \AttributeTok{padj =} \StringTok{"oleson"}\NormalTok{,}
                   \AttributeTok{cores =} \DecValTok{2}\NormalTok{)}
\end{Highlighting}
\end{Shaded}

For each, we can then produce a model summary, as well as a plot of
difference curves

\begin{Shaded}
\begin{Highlighting}[]
\FunctionTok{summary}\NormalTok{(boot1)}
\CommentTok{\#\textgreater{} }
\CommentTok{\#\textgreater{} bdotsBoot Summary}
\CommentTok{\#\textgreater{} }
\CommentTok{\#\textgreater{} Curve Type: doubleGauss }
\CommentTok{\#\textgreater{} Formula: Fixations \textasciitilde{} (Time \textless{} mu) * (exp({-}1 * (Time {-} mu)\^{}2/(2 * sig1\^{}2)) * (ht {-} base1) + base1) + (mu \textless{}= Time) * (exp({-}1 * (Time {-} mu)\^{}2/(2 * sig2\^{}2)) * (ht {-} base2) + base2) }
\CommentTok{\#\textgreater{} Time Range: (0, 2000) [501 points]}
\CommentTok{\#\textgreater{} }
\CommentTok{\#\textgreater{} Difference of difference: FALSE }
\CommentTok{\#\textgreater{} Paired t{-}test: TRUE }
\CommentTok{\#\textgreater{} Difference: DB\_cond {-}{-} 50 65 }
\CommentTok{\#\textgreater{} }
\CommentTok{\#\textgreater{} Autocorrelation Estimate: 0.99976 }
\CommentTok{\#\textgreater{} FWER adjust method: oleson }
\CommentTok{\#\textgreater{} Alpha: 0.05 }
\CommentTok{\#\textgreater{} Adjusted alpha: 0.023752 }
\CommentTok{\#\textgreater{} Significant Intervals:}
\CommentTok{\#\textgreater{} NULL}

\FunctionTok{plot}\NormalTok{(boot1)}
\end{Highlighting}
\end{Shaded}

\begin{center}\includegraphics{bdots_files/figure-latex/unnamed-chunk-14-1} \end{center}

\end{document}
