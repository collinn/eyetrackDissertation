\documentclass{article}
\title{What, me saccade?}
\date{}

\usepackage{setspace}
\doublespacing

\usepackage[margin=1in]{geometry}
\usepackage{amsmath}
\usepackage{graphicx}
\newcommand{\xt}{\texttt}% 
\usepackage{listings}

\usepackage{subfigure}
\usepackage{float}

\graphicspath{{img/}}
\begin{document}

%https://www.namsu.de/Extra/klassen/latex-article-template.html

\maketitle

%\begin{abstract}
abstract
\end{abstract}

\section{Introduction}
intro

\paragraph{Outline}
outline




\section{Where are we going?} 

Somewhere I need to be clear with my language in that typically a saccade is a period of movement lasting about 30ms, followed by a fixation lasting however much time. I will be talking specifically about saccade onset or fixation onset which is not associated with a duration but rather a specific instance in time.

Having given due consideration to the state of things are they are, we find ourselves in a time of moral reflection, reexamining the underlying relationship between lexical activation, the mechanism of interest, and the physiological behavior we are able to observe (here, specifically eye-tracking, rather than discussion on other behavioral tasks, i.e., Spivey mouse tracking). This is referred to in the literature as the linking hypothesis. And while there are a number of competing hypothesis, they each share a collection of implicit assumptions relating what is observed to what is being studied.

In particular, we consider a contribution presented by McMurray 2022 in which he probed the relationship between the observed dynamics of the fixations curves and the underlying dynamics of activation under a variety of assumptions. In short, he showed that curves reconstructed using the standard  (standard being determing proportion of fixations, may specify that in more detail earlier) analysis in the VWP were poor estimates of the underlying system, with the magnitude of bias increasing on the complexity of the mechanisms involved.

[transition paragraph?]

From allopenna -- ``We made the general assumption that the probability of initiating an eye movement to fixate on a target object $o$ at time $t$ is a direct function of the probability that $o$ is the target given the speech input and where the probability of fixating $o$ is determined by the activation level of its lexical entry relative to the activation of the other potential targets."

[other transition paragraph?] Really, before I lay out the biases I do have to talk about simulations or at least a generating curve, otherwise it kind of doesn't make any sense. In that case, I should probably just specify the simulation I will do to replicate it in high level detail. Maybe I will introduce the saccade method here in contrast to the standard analysis and THEN describe the sources of bias from the first. That way I can use saccade notation later

From this, and what we ultimately argue here, the observed bias can be partitioned into two distinct components:

\begin{singlespace}
\begin{enumerate}
\vspace{-3mm}
\item The first source of bias, which is the primary emphasis of my proposal, is what I call the ``added observation" bias. This involves the fact that in  a standard analysis of VWP data is, the entire duration of a fixation is indicated with a $\{0,1\}$  at any time, $t$, without having observed any behavior associated with the initiation of an eye movement at that time.
\item The second source of bias is ``delayed observation bias". This bias arises from the fact that an eye movement launched at some time $t$ was planned at some time prior. This includes both the refractory period of an existing fixation, as well as oculomotor delay
\end{enumerate}
\end{singlespace}

The first source of bias, the ``added observation" bias, arises singularly from the fact that a standard analysis does not differentiate between the instance of saccade onset and the subsequent fixations in the observed data. To illustrate, consider a situation in which there is no delayed observation and that a probability that an eye movement launched at time $t$ will fixate on the target is directly determined by activation at time $t$, a la Allopenna 1998. That is, when we observe a saccade $s_t$ launched at time $t$, we are sampling directly from the activation curve following some distribution at that point in time, 

\begin{equation} \label{eq:saccade_dist}
s_t \sim Bin(f_{\theta}(t))
\end{equation}
where $f_{\theta}(t)$ is assumed to be the activation curve. What to make, then, of the subsequent fixation recorded at $t+1$? Under the current method, the ongoing fixation is treated as a readout of the activation curve at each subsequent time for the duration of the fixation. In other words, we treat the initiation of an eye movement at time $t$, governed by the underlying dynamics we wish to retrieve, as identical with the subsequent fixation over $n$ time points from $t+1$ to $t+n$, including the period of time in which there is a necessary refractory period and no new information about the underlying activation could possibly be collected from the eye mechanics. An illustration of this bias is given in Figure~\ref{fig:folly_of_fixation}


\begin{figure}[H]
    \centering
    \subfigure[]{\includegraphics[width=0.45\textwidth]{sac1.png}} 
    \subfigure[]{\includegraphics[width=0.45\textwidth]{sac2.png}} 
    \subfigure[]{\includegraphics[width=0.45\textwidth]{sac3.png}}
    \caption{These illustrations can all be made larger (they were made for slides in an image editing program), but they illustrate the main point. \textbf{(a.)} here we see an example of a generating logistic function \textbf{(b.)} at some time, $t$, a saccade is launched (in the algorithm, a binomial is drawn with probability $Bin(f_{\theta}(t))$ \textbf{(c.)} at subsequent times, $t+1, \dots, t+n$, we are recording ``observed" data, adding to the proportion of fixations at each time but without having gathered any additional observed data at $f_{\theta}(t+1), \dots,f_{\theta}(t+n)$, thus inflating (or in the case of a monotonically increasing function like the logistic, deflating) the true probability. }
\label{fig:folly_of_fixation}
\end{figure}

The consequence of this is that the \textit{amount} of observed data is artificially inflated. And in the particular case of the four parameter logistic funciton, this acts to artificially \textit{deflate} the observed probability associated with the added observations. That is, as this function is monotonein time, it follows that $f_{\theta}(t) < f_{\theta}(t+n)$ for all $t$ and $n$. As such, a saccade observed at $t$ with some probability $f_{\theta}(t)$ will also function as an observation at time when the underlying activation is actually $f_{\theta}(t+n)$, thereby ``slowing" the rate of activation. As we will see in the simulations, the result is a delayed crossover parameter and a flatter slope. [comment of relationship of total variation with observed bias, tie back to double gauss]

Finally, a quick comment on the delayed observation bias. It is well established in the literature that it takes around 200ms to plan and launch a saccade meaning that a saccade launched at time $t$ was likely planned around 200ms earlier (Viviani 1990). This phenomenon, known as oculomotor delay, is typically accounted for by shifting the observed data 200ms back before performing any analysis. While this presents no issue when the oculmotor delay is always fixed at 200ms, it is worthwhile considering the impact of this delayed observation when the true delay has an associated variability.

---

While there is no immediate solution to the effects of randomness in the delayed observation bias, we argue that the added observation bias can be rectified by using \textit{only} the observed times associated with saccade onset in the recovery of the underlying dynamics.


\paragraph{Saccade Method:} Here are a few points to be made in whatever amount of detail. First, we have to rectify the fact that we are now comparing essentially two different curves: one for the proportion of fixations, the other the probability of launching a saccade. Functionally this may be of little importance. Next, we should mention that we can fit this to the same curve (four parameter logistic) using the exact same methods (bdots). Lastly, we can maybe repeat (or move here) a mathematical description of the saccade method, namely what was shown in Equation~\ref{eq:saccade_dist}. This is nice because it lends itself to the argument that this is mathematically tractable in that we are clearly specifying the mechanism/distribution. This is less clear in the fixation method where the empirically observed $y_t$ follows no clear distribution. Finally, we should speak to the fact that we are omitting what appears to be ``information gathering behavior". This was addressed in McMurray 2022. I will elaborate more in the discussion, but in short the idea that there is info gathering behavior information in the fixations violates the assumption that activation is running in parallel from visual stimuli. By introducing the saccade method, we are leaving the fixations as an entirely separate component with some potentially interesting avenues to pursue.

\section{Simulations}

Simulations we created to isolate and investigate the aforementioned types of bias under a variety of conditions but in the context of fixations to a target object using the four-parameter logistic function denoted $f_{\theta_i}(t)$, where $t$ is time and $\theta_i$ is a subject-specific set of parameters drawn from an empirically determined distribution from normal hearing participants in (Farris-Trimble et al., 2014). 1,000 subject parameters were drawn at random, and for each subject 300 trials were simulated.

Trials were simulated with a sequence of fixations with a binomial probability that the subject would fixate on a target. A fixation onset at time $t$ was determined by the generating function, $f_{\theta_i}$ at time $t - \rho(t)$ -- here, $\rho(t)$ represents a delayed observation that will differ between simulations which I elaborate on in the next section. The duration of each fixation was randomly drawn from a gamma distribution with values independent of time and previous fixations (this is consistent with the generating curve acting independent of visual stimuli). Consistent with a standard VWP analysis, an indicator of $\{0,1\}$ was made at each 4ms interavls recording whether or not the current fixation was towards the target or not. In addition to this, I separately recorded the onset time of each fixation along its destination. Each trial continued until the cumulative sums of fixation durations surpassed 2000ms. (maybe here bring in the $z_{ijt}$ notation, show that fixation is $y_{it}$ and $\mathcal{S}_i = \{(s_k, t_k)\}$ for each of the saccade undifferentiated by trial).

A saccade launched at time $t$ was decided at time $t-\rho(t)$
 

All saccade and fixation data was then fit to the four parameter logistic function with the R package \xt{bdots} (v2) using the \xt{logistic()} function. Given sensitivity to the starting parameters when fitting the curves and to ensure consistency in the fitting algorithm, both groups were given starting parameters \xt{params = c(mini = 0, peak = 1, slope = 0.002,  cross = 750)}. The results were not sensitive to the starting parameters. For curves fit to the fixation data, fitted functions with $R^2 < 0.8$ were discarded; for saccade data, fits were excluded if the base parameter estimate exceeded the base parameter or if the slope or crossover estimates were negative. Only subjects who passed both criteria were included. In all, 996 of the original 1000 subjects were kept under fixed delay conditions and 903 under the random delay conditions.

Lastly, each section will present a histogram of the observed bias in the recovery of the generating parameters. Each section will also present a representative collection of the fitted curves, both using the saccade and fixation methods, against the original, generating curves. Summary statistics on the quality of fits are reserved for the Results section where some more general comments are made. 


\subsection{Fixed Delay}

Will elaborate on plots, analysis of histograms, etc., but not integral for general organization of paper right yet


\begin{figure}[H]
    \centering
%    \subfigure[]{\includegraphics[width=0.45\textwidth]{fixed_delay_par_bias_fixation.pdf}} 
%    \subfigure[]{\includegraphics[width=0.45\textwidth]{fixed_delay_par_bias_saccade.pdf}} 
    \caption{Distribution of parameter bias for fixation and saccade methods under fixed-delay simulation. The bias induced in the fixation method is all a consequence of the added observation bias./ We see evidence that added observation bias has the effect of ``pulling" the curve at both ends, resulting in later crossover and less steep curves}
\label{fig:fixed_par_bias}
\end{figure}




\begin{figure}[H]
\centering
%\includegraphics{fixed_pb_curves.pdf}
\caption{Representative collection of fixed-delay curve, including  the generating function, as well as estimated curves from fitting data using fixation and saccade methods}
\label{fig:fixed_pb_curves}
\end{figure}

\subsection{Random Delay}


Will elaborate on plots, analysis of histograms, etc., but not integral for general organization of paper right yet


\begin{figure}[H]
    \centering
%    \subfigure[]{\includegraphics[width=0.45\textwidth]{random_delay_par_bias_fixation.pdf}} 
%    \subfigure[]{\includegraphics[width=0.45\textwidth]{random_delay_par_bias_saccade.pdf}} 
\caption{Distribution of parameter bias for fixation and saccade methods under random-delay simulation. The bias induced in the fixation method is all a consequence of the added observation bias AND delay bias, which has consequence of further shifting crossover parameter underestimating slope, but now with saccade too. We see evidence that added observation bias has the effect of ``pulling" the curve forward, resulting in later cross over and less steep curves}
\label{fig:random_par_bias}
\end{figure}



\begin{figure}[H]
\centering
%\includegraphics{random_pb_curves.pdf}
\caption{Representative collection of random-delay curve, including  the generating function, as well as estimated curves from fitting data using fixation and saccade methods. Note the distortion in shape apparent now in both saccade and fixation curves. Specifically, note how none of the recovered curves are ever steeper than the generating}
\label{fig:random_pb_curves}
\end{figure}

\subsection{Results}

Perhaps unsurprisingly, Table~\ref{tab:mise_fixed_delay} demonstrates that (1) situations in which there is no delay between the generating function and observed behavior are easier to recover parameters and (2) the saccade method performed much better in all these cases. This table only includes MISE, I could add $R^2$, though the results will functionally be the same.

% latex table generated in R 4.2.1 by xtable 1.8-4 package
% Fri Jan 13 14:36:54 2023
\begin{table}[ht]
\centering
\begin{tabular}{llrrrrrr}
  \hline
Curve & Delay & Min. & 1st Qu. & Median & Mean & 3rd Qu. & Max. \\ 
  \hline
Fixation & Fixed & 1.95 & 8.18 & 11.40 & 13.28 & 15.98 & 215.67 \\ 
  Saccade & Fixed & 0.01 & 0.16 & 0.32 & 0.52 & 0.56 & 78.22 \\ 
  Fixation & Random & 20.25 & 50.95 & 68.60 & 73.08 & 90.92 & 192.56 \\ 
  Saccade & Random & 5.74 & 21.42 & 29.29 & 33.40 & 40.63 & 185.79 \\ 
   \hline
\end{tabular}
\caption{Summary of mean integrated squared error of the fits with their generating curves}
\label{tab:mise_fixed_delay}
\end{table}

\subsection{Discussion}

This section needs to be tightened and I have said some things elsewhere. Instead, let this be a general collection of thoughts for now.



I would like to speak a little bit more on the concept of ``information gathering behavior". One of the primary benefits of the proportion method is that it indirectly captures the duration of fixations, with longer times being associated with stronger activation. This also becomes important when differentiating fixations associated with searching patterns (i.e., what images exist on screen?) against those associated with consideration (is this the image I've just heard?). There seems to be a general consensus also that longer fixations correspond to a stronger degree of activation, but a crucially overlooked aspect of this is the implicit assumption that fixation length and activation share a linear relationship. Specifically, insofar as the construction of the fixation curves is considered, a fixation persisting at 20ms after onset (and well within the refraction period) is considered identical to a fixation persisting at 400ms. More likely it seems this would be more of an exponential relationship, with longer fixations offering increasingly more evidence of lexical activation. By separating saccades and fixations at the mathematical level, we are able to construct far more nuanced models (one proposal, for example, might be weighting the saccades by the length of their subsequent fixation, or perhaps constructing a modified activation curve $f_{\theta(t)}(t)$ whereby the parameters themselves can accelerate based on previous information. But this is neither here nor there).

Speaking to the mathematical treatment, there is a wonderful simplicity in letting the saccades themselves follow a specific distribution, namely

\begin{equation} \label{eq:saccade_dist}
s_t \sim Bin(f_{\theta}(t))
\end{equation}
or, with random oculomotor delay $\rho(t)$ (which I haven't really elaborated on as a separate mechanism), 
\begin{equation} \label{eq:saccade_dist_rho}
s_t \sim Bin(f_{\theta}(t-\rho(t)))
\end{equation}
This is in contrast to the fixation method, where the proportion of fixation curves can be described
\begin{equation} \label{eq:single_fix_measure}
y_t = \frac{1}{J} \sum z_{jt}.
\end{equation}
Here, is there a clear distribution for what $y_t$ follows? Under independence it may be the sum of binomials, but then what can be said about the relation of $y_t$ to $y_{t+1}$, given that they may or may not share overlapping fixations from different trials? This is addressed to some degree in Oleson 2017, but this seems more of an ad hoc adjustment to account for this in retrospect. In contrast, the proposed saccade method makes no assumption of trial-level relationship and instead considers all saccades over all trials as binomial samples from the same generating curve in time.

This of course does ignore trial/word/speaker variability, but then perhaps it is time that we shift our language to speaking about a distribution of generating curves for a subject rather than a particular level of activation (note too that this utility is also reflected in the conversation regarding p-values against confidence intervals). 

The arguments presented here has hoped to satisfy two goals, agnostic to the linking hypothesis or functions ultimately decided upon. Foremost is the recognition that saccades and fixations are governed by separate mechanisms, and treating them as such allows for fewer assumptions. For example, reconsider again the quote from Allopenna 1996:

 ``We made the general assumption that the probability of initiating an eye movement ot fixate on a target object $o$ at time $t$ is a direct function of the probability that $o$ is the target given the speech input and where the probability of fixating $o$ is determined by the activation level of its lexical entry relative to the activation of the other potential targets."
 
Under the saccade method, we omit the entirety of ``and where the probability of fixating $o$ is determined by the activation level of its lexical entry relative to the activation of the other potential targets" while still retaining the entirety of the utility in fitting \textit{the same non-linear curves} to less of the data. This decoupling allows the typical time-course utility of the VWP to be used in conjunction with other  methods treating aspects of the fixations separately.

Second to this, we have put a name to two important sources of potential bias in recovering generating curves in such a way as to be generalizable beyond the specifics of the assumptions of the simulation (both here and in McMurray 2022). The first, of course, addresses what was just discussed in the decoupling of saccade and fixation data. The utility of the second comes in that it makes no assumptions as to the source of the delayed observation, removing (possibly) unnecessary specifications between oculomotor delay and general mechanics when the goal is to simply recover the generating function. This may be less relevant when the goal of a study is to specifically address the mechanics of decision making (which itself seems to be difficult to pin down).

In short, what we have hoped to accomplish here is not to drastically change the original assumptions presented in Allopenna (1996) and elaborated upon in Magnuson (2019), but rather to qualify them in statistically sound ways. And really, that is pretty much it. Saccade method is neat, works the same way as the proportion of fixation method, has a more justifiable model while reducing assumptions and allowing room for others.

As a not really conclusion, I am sometimes left to wonder to what degree the proportion of fixation method was a  ``local minimum" is the pursuit of utilizing eye-tracking data. The proportion of fixations created an ostensible curve, prompting McMurray to establish theoretically grounded non-linear functions to model them. These, in turn, where shown to be suitable functions with which to model saccade data over a period of trials. Had saccades lent themselves so naturally to visualizing as the proportion of fixations, perhaps that is where we may have started.


\end{document}






