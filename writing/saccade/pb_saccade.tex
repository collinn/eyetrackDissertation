\documentclass{article}
\title{What, me saccade?}
\date{}

\usepackage{setspace}
\doublespacing

\usepackage[margin=1in]{geometry}
\usepackage{amsmath}
\usepackage{graphicx}
\newcommand{\xt}{\texttt}% 
\usepackage{listings}

\usepackage{subfigure}
\usepackage{float}

\graphicspath{{img/}}
\begin{document}

%https://www.namsu.de/Extra/klassen/latex-article-template.html



%\begin{abstract}
abstract
\end{abstract}

\section{Introduction}
intro

\paragraph{Outline}
outline







%Here I think are some of the main takeaways. First Bob showed that even under moderate assumptions, the fixation method is unable to recover unbiased parameter estimates for the generating function. Here, we examined the sources of this bias and demonstrated that by removing the period of fixation from the observed data, we are better able to estimate the generating curve. 
%
%Of course, the conclusions drawn from this rest on the tacit assumption that there is some parameteric generating function mediating the relationship between word activation and physiological behavior. By no means do we seek to argue for this either -- this lies in the domain of the linking hypothesis, a.k.a someone else's problem. However, the content of the arguments made is worth consideration. In particular, putting names to the two types of bias observed almost certainly have parallels in the empirical world: the oculomotor delay (delay bias) being a known phenomenon, and the added observation bias being tautologically true under moderate assumptions about the linking hypothesis. At very least, there is the question of the linear relationship between fixation length and activation, casting doubt on the validity of treating indicators of fixation equally at the beginning of a fixation period (and especially during the refraction  period of a fixation) as those at the end. Treating only the saccades as observations removes this issue and is more defensible from a theoretical/statistical perspective. To what degree the proposed saccade method is representative of the true state of nature is up for debate.
%
%Insomuch as it relates to the fixation method, it is worth recalling the proportion of fixation method itself was never (I think) argued for from the ground up. That is, its validity and subsequent adoption was a consequence of its agreement with the predictions of the TRACE model. To this end, we have shown that even with agreement to TRACE being the guiding principle, the saccade method shows greater fidelity to what would be predicted, even after accounting for researcher degrees of freedom.
%
%[I need to reread magnuson before using such strong langauge here]
%The conclusions that we draw from this are twofold. Even under moderate assumptions regarding the linking hypothesis, the fixation method contains at least one source of bias by conflating two very distinct types of data. Really, that's the only main conclusion, that and saccade method is cool. I said twofold above because twofold is a cool thing to have in a concluding paragraph and im pretty sure that onefold isn't a word, and even if it is it isn't as neat of a word as twofold.
%





\end{document}






