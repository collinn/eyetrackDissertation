\documentclass{article}
\title{What, me saccade?}
\date{}

\usepackage{setspace}
\doublespacing

\usepackage[margin=1in]{geometry}
\usepackage{amsmath}
\usepackage{graphicx}
\newcommand{\xt}{\texttt}% 
\usepackage{listings}

\usepackage{subfigure}
\usepackage{float}

\graphicspath{{img/}}
\begin{document}

%https://www.namsu.de/Extra/klassen/latex-article-template.html

\maketitle

%\begin{abstract}
abstract
\end{abstract}

\section{Introduction}
intro

\paragraph{Outline}
outline



\begin{abstract}
Basically there is the vwp and it is used as a proxy for word recognition. This use follows from allopenna 1996, in which he showed the the proportion of fixations to different referrents matches what would be expected with TRACE (after suitable transformation, of course). This resulted in over two decades of vwp use for such purposes. In 2021, mcmurray asked if that curve really was what we thought it was. Through an analysis of generating hypotheses of increasing (but still relatively minimal) complexity via simulation, bob showed that even in cases of moderate complexity, we were not able to positively recover the generating curve responsible for eye movements. why was this, and what are the implications for understanding these curves? In this paper, we revist the 2021 princess bride paper and offer an explanation for the demonstrated bias of the simulations. from this, we propose a new method for using vwp data to estimate underlying activation curves. finally, we top this all off by comparing the proposed method to allopena 1996 in a head-to-head, single elimination match up, a battle suitable for mount olympus itself in which two ultimate theories fight to a bloody and violent death to see which has been ordained by god to reign as total champion in the hearts and minds of language psychologists the world over.
\end{abstract}

\section*{Notes:}

\begin{singlespace}
\begin{enumerate}
\vspace{-3mm}
\item Sections that are more ``narrative" are less fleshed out. This includes VWP, TRACE, general history
\item Citations are hard coded in here awaiting a bib to be created
\item Some plots/graphics need to be redone for size
\item There is some meta commentary, partially for the reader, mostly for me
\item Spell check in my IDE is also not trivial, so I have not done that either (sorry)
\end{enumerate}
\end{singlespace}


\section{Introduction}

Spoken words create analog signals that are processed by the brain in real time. That is, as the spoken word unfolds, a cohort of possible resolutions are considered until the target word is recognized. The degree to which a particular candidate word is recognized is known as activation. An important part of this process involves not only correctly identifying the word but also eliminating competitors. For example, we might consider a discrete unfolding of the word ``elephant" as ``el-e-phant". At the onset of ``el", a listener may activate a cohort of potential resolutions such as ``elephant", ``electricity", or ``elder", all of which may be considered competitors. With the subsequent ``el-e", words consistent with the received signal, such as ``elephant" and ``electricity" remain active competitors, while incompatible words, such as ``elder", are eliminated. Such is a rough description of this process, continuing until the ambiguity is resolved and a single word remains.

%[don't like this next section]


Our interest is in measuring the degree of activation of a target, relative to competitors. Activation, however, is not measured directly, and we instead rely on what can be observed with eyetracking data, collected in the context of the Visual World Paradigm (VWP) (Tannenhaus 1995)\cite{tanenhaus1995integration}. In the last few years, researchers have begun to rexamine some of the underlying assumptions associated with the VWP, calling into question the validity of current methods. We present here a brief history of the VWP and an examination of contemporary concerns. We further address some of these directly, presenting a different method for relating eyetracking data to word activation. We demonstrate how this is resolve and conclude by comparing our new method to the original impetutus for the current. This paragraph needs work but it mostly covers the gist of what I am trying to convey, namely we are about to go from history $\rightarrow$ current stuff $\rightarrow$ proposal and comparison $\rightarrow$ back to beginning.

%\textbf{Research goals: } Some more context is helpful hear to try an understand what exactly it is that researchers are trying to learn from this data. As mentioned previously, we are largely concerned with ``activation". In particular, though, we are often interested in how the activation of competing words compare. Specifically, we might ask, ``at what point in the audio signal does the subject begin to identify the target word, relevant to competitors." This is of special interest in studying language development in typically developed children and those with cochlear implants, where atypical children may require more of a signal before they are able to disambiguate what they are hearing. As such, it is often of interest to ask when and how activation differs both between target words and competitors, as well as between different subjects. It is largely these last two areas that have dominated much of the VWP research.

I addressed above ``contemporary concerns" but i could elaborate on the more fully here. In particular, this paper is motivated by the realization that the linking hypothesis is implicitly xyz with a few contenders (none of which implemented) and then bob pretty much showing that the leading implicit assumption is ``patently false".


\section{A brief history}
We begin with a brief history to give context to later discussion. In particular, we will consider one of the leading theoretical models in speech perception, TRACE, followed by the introduction of the leading experimental paradigm, the VWP. We examine empirical evidence for the relation between these, and relevant theoretical advancements that have been made. Topics here are presented only briefly and limited to those directly relevant to the present work. For a fuller discussion of the history and uses of vwp, use google. (Or Huettig 2011b? Haven't read, found this citation in magnuson)

An outline of the presentation (for internal use only):

\begin{singlespace}
\begin{enumerate}
\vspace{-2mm}
\item TRACE in 1986 along with connectionist model of language
\item VWP by Tannenhaus 1995
\item VWP + TRACE, allopenna 1996
\item As far as I can tell, it's Bob's 2010 paper that was first to 
\begin{enumerate}
\item Look at individual differences in word recognition (relevant for the ``group distribution of curves" hypothesis and
\item Introduce parametric forms to be fit to the data (the assumption we continue to run with), or at very least, introduce ones that are interpretable
\end{enumerate}
\end{enumerate}
\end{singlespace}

I don't really like any of these paragraphs and each needs to be fleshed in more detail, but its mostly expository and not mission critical.

\paragraph{TRACE } How speech is perceived and understood has been a subject of much debate for a significant portion of psycholinguistic's history. Starting in the 1980s and persisting today, many researchers subscribe to what is known as the connectionist model of speech perception. Briefly, this model posits that speech perception is best understood as a hierarchical dynamical system in which aspects of the model are either self reinforcing or self inhibiting with feedforward and feedback mechanisms. For example, hearing the phoneme \textbackslash h\textbackslash  \ as in ``hit" will ``feedfoward", cognitively activating words that begin with the \textbackslash h\textbackslash \  sound. These activated words then ``feedback" to the phoneme letter, inhibiting activation for competing phonemes such as \textbackslash b\textbackslash \ or \textbackslash t\textbackslash. In 1986, McClelland and Elman introduced the TRACE\footnote{TRACE doesn't stand for anything -- the name is a reference to ``the trace", a network structure for dynamically processing things in memory} model implementing theoretical considerations into a computer model \cite{elman1985speech}. Maybe useful here to discuss activation, sigmoid shape, etc., 

\paragraph{VWP} To briefly illustrate, the VWP is an experimental design in which participants undergo a series of trials to identify a spoken word. Typically, each trial has a single target word, along with multiple competitors. The target word is spoken, and participants are asked to identify and select an image on screen associated with the spoken word. Eye movements and fixations are recorded as this process unfolds, with the location of the participants' eyes serving as proxy for which words/images are being considered. 

\paragraph{Relating TRACE to VWP} It was against simulated TRACE data that Allopenna (1998) found a tractable way of analyzing eye tracking data \cite{allopenna1998tracking}. By coding the period of a fixation as a 0 or 1, depending on the referent and taking the average of fixations towards a referent at each time point, Allopenna was able to create a ``fixation proportion" curve that largely reflected the shape and competitive dynamics of word activation suggested by TRACE, both for the target object, as well as competitors. This also served to establish a simple linking hypothesis, specifically, ``We made the general assumption that the probability of initiating an eye movement to fixate on a target object $o$ at time $t$ is a direct funtion of the probability that $o$ is the target given the speech input and where the probability of fixating $o$ is determined by the activation level of its lexical entry relative to the activations of other potential targets." Further of note is what this linking hypothesis does not include, namely:

\begin{singlespace}
\begin{enumerate}
\vspace{-3mm}
\item No assumption that scanning patterns in and of themselves reveal underlying cognitive processes
\item No assumption that the fixation location at time $t$ necessarily reveals where attention is directed (only probabilistically related to attention)
\end{enumerate}
\end{singlespace}

Other assumptions included here include that language processing proceeds independent of vision (Magnuson 2019), and that visual objects are not automatically activated. Or, more succinctly, it assumes that fixation proportions over time provide an essentially direct index ofe lexical activation, whereby the probability of fixating an object increases as the likelihood that it has been referred to increases.


While other linking hypotheses have been presented (Magnuson 2019), that there is \textit{some} link between the function of fixation proportions and activation has guided the last 25 years of VWP research.



\paragraph{Parametric Methods and Individual Curves} While there have most certainly been advancements to the use of the VWP for speech perception and recognition (and expanded into related domains, such as sentence processing and characterizing language disorders (according to Bob)), we  limit ourselves here to one in particular. In 2010, McMurray et al expanded the domain of the VWP by introducing emphasis on individual differences in participant activation curves. Two aspects of this paper are relevant here. First, although they were not the first to introduce non-linear functions to be fit to observed data, they did introduce a number of important parametric functions in use today, namely the four (or five) parameter logistic and the double-gauss (asymetrical gauss), the primary benefit being that the parameters of these functions are interpretable, that is, they ``describe readily observable properites." Second, which I suppose was also introduced by Mirman (2008) to some degree (though I have not read it yet, just pulling from Bob) is specifying individual subject curves across participants. This has been critical in that:

\begin{singlespace}
\begin{enumerate}
\vspace{-3mm}
\item The parameters of the functions describe interpretable properties
\item This made the idea of distributions of parameters for a particular group a relevant construct
\end{enumerate}
\end{singlespace}

Though not stated directly, this also served as the impetus for investigating group differences in word activation through the use of bootstrapped differences in time series (Oleson 2017) and the subsequent development of the \xt{bdots} software in R for analyzing such differences. (A history of exploring differences in group curves can be found in (Seedorff 2018 bdots paper)).

This brings us to the current day, where the state of things is such that TRACE-validated VWP data is widely used to measure word recognition by collecting data on individual subjects and fitting to them non-linear parametric curves with interpretable parameters. Context in hand, we are now able to introduce some of the main characters of our story, specifically how data in the VWP is understood and used. 



\begin{figure}[h]
\centering
\includegraphics[scale=0.4]{logistic_label.png}
\caption{An illustartion of the four-parameter logistic and its associated parameters, introduced as a parametric function for fixations to target objects in McMurray 2010. Can describe the parameters in detail, but should also have the formula itself somewhere to be referenced.}
\label{fig:bob_diagram_full}
\end{figure}



\section{Where we are now}

Context in hand, we are ready to introduce some of the characters of our story. this includes the finer points of the VWP, eye tracking data, and how allopena's introduction ties in with bob's parametric proposition.

\subsection{anatomy of eye  movements}

There are three components of eye movements with which we are concerned with here. The first two, saccades and fixations, are associated with physical mechanics of eye movements; the third, oculmotor delay, is a phenomenon related to the association between cognitive activaiton and physiological response. We will briefly introduce each of these topics here. 

\paragraph{Saccades and fixations:} Rather than acting in a continuous sweeping motion, as our perceived vision might suggest, our eyes themselves move about in a series of short, ballistic movements, followed by brief periods of stagnation. These, respectively, are the saccades and fixations. 

The short ballistic movements are known as saccades, periods of between 20ms-60ms (source? more accurate times?) in which they eye is in motion and during which time we are effectively blind. Once in motion, saccades have no ability to change their intended destination. Following the movement itself is a period of stillness known as a fixation, itself made up of a necessary refraction period from the saccade (time?) followed by a period of voluntary fixation; the typical duration of a fixation is (some length). Together, an initiating saccade and its subsequent fixation is known colloquially as a ``look". See Figure~\ref{fig:sac_fix_look}.

\begin{figure}
\centering
\includegraphics[scale=0.25]{sac_fix_look.png}
\caption{This image needs to be recreated for size. Illustrates saccade, fixation, and look}
\label{fig:sac_fix_look}
\end{figure}

\paragraph{Oculomotor delay:} While the physiological responses are what we can measure, they themselves are not what we are interested in. Rather, we are interested in determining word activation, itself governing the cognitive mechanism fascillitating the movements in the eyes. It's suspected/stated/known (source?) that upon finishing a particular saccade, the mind is already anticipating where it will move next. Length of about 200ms also thrown around a lot. What is relevant for our purpose here, however, is that the period of oculmotor delay is random, resulting in biased observations between what we are able to measure and what we are interested in discovering. How this phenomenon relates to saccades and fixations is demonstrated in Figure~\ref{fig:sac_fix_look_om}.


\begin{figure}
\centering
\includegraphics[scale=0.25]{om_delay2.png}
\caption{this also could probably be reformatted or made bigger}
\label{fig:sac_fix_look_om}
\end{figure}

Alternatively, there is a full figure i could use here:


\begin{figure}
\centering
\includegraphics[scale=0.5]{labeled_full_diagram.png}
\caption{This figure actually doesn't look too bad, but may be better when articulating how saccades measured and why (also includes info on $f(t)$, $\rho$, etc., so maybe we will present this later around the time of simulation}
\label{fig:full_diagram_looks}
\end{figure}


This section on OM I'm less confident about. I think that it's important to include, but it will only be used in passing in the main body of the paper. When I look at simulations (a twist on princess bride paper), I combine idea of OM and oculor mechanics into a single mechanism, both of which result in (random) delay between generating mechanism and observed behavior.

\subsection{VWP data}


We now consider how the aforementioned mechanics relate to the VWP. In a typical instantiation of the VWP, a participant is asked to complete a series of trials, during each of which they are presented with a number of competing images on screen (typically four). A verbal cue is given, and the participants are asked to select the image corresponding to the spoken word.

An individual trial of the VWP may be short, lasting anywhere from 1000ms to 2500ms before the correct image is selected. Prior to this, the participants eyes scan the environment, considering images as potential candidates to the spoken word. As this process unfolds, a snapshot of the eye is taken at a series of discrete steps (typically every 4ms) indicating where on the screen the participant is fixated. While there is evidence of cognition happening behind the scenes in a continuous fashion (spivey, mouse trials), an individual trial of the vwp may contain no more than four to eight total ``looks" before the correct image is clicked, resulting in a paucity of data in any given trial.

To create a visual summary of this process aggregated over all of the trials, a la Allopena, a ``proportion of fixations" curve is created, aggregating at each discrete timepoint the average of indicators indicating that a participant is fixated on a particular image. A resulting curve is created for each of the competing categories (target, cohort, rhyme, unrelated), creating an empirical estimate of the activation curve, $f_{\theta}(t)$. See Figure~\ref{fig:bob_diagram_full}. Mathematically, it looks like this:

\begin{equation}
y_{it} = \frac1J \sum z_{ijt}
\end{equation}
where $z_{ijt}$ is an indicator $\{0, 1\}$ for subject $i$ in trial $j$ at time $t$ and such that
\begin{equation}
f_{\theta}(t) \equiv y_t.
\end{equation}


\begin{figure}
\centering
\includegraphics[scale=0.45]{bob_vwp_full.png}
\caption{This screenshotted from Bob's princess bridge paper. i would like to reconstruct a similar illustration here as it does a great job illustrating the point. \textit{However}, this section as it stands may make more sense elaborated elsewhere, in particular where I give a mathematical treatment to what the ``fixation curve" is}
\label{fig:bob_diagram_full}
\end{figure}

Nonlinear parametric curves can be fit to the observed $y_t$ as is done in the \xt{bdots} package in R (see chapter 1).

Here, I've copied and pasted what I had about this elsewhere. There are bits and pieces I like from each, so for now going to leave them both and make them ``do it" and have babies later. 

One method employed to use this data involves measuring intervals of fixation to a target over a series of trials. For each trial $j$ and time point $t$ (typically sampled at intervals of 4ms, i.e., $t = 0, 4, \dots, 2000$), we collect a sample of $z_{it}$, an indicator of whether a participant is fixated on the target object at that point in time. Averaging over the collection of trials, we construct an estimate of $f_{\theta}(t)$, 

$$
y_t = \frac{1}{J} \sum_{j} z_{jt}.
$$
In other words, it is implicitly assumed that the trajectory of the eye follows the trajectory of activation, where the average proportion of fixations at a particular time is a direct estimate of activation. As each individual trial is only made up of a few ballistic movements, the aggregation across trials allows for these otherwise discrete measurements to more closely represent a continuous curve. Curve fitting methods, such as those employed by `bdots`, are then used to construct estimates of function parameters fitted to this curve.

[this really belongs with a compare-and-contrast section following saccades. primary benefit of this....over what?] One of the primary benefits of this method is that it captures the duration of fixations, with longer times being associated with stronger activations. This becomes important when differentiating fixations associated with searching patterns (i.e., what images exist on screen?) against those associated with consideration (is this the image I've just heard?). A shortcoming, however, is that it conflates two distinct types of data, generated via different mechanisms, the fixation and saccade. 

\section{where we are going} 


This is a bit ahead of myself, but somewhere we need to get consistent notation and keep things in order. Here is what they are:

\begin{singlespace}
\begin{enumerate}
\vspace{-3mm}
\item There is some idea of activation function. Under assumptions of allopena, this is basically the proprotion of fixation curve less oculomotor delay
\item There is the ``generating function" which I also want to call activation. In McMurray 2022, this is the four parameter logistic that determines probability of fixating on target
\item There is the fixation proprotion curve. Until now, this has been our empirical estimate of the generating fuction, but we will show that this is not quite right. Do we give it a different function name? Can this be $f_{\theta}$?
\item We will have the saccade curve. In some sense, this is similar to the proprotion of fixations in that the proportion of fixation curves is giving probability of fixating at target at some time $t$, whereas saccade curve is probability of a saccade launched at $t$ landing on the target. In situation with no OM, this is an unbiased estimator or the generating curve
\end{enumerate}
\end{singlespace}

Ok, so now what

\subsection{princess bride paper}

From the abstract of this paper: ``All theoretical and statistical approaches make the tacit assumption that the time course of fixations is closely related to the underlying activation in the system. However, given the serial nature of fixations and their long refractory period, it is unclear how closely the observed dynamics of the fixation curves are actually coupled to the underlying dynamics of activation."


This is a critical statement to be made, and in a sense it ties into the general idea of how we are handling the linkning hypothesis. I more or less adopt the same assumptions as Allopena, which essentially stipulates that our fixations are unpolluted index of lexical activation, independent of visual stimuli. This makes no attempt to differentiate ``scanning behavior" from intendend fixations, nor do I make any assumptions regarding lenght of fixations.

%\begin{singlespace}
%\begin{enumerate}
%\vspace{-3mm}
%\item Some measure of activation is occuring in the mind as a consequence of audio stimuli independent of visual stimuli
%\item Through some indeterminate mechanism, this level of activation governs the mechanics of eye movements, specifically the direction/location of saccades and duration of fixation
%\item The resulting mechanism governing \textit{saccades}, namely, where to look \textit{at a specific period in time} follows a sigmoid curve, here specifically the four-parmeter logistic ($\left arrow$ necessary assumption since I'm fitting this data to specific parametric curve)
%\end{enumerate}
%\end{singlespace}

\textbf{n.b., bob refers to these as ``fixation curve"}

In 2022, McMurray brough into question the validitity of a standard VWP analysis, and a more thorough treatment of his presented arguments is warranted but for the time being counts as narrative and so the elaboration will wait. For now, we will present those elements that are crucial for understanding the direction of the methodology to be presented.

In short, the question that is being gotten at is: are we able to recover the underlying dynamics of the system in question (activation) in light of the ``nature of the fixation record as a stochastic series of discrete and fairly long lasting physiologicall constrained events?" In short, the answer is no. McMurray notes that the typical, unspoken assumption implicit in VWP studies is the ``high frequency sampling" (HFS) assumption, which states that the underlying activation at some time determines the probability of fixation. He then goes on to note that this is ``patently" untrue and is nothing more than a polite fiction.

Nonetheless, it is useful to compare the relationship of the underlying dynamics (as we will elaborate upon further, a generating function) with the observed data in the context of the HFS, relative to other, more complex assumptions. This is done through a series of simulations, each with their own set of stochastic mechanics determining eye movements and fixations. In all cases, however, it is assumed that there is exists an underlying generating function that, at any particular time, is responsible for dictating some aspect of a subsequent fixation. We will start with an overview of the general algorithm for an individual subject, followed by a brief summary of each of the simulations.

\paragraph{Algorithm:}
\begin{singlespace}
\begin{enumerate}
\item A set of generating parameters for the four-parameter logistic is drawn from an empirically determined distribution. This curve, $f_{\theta}(t)$, is treated as the probability of fixating on a target at time $t$
\item After a random offset start time, $t_0$, a binomial random event is drawn determining the probability of fixating on the target, $p \sim Bin(f_{\theta}(t_1))$
\item After this initial draw, a fixation occurs for a period of time:
\begin{enumerate}
\item Under the HFS assumption, this period is instantaneous -- that is, whatever the time, $t$ is also the probability of fixation
\item Under the FBS assumption, the length of the fixation draws from a gamma distribution, ending at time time $t_2$
\item Under the FBS+T assumption, again a random length fixation is drawn from a gamma distribution, but with a higher mean value if the fixation is drawn to the target
\end{enumerate}
\item idk im desribing this weird -- maybe come back to this list later
\end{enumerate}
\end{singlespace}

what I can do instead is offer a brief written summary of each of the methods.

\paragraph{High Frequencly Sampling (HFS)} The underlying activation of the word \textit{is} the probability of fixating at a particular time.

\paragraph{Frequency Based Sampling (FBS)} The FBS assumption differs from that of the HFS assumption in that the observed data is gathered from a period of fixations of random duration. Once each fixation is ``drawn", the subject remains fixated on a particular object for the full length of the fixation. The next fixation's location is determined at the \textit{onset} of the previous fixation. In particular, this simulation assumes that immediately once a fixation is made, the subject begin's preparing to launch their next saccade

\paragraph{Frequency Based Sampling + Target (FBS+T)} This simulation is identical to the previous with the exception that duration of fixations to the target, while still random, follow a different different distribution than fixations to non-targets, with longer durations afforded to target fixations to account for ``information gathering behavior". 

As the complexity of the assumptions increased, so did the observed bias in recovering the parameters of the generating function.

At any rate, I'm getting to caught up in the particulars when what I really want or need to say is quite simple. It comes down to this: \textit{the only observed behavior governed by the generating curve is the saccade when launched.} 


\paragraph{Saccades}

The entirety of the bias resulting from FBS and FBS+T are the results of two facts, or put differently, there are two sources of bias that we need to consider:

\begin{singlespace}
\begin{enumerate}
\item We were ``observing" data points $\{0,1\}$ at any time $t$ without having observed any behavior from the generating curve at that time. Let's call this added observation bias.
\item When we did sample directly from the curve at fixation onset, we were actually sampling from the onset of the previous fixation. We will call this delay bias.
\end{enumerate}
\end{singlespace}

One partial solution to this, then, is to simply use the saccade data, or only collect as $\{0,1\}$ the instance at which a fixation occurs, discarding the rest. While this does not address the delay bias, it does remove a significant amount of bias from the added observations. One obvious shortcoming is that is dismisses all ``information gathering behavior" that could otherwise be gleaned from the duration of fixations. To what effect this or other enhancements may have on the efficiency of this data are yet to be seen, but at very least it offers a more clearly defensible relationship between the observed data and the generating function.

The idea of information gathering behavior is a useful one, but it assumes a linear relationship between the length of time of the fixation and strength of activation. However, one might suspect that after a period of necessary refraction, each subsequent period of time gives exponentially more weight to the argument of activation. A potential consequence of this is that an indication of fixation 50ms following the launch of a saccade may convery different information than the indication of a fixation still present 300ms following a saccade, despite the fact that these are recorded equally as $\{0,1\}$. That is, under the present system, rather than indicating the gathering of more information, longer fixations simply increase both the bias due to added observation \textit{and} the amount of bias on account of delay (as the subsequent fixation will have been determined further removed from its occurence when following a longer fixation). On the other hand, a mechanism for recording information-gathering behavior may be more readily implemented in a saccade-style method whereby each saccade is weighted by the length of its subsequent activation, for example.

Note too that this is consistent with a linking function in which lexical processing runs independently of visual display and is directly related to audio stimuli. It is further interesting to observe that some of the more complex mechanics (i.e., fixations to target object lasting longer) seek to introduce mechanisms by which visual stimuli contribute to word processing. In this case, however, fixations to target only impact observable behavior rather than ``accelerate" the underlying activation. Given the added observation bias, this inadvertently has the effect of shifting the crossover parameter forward while also decreasing the slope.


\begin{figure}
    \centering
    \subfigure[]{\includegraphics[width=0.45\textwidth]{sac1.png}} 
    \subfigure[]{\includegraphics[width=0.45\textwidth]{sac2.png}} 
    \subfigure[]{\includegraphics[width=0.45\textwidth]{sac3.png}}
    \caption{These illustrations can all be made larger (they were made for slides in an image editing program), but they illustrate the main point. \textbf{(a.)} here we see an example of a generating logistic function \textbf{(b.)} at some time, $t$, a saccade is launched (in the algorithm, a binomial is drawn with probability $Bin(f_{\theta}(t))$ \textbf{(c.)} at subsequent times, $t+1, \dots, t+n$, we are recording ``observed" data, adding to the proportion of fixations at each time but without having gathered any additional observed data at $f_{\theta}(t+1), \dots,f_{\theta}(t+n)$, thus inflating (or in the case of a monotonically increasing function like the logsitic, deflating) the true probability. }
\label{fig:folly_of_fixation}
\end{figure}

------

From here, we will describe the proposed saccade method in more detail, compare the results of using saccade data against what was found in McMurray 2022, see that it largely resolves a portion of the bias, and then ask the natural question: how does this stack up against what allopenna found?

Also note somewhere that we are primarily limiting discussion to the logistic here. A brief treatise on the asymetric gaussian is given in the appendix


\subsection{Saccade method}

If we are to consider eyetracking data samples from some probabalistic curve, it becomes necessary to differentiate between the two types. A saccade launched at some time, $t$, can be considered a sample from a data-generating mechanism at $t$. The duration of time between a given saccade and the one following follows a different mechanism altogether. By clearly delineating the mechanism from which we are sampling, we are able to reduce observed bias in the reconstruction of the activation curve.

In light of this, and in contrast to the fixation method, we propose estimating the activation curve with the saccade data alone. The primary benefit of this is two-fold. First, as suggested above, by decoupling two different types of data we are able to be more precise in what it is we are sampling. Second, as explained in the previous section, we are removing the added observation bias. 


An important difference between these two methods is in the structure of the data itself. Whereas the former collects an array of data, with an observation for each time point in each trial, the saccade method is sparse, with the observed data indicating the outcome of the saccade, as well as the time observed. It is best represented as a set of ordered pairs, $\mathcal{S} = \{(s_{j}, t_j)\}$, with $j$ indexing each of the observed saccades, and with

\begin{equation}
s_{j} \sim Bern(f_{\theta}(t_j)).
\end{equation}

A value of $s_j = 1$ indicates a saccade resulting in a fixation on the target. 


As with the proportion method, the observed data can be used as input for \xt{bdots} to construct estimates of generating parameters. 


\section{Simulations Against Princess Bride}

Here we now replicate the results of the princess bride paper, though with a few adjustments in light of the previous discussion. In particular, we noted that the two types of bias presented in the original simulations were added observation bias and delay bias. The first, added observation, we are addressing with the propsed saccade method. As to the second, we first elaborate here with a brief discussion of the delay bias and how it may also related to oculomotor delay.

In the original princess bride paper, FBS and FBS+T were only differentiated by the amount of additional bias introduced in FBS+T. Specifically, by drawing fixations to the target from a gamma distribution with a larger mean, we were both increasing the amount of added observation and delay bias, both consequence of the longer fixation period. Understanding that these differ in degree rather than kind, we collapse them into a single construct here, as we will elaborate on shortly.

We also make adjustments to how we deal with oculomotor delay. As was shown in the simulations with the HFS assumption, a fixed oculomotor delay simply resulted in a horizontal shift of the estimated function, having otherwise no impact on the \textit{shape} of the function. In contrast, added observation and delay bias both drastically impact the final shape. Further, in typical instances in which we are using the VWP, we are more frequently concerned with the relative difference between two curves rather than the curves themselves. The magnitude and relative location of such differences will be preserved under a horizontal shift, having ultimately little consequence on the resulting analysis. 

In light of this, we will seek to combine the functional impact of oculomotor delay with the impact of more complex eye behavior in the following way: we recognize that with the exception of the HFS assumption (which is not considered here), any observation at time $t$ will have been prompted at some time previously (that is, drawn from the activation curve). The length of this delay will be denoted $\rho(t)$. In the case of the HFS assumption, for example, this simply would have been $\rho(t) = 200ms$. For the FBS and FBS+T case, it would have been $\rho(t) = 200ms + \text{length of previous fixation}$. As such, we can reduce the conditions under which we compare the fixation and saccade methods to two scenarios:

\begin{singlespace}
\begin{enumerate}
\item $\rho(t)$ is a constant function, including zero
\item $\rho(t)$ is a random variable, independent of the value of $t_j$
\end{enumerate}
\end{singlespace}

As such, we will not be including as a part of this the original 200ms oculomotor delay. Instead, we will consider cases in which $\rho(t) = 0$ and $\rho(t)$ follows a gamma distribution, independent of time and of current or previous fixations (I should report mean/variance).

---

Not sure yet if this notation (below) comes up

---

As in the princess bride paper, we will let $f_{\theta}(t)$ be a four parameter logistic, representing of generating or activation curve. Understanding that what we observe at time $t$ was drawn from this function at time $t - \rho(t)$, we will differentiate the underlying activation curve the the observed data, 

\begin{equation}
g_{\theta}(t) = f_{\theta}(t - \rho(t))
\end{equation}


Each simulation will be conducted with $N = 300$ trials, sampled from the same data generating function for each, with the attempted recovery of the generating curve done using the \xt{bdots} package. 

Note somewhere: this and all other code used available on my github

\subsection{Overview}

Simluations here we conducted with only mechanisms related to fixating on the target object, that is, constructed from the four-parameter logistic curve which we here call the generating curve: 

\begin{equation}
f_{\theta}(t) = \frac{p-b}{1 + \exp \left(\frac{4s}{\text{p}-b} (x - t) \right)} + b.
\end{equation}

An empirical distribution of these parameters was generated from prior studies (Farris-Trimble, McMurray 2013). From this distribution, each subject drew a set of parameters to construct their own individual generating curve.

For an individual subject, a single trial consists of selecting a random onset time $t_0$ and, based on their generating curve, selecting with binomial probability $f_{\theta}(t_0)$ if a particular saccade would be directed towards the target. Following each draw, a fixation length is drawn from a gamma distribution, the end of which designates the subsequent time, $t_1$. Here, a second draw from the binomial is performed; the probability of fixating on the target differs depending on the method and will be explained in more detail in the following section. This process of launching saccades with some probability based on $f_{\theta}(t)$ with durations of fixations following a gamma distribution is repeated until the sum of all fixations in a single trial exceeds 2000. Two measures are recorded each trial: the first, a \xt{data.frame} indicating the onset time of each saccade, along with an indicator of whether or not it was launched towards the target. Second, a fixed length vector recorded at intervals of 4ms an indicator of the fixation. This data made up the saccade and fixation data, respectively.

For gamma -- empirically determined (though that doesn't really matter here), shape parameter 4.88, scale parameter 35.035. This gives us a gamma distribution with a mean value of 171.18 and a standard deviation of 77.443)

Each subject was simulated to have 300 trials, and 1000 subjects were randomly generated. For each subject, saccade data was concatenated as to preserve each saccade launch, its onset time, and whether or not it was launched towards the target object. Fixation data was averaged across trials at each 4ms period to create a proportion of fixation curve.

All saccade and fixation data was then fit to the four-parameter logistic function with the R package \xt{bdots} (version 1.2), using the \xt{logistic()} function with starting parameters \xt{params = c(mini = 0, peak = 1, slope = 0.002, cross = 750)}. This was to ensure consistency in the fitting algorithm performed by \xt{gnls} which is sensitive to starting parameters. For curves fit to fixation data, fitted functions with $R^2 < 0.8$ were not included; for saccade data, fits were excluded if the peak parameter estimate exceeded the base parameter or if the slope or crossover were negative. Only subjects whose curves passed both criteria were included in the final analysis. In all, 996 of the original 1000 subjects were kept.

Finally, for each simulation we investigate the distribution of parameter biases between those used for the generating curves and those recovered by \xt{bdots}. We also consider a representative collection of fitted curves for both saccade and fixation  methods against the generating curve. Finally, we consider metrics of mean integrated squared error (MISE) and $R^2$ between the two methods under both simulation conditions. 
 
 
 

\subsubsection{Fixed Delay}

Horizontal shifts in the data (such as those introduced with a 200ms oculomotor delay) do not change the shape of the data, and as was demonstrated in McMurray 2022, a simple shift under the HFS assumption was able to fully recover the generating parameters. We then choose to begin with the assumption $\rho(t) = 0$, or with no delay between when a saccade is launched and its destination. In terms detailed earlier, this means that when a fixation ends at $t_j$ and a subsequent saccade is launched, it has a probability of fixating on the target of $f_{\theta}(t_j)$. As such, we should expect the saccade method to be an unbiased estimate of the generating function (and this is indeed the case as evidenced by the figures). Accordingly, the degree to which bias is introduced in the fixation method will be solely the consequence of the added observation bias.


\begin{figure}[H]
\centering
\includegraphics{fixed_delay_par_bias.pdf}
\caption{Distribution of parameter bias for fixation and saccade methods under fixed-delay simulation. The bias induced in the fixation method is all a consequence of the added observation bias. Somewhere note these are TRUE PARS - FIT PARS. We see evidence that added observation bias has the effect of ``pulling" the curve at both ends, resulting in later crossover and less steep curves}
\label{fig:fixed_par_bias}
\end{figure}

Indeed, this is what we see when looking at the histograms of the observed bias, where negative values indicate that the fitted parameters were larger than those generating. This has the additional benefit of making theoretical sense: if the crossover point represents the time in which the probability of fixating on the target, added observations occuring \textit{before} the crossover point would artificially inflate the number of ``0"s observed, pushing the estimated crossover point forward. Further, fixations \textit{after} the crossover point would artificially inflate the number of observed  ``1"s. Given the variability of the binomial is smallest with probabilities close to 0 or 1, this effect would have the largest consequences near the beginning and end of the simulated trial data, necessitating a more gradual slope in the center, which is also evidenced in Figure~\ref{fig:fixed_par_bias}. The impact that this has on the curves themselves is also illustrated in Figure~\ref{fig:fixed_pb_curves}.


\begin{figure}[h]
\centering
\includegraphics{fixed_pb_curves.pdf}
\caption{Representative collection of fixed-delay curve, including  the generating function, as well as estimated curves from fitting data using fixation and saccade methods}
\label{fig:fixed_pb_curves}
\end{figure}


\subsubsection{Random Delay}

While estimating the generating functions in the absence of any delay is ideal for demonstrating the added observation bias, it is a poor reflection of any actual mechanics governing the link between lexical access and physiological behavior. In the random delay simulation presented here, we modify the behavior of the simulating function in one way. Rather than assuming that a saccade launched at time $t_j$ draws from a binomial with probability fixating on the target at $t_j$, we instead assume that this probability is determined at the onset of the previous fixation. This has the effect of adding a \textit{delayed observation bias}, with the delay following a gamma distribution with $\mu = 171.18$ and $\sigma = 77.443$. The results of this can be seen in Figure~\ref{fig:random_par_bias}.


\begin{figure}[H]
\centering
\includegraphics{random_delay_par_bias.pdf}
\caption{Distribution of parameter bias for fixation and saccade methods under random-delay simulation. The bias induced in the fixation method is all a consequence of the added observation bias AND delay bias, which has consequence of further shifting crossover parameter underestimating slope, but now with saccade too. Somewhere note these are TRUE PARS - FIT PARS. We see evidence that added observation bias has the effect of ``pulling" the curve forward, resulting in later cross over and less steep curves}
\label{fig:random_par_bias}
\end{figure}

The behavior of the bias is quite similar to that in the fixed delay method, and the theoretical reasons governing this behavior are also similar. The exception here is that \textit{all} observations are artifically deflated from the true probability as the generating function is monotone and any saccade launched at time $t$ was in fact determined at time $t - \rho(t)$ where $f_{\theta}(t - \rho(t)) < f_{\theta}(t)$ for all $t$.

For the fixation method, we still see evidence of the ``pulling" behavior at both ends of the curve, resulting in a consistent underestimate of the slope parameter; the bias of the crossover parameter is more pronounced and with greater variability, a combination of the delay observation acting in concert with the added observiation bias. Alternatively, consideration of the saccade method shows some degree of bias in the estimate of the slope parameter, with clear evidence of the delay bias in the crossover parameter. While the reasons for the bias in the crossover parameter are evident, those for the bias in the slope parameter are less so and may simply be a consequence of the lack of independence between these two when fit with \xt{gnls}.




\begin{figure}[H]
\centering
\includegraphics{random_pb_curves.pdf}
\caption{Representative collection of random-delay curve, including  the generating function, as well as estimated curves from fitting data using fixation and saccade methods}
\label{fig:random_pb_curves}
\end{figure}

\subsection{Discussion}

Perhaps unsurprisingly, Table~\ref{tab:mise_fixed_delay} demonstrates that (1) Situations in which there is no delay between the generating function and observed behavior are easier to recover parameters and (2) the saccade method performed much better in all these cases. This table only includes MISE, I need to add $R^2$, though the results will functionally be the same (to the degree maybe don't need $R^2$).

% latex table generated in R 4.2.1 by xtable 1.8-4 package
% Fri Jan 13 14:36:54 2023
\begin{table}[ht]
\centering
\begin{tabular}{llrrrrrr}
  \hline
Curve & Delay & Min. & 1st Qu. & Median & Mean & 3rd Qu. & Max. \\ 
  \hline
Fixation & Fixed & 1.95 & 8.18 & 11.40 & 13.28 & 15.98 & 215.67 \\ 
  Saccade & Fixed & 0.01 & 0.16 & 0.32 & 0.52 & 0.56 & 78.22 \\ 
  Fixation & Random & 20.25 & 50.95 & 68.60 & 73.08 & 90.92 & 192.56 \\ 
  Saccade & Random & 5.74 & 21.42 & 29.29 & 33.40 & 40.63 & 185.79 \\ 
   \hline
\end{tabular}
\caption{Summary of mean integrated squared error of the fits with their generating curves}
\label{tab:mise_fixed_delay}
\end{table}

As an aside, this section needs to be where I present my strongest argument, as I have come to conclude that the section on TRACE doesn't offer stronger evidence one way or the other for fixation or saccade method (I will elaborate further in that section). Parts of this may belong in general discussion at the end instead of here.

There has recently been what appears to be a renewed interest in establishing a more concrete link between lexical activation/word recognition and what is being measured with the VWP. On a grander scale, this comes down to a collection of competing hypotheses presented by Magnuson (2019). Yet, seemingly independent of the theoretical argument is what appears to be a more practical issue, namely dealing with VWP data in a tractable way. McMurray 2022 seemed to suggest that many of us carry an implicit HFS assumption and that any deviations from this (of which there certainly are) raises a number of issues in terms of what we are actually measuring with proportions of fixations in the VWP.

McMurray 2022 did not set out to establish a competing hypothesis linking theory to data; rather, he demonstrated that even under moderate deviations from HFS, what we recover is a fundamentally biased index of the mechanism we are purporting to measure. The simulations presented did commit to few additional assumptions, most generally that a generating function assume some parametric form and that this function directly determines the probability of launching a saccade (and consequently fixating) on the target object. Distilled from the particulars of these assumptions, however, are two phenomena worthy of consideration for any hypothesis linking activation to eye mechanics:

\begin{singlespace}
\begin{enumerate}
\vspace{-3mm}
\item There is certainly a delay between the mechanism of interest and any physiological behavior, and this delay almost certainly has a random component (as opposed to a fixed 200ms delay). This includes aspects of oculomotor delay, saccade refractionary periods, etc.,. We can be agnostic to any of the particular details of this delay so long as an appropriate distribution of the delay can be estimated
\item The mechanisms governing saccades and fixations are perhaps necessarily different, and there is utility is treating them as observations from separate processes. This allows accomodation of a wider range of possibilities, from scanning behavior, information gathering from the visual world, and account for length of fixation, etc.,.
\end{enumerate}
\end{singlespace}

Note: relating to the first point, while it was not shown here explicly (as we only included the assumptions with FBS rather than FBS+T, there is likely a nonlinear effect to the added observation bias in light of greater and greater delayed observation bias (that is, more added observations that themselves are more biased). We sought merely to demonstrate the existence of these biases rather than the effect, though it is likely that with investigation we could show further benefit to the saccade method when the fixation durations are more random, further distorting the ``shape" of the estiamted function. 

In short, what we have hoped to accomplish here is not to drastically change the original assumptions presented in Allopenna (1996) and elaborated upon in Magnuson (2019), but rather to qualify them in statistically sound ways. This starts by separating the processes governing saccade movement and duration of fixation.

As a not really conclusion, I am left to wonder to what degree the proportion of fixation method was a  ``local minimum" is the pursuit of utilizing eye-tracking data. The proportion of fixations created an ostensible curve, prompting McMurray to establish theoretically grounded non-linear functions to model them. These, in turn, where shown to be suitable functions with which to model saccade data over a period of trials. Had saccades leant themselves so naturally to visualizing as the proportion of fixations, perhaps that is where we may have started.

\section{Compare with TRACE}

I have a few issues with this section, and as I have fleshed out my reading and understanding of things, my intention with this section has changed. Originally (and as is still mentioned in the abstract), my hope was to show that non-linear functions fit with empirical saccade data would be a better match to what is predicted by TRACE than what is found using fixation data. 

This, however, seems to be the wrong thing to do. There is apparently a magnificent number of ways with which to transform TRACE activation data to probabilities of fixation. Neverminding the fact that the saccade curve is a fundamentally \textit{different} function than the proportion of fixations, calling into question the value of a direct comparison as well as calling into question the validity of suggested transformations for evaluating the saccade method, including those in McMurray (2010). This is related to an observation made earlier by Allopenna and friends, 

``It is important to note that although the TRACE simulations provided good fits to the behavioral data, the results should be taken as evidence in support of the class of continuous mappping models, rather than support for particular architectural assumptions made by TRACE."

Of course, this finds us in a bit of a circular loop -- Allopenna suggested that the consistency of evidence was used to support the assumptions of TRACE, and here we are suggesting TRACE as evidence for the saccade method.

What seems more appropriate, then, is to demonstrate that there \textit{exist} transformations of TRACE data in which \textit{either} the saccade or fixation methods creates a better fit (as measured by $R^2$ or MISE). As such, it makes less sense to use TRACE to show which is ``better", but rather to use TRACE to demonstrate that what is estimated with the saccade method continues to be consistent with the continuous mapping model of lexical activation. Having established theoretical consistency, my argument for the saccade method will rest on what was presented in the previous section, namely the separation of saccades and fixations, and the problems illustrated with the ``added observation" bias. 

As it is, I have a few sections here addressing high level concerns. How it should be precisely organized is up for debate, but the work itself should be largely finished.


\subsection{On fitting saccade data}

Only using the TD subjects here

I will go into more detail later on the precise transformations that I did to arrive at the empirical data from the raw data.  One key thing to note, however, is that rather than using saccade data, I finished cleaning the fixation data and then transformed this to saccade data based on the start time of subsequent fixations. This helped address ambiguities that resulted from deciding which saccades to be included. For example, if we neglected to include any saccades that began before the onset of the audio signal, the first saccades recorded (generally) had a probability of fixating on the target of about 0.25, resulting in an empirical curve with a base parameter much closer to 0.2. This was addressed by artificially setting the first saccade to have occured at $t = 0$, which nearly always had an associated recorded fixation not at the target. After making this adjustment, the baseline of the saccade  curve matched nearly that of the fixation curve, making the saccade curve more closely match the shape of the fixation curve.

Slightly more of an indulgence was how to treat the end of the saccade curves at each trial. Necessarily in all cases, following the last saccade, recorded fixations were constant until the end of the trial, drastically increasing the ``added observation" bias and resulting in a fixation curve with a peak much closer to 1.

On one hand, this could perhaps have been dealt with by addressing response times and making the appropriate adjustments. Or, far more simply (and with fewer researcher degrees of freedom), I simply added one last saccade to each trial with it's target location being that of the last fixation (typically the VWP Target). The rest of the analysis does not depend on this decision in any fashion, and as the results are functionally the same, I elected to use the saccade curve with the inflated data. This most closely matches our expectation of the relation between the fixation and saccade curves and addresses (to some degree) the asymptotic behavior of the saccades which would otherwise be uncollected. A demonstration of these differences is given in Figure~\ref{fig:saccade_inflate}.

\begin{figure}
\centering
\includegraphics{sac_inflate_compare.pdf}
\caption{this is what happens when i inflate saccade with additional saccade at last endpoint}
\label{fig:saccade_inflate}
\end{figure}

\subsection{Collecting TRACE data}

Only using TRACE data with default parameters from jTRACE

My primary concern with the TRACE data is I am seemingly unable to reconcile it visually with what is presented in the 2010 SLI paper. Specifically, I never acheive a baseline near 0. Things I have played with include:

\begin{singlespace}
\begin{enumerate}
\vspace{-3mm}
\item Adjusting the temperature in luce choice rule, both with constant factors as well as sigmoidal shapes with different parameters
\item Adjusting scaling factor to have baseline and peak values of 0 and 1 rather than values from the object being scaled
\end{enumerate}
\end{singlespace}

Referring back to an email we exchanged 12/14/2022, you gave me a list of adjustments to make to the scaling factor, including swaping the activation and crossover, as well as expanding the exponential term in the denomenator. I did this and confirmed that, as you had, the function goes from 0.0002 at maxact=-0.2 to .739 at maxact = .55. The issue, though, is that this is performed \textit{after} luce choice rule implemented. In that situation, the minimum activation observed is 0.25 rather than -0.2. This made me think that perhaps some other permutation of transformations would result in a curve starting closer to 0 and peaking nearer to .75

From there though, I can't find any combination of things that seems to correct for this. It might be a bit much, but I want to present plots of the target activation at each step to demonstrate how it changes. The order of plots is:

\begin{singlespace}
\begin{enumerate}
\vspace{-3mm}
\item Raw TRACE data
\item TRACE data after performing LCR with sigmoidal temperature data
\item Scaling factor after correcting formula
\item Scaling factor times TRACE factor after luce 
\item Here I tried multiplying the scaling factor times the raw TRACE, skipping luce choice rule (thinking that this would  preserve baseline values near zero)
\end{enumerate}
\end{singlespace}

Plot (c) in Figure~\ref{fig:shades_of_trace} is both the result following the list of transformations in the appendix of (2010) and also what is used in the analysis here. My only concern is that it's baseline value seems much larger than I feel that it should.


\begin{figure}[H]
    \centering
    \subfigure[]{\includegraphics[width=0.45\textwidth]{TRACE_test/raw_trace.pdf}} 
    \subfigure[]{\includegraphics[width=0.45\textwidth]{TRACE_test/luce_choice.pdf}} 
    \subfigure[]{\includegraphics[width=0.45\textwidth]{TRACE_test/scaling_factor.pdf}}
        \subfigure[]{\includegraphics[width=0.45\textwidth]{TRACE_test/scaling_times_luce.pdf}}
            \subfigure[]{\includegraphics[width=0.45\textwidth]{TRACE_test/skipping_luce.pdf}}
    \caption{sdfg}
\label{fig:shades_of_trace}
\end{figure}


%
%
%\subsection{TRACE Data}
%
%In McMurray (2011), a range of simulated TRACE data was collected across the space of hyperparameters, each simulation containing 14 trials, each with a Target, Cohort, Rhyme, and Unrelatd object (Appendix B. in Bob's paper). Here, we included only TRACE data associated with the default parameters specified in jTRACE (Magnuson) and found the average TRACE activation across frames/cycles for each object (target, cohort, etc.,).
%
%As has been discussed elsewhere (Allopena, Magnuson, Bob), TRACE outputs a set of activations relative to all of the words included in its lexicon, and what is needed is a linking function the is able to return a simple mapping of TRACE activations to an estimated probability. We follow the steps given in the appendix of McMurray (2011), which we briefly outline here. 
%
%First is an implementation of the Luce Choice Rule (from somewhere) which dictates that, when considering the probability of fixation of a candidate word, we need only consider the probability of fixation relative to the other candidates present rather than our entire lexicon. This means that in the context of the VWP, our probability of fixation would be relative to the (usually three) other objects on the screen, rather than the thousands of words with which we may be familiar.  
%
%Really, though, does this have to be done at all? Can I not simply point to bob's paper and be like, look, this is the trace data and this has all the data on the subjects. here we are just going to look at the results because nothing else matters or is relevant. Or this can be in the appendix

\subsection{Comparisons}

Here is where I would suggest the consistency of the models. What I show here is a moderated version of this, namely I show that there are two transformations of TRACE (changing the parameters of the sigmoid function for Luce choice rule) that each match one or the other of the fixation/saccade curves better. As such, neither is superior in any sense, but both are in the realm of consistency.


\begin{figure}[H]
    \centering
    \subfigure[]{\includegraphics[width=0.45\textwidth]{sac_fix_trace_1.pdf}} 
    \subfigure[]{\includegraphics[width=0.45\textwidth]{sac_fix_trace_2.pdf}} 

    \caption{Examples of different temperatures used in LCR and how this effects TRACE activation. In (a), this leads to greater consistency with the saccade curve; in (b), with the fixation curve. This is evidenced also by RMS values}
\label{fig:shades_of_trace2}
\end{figure}

%\begin{figure}[H]
%\centering
%\includegraphics{sac_fix_trace_compare.pdf}
%\caption{this is just the mean value of the curve parameters. also only includes NH subjects. I feel like confidence intervals would make this chart look messy so might just instead include table of mean integrated squared error using approxfun since trace only has 108 data points and need a function to integrate in R}
%\end{figure}

Actually, can have two tables here and see which makes sense. We can look at the boring as normal mean vs trace and see what shakes OR we can find MISE of each curve against trace predictions. lets do that. 30 Subjects retained after removing shitty fittys


% latex table generated in R 4.2.2 by xtable 1.8-4 package
% Wed Jan 18 18:41:24 2023
\begin{table}[ht]
\centering
\begin{tabular}{rllrrrrrr}
  \hline
 & Method & TRACE & Min. & 1st Qu. & Median & Mean & 3rd Qu. & Max. \\ 
  \hline
1 & Fixation & TRACE1 & 0.1148 & 0.1743 & 0.2181 & 0.2226 & 0.2583 & 0.4407 \\ 
  2 & Saccade & TRACE1 & 0.0749 & 0.1051 & 0.1396 & 0.1449 & 0.1655 & 0.2933 \\ \hline
  3 & Fixation & TRACE2 & 0.0991 & 0.1270 & 0.1529 & 0.1606 & 0.1712 & 0.3875 \\ 
  4 & Saccade & TRACE2 & 0.0957 & 0.1404 & 0.1830 & 0.1879 & 0.2275 & 0.3734 \\ 
   \hline
\end{tabular}
\caption{Summary of RMS of two transformations of TRACE against saccade and fixation method}
\label{tab:mise_trace}
\end{table}



\section{Discussion}

what have we learned?

Here I think are some of the main takeaways. First Bob showed that even under moderate assumptions, the fixation method is unable to recover unbiased parameter estimates for the generating function. Here, we examined the sources of this bias and demonstrated that by removing the period of fixation from the observed data, we are better able to estimate the generating curve. 

Of course, the conclusions drawn from this rest on the tacit assumption that there is some parameteric generating function mediating the relationship between word activation and physiological behavior. By no means do we seek to argue for this either -- this lies in the domain of the linking hypothesis, a.k.a someone else's problem. However, the content of the arguments made is worth consideration. In particular, putting names to the two types of bias observed almost certainly have parallels in the empirical world: the oculomotor delay (delay bias) being a known phenomenon, and the added observation bias being tautologically true under moderate assumptions about the linking hypothesis. At very least, there is the question of the linear relationship between fixation length and activation, casting doubt on the validity of treating indicators of fixation equally at the beginning of a fixation period (and especially during the refraction  period of a fixation) as those at the end. Treating only the saccades as observations removes this issue and is more defensible from a theoretical/statistical perspective. To what degree the proposed saccade method is representative of the true state of nature is up for debate.

Insomuch as it relates to the fixation method, it is worth recalling the proportion of fixation method itself was never (I think) argued for from the ground up. That is, its validity and subsequent adoption was a consequence of its agreement with the predictions of the TRACE model. To this end, we have shown that even with agreement to TRACE being the guiding principle, the saccade method shows greater fidelity to what would be predicted, even after accounting for researcher degrees of freedom.

[I need to reread magnuson before using such strong langauge here]
The conclusions that we draw from this are twofold. Even under moderate assumptions regarding the linking hypothesis, the fixation method contains at least one source of bias by conflating two very distinct types of data. Really, that's the only main conclusion, that and saccade method is cool. I said twofold above because twofold is a cool thing to have in a concluding paragraph and im pretty sure that onefold isn't a word, and even if it is it isn't as neat of a word as twofold.

something to have somewhere re decoupling of saccade and fixation and infromation gaterhing. somewhre presetned idea of weighting saccades but the info gathering itself may be implicit, whereby longer fixations simply increase propensity of movements to target, etc., removing need to control for it explicitly. Idk im not really going to touch linking hypothesis stuff because it would involve me pretending to know a lot more about this than i do.


\section{limitations}

probably good idea to keep running list of these all in one place

\begin{enumerate}
\item linking hypothesis/cognition curve
\item trace parameters maybe/general degrees of freedom
\item only evidenced on logistic, though for practical not theoretical reasons
\item adding parametric form (necessity for saccade method)
\item oculomotor delay, where to discuss
\end{enumerate}

\bibliographystyle{plain} % We choose the "plain" reference style
\bibliography{../bib/dissertation} % Entries are in the refs.bib file

\section{appendices}

Here  I am just including more or less random sections that either do not have a definite place yet in the main body of the paper, are part of what might be considered future work, or truly are things that belong in the appendix

\section*{Appendix A}



Here, I want to cover a handful of things:

\begin{enumerate}
\item Where did this trace data come from (bob's paper)
\item How could it be reconstructed in trace (parameters)
\item What words were used
\item How did I get it into acceptable format for me
\item Luce choice rule, time transformation, scaling term
\end{enumerate}

We validate our arguments above using a fortuituous study by McMurray (2010) in which subject data was collected to be analyzed against a collection of hyperparameters in TRACE testing a number of theoretical constructs in language impairment, including sensory and phonological confusion, vocabulary size, etc., presenting us with both a collection of empirically collected data from the subjects, as well as accompanying TRACE data, simulated with words used in the empirical portion of the study. Using this, we will be able to examine the relationship between empirically collected look data, saccade data, and validating TRACE data.

I don't want to in the paper go through the entire thing of how i manipulated either trace or the empirical data. i might only metnion that for a brief section in considered the implications of adjusting for the response time -- i saw that bob did do this in his in sort of a weird way, cutting off each person at their own median so that half of the observed data would be truncated, half of it would not be truncated. i dont remember the  exact reason for this but it seemed fine. not what i did though what i did was more standard.

There is a question if i should not limit things to only starttimes that occur before 1780, which i believe is the total number of cycles. of course, having that be the end vs another would have impacts on peak, slope etc. researcher degrees of freedom, hooray! 

\section*{Appendix B}

Changed mind on this section.

\section*{Appendix C}

\paragraph{Oculomotor Delay}


We begin with an assumption that the curve of interest can be represented parametrically. For example, the four parameter logistic, defined as

$$
f(t|\theta) = \frac{h-b}{1 + \exp\left(4 \cdot \frac{s}{h-b}(x - t) \right)} +b,
$$ 

is often used to describe the trajectory of probability of a subject launching a saccade and fixating on the target location while simultaneously used as a proxy for word activation. To illustrate, a subject with the depicted fixation curve may initiate a saccade beginning at time $t = 970$, with a probability of $p = 0.5$ of subsequently resting on the target:

[insert image]


Mentioned previously, this same parametric curve is used in both the proportion and saccade methods.

As saccades are easily gathered from available eyetracking data, we are, in principle, able to collect samples directly from this curve. This goal is complicated, however, by oculomotor delay. That is, an observed saccade at $t_j$ is likely a sample from the fixation curve $f_{\theta}(t)$ at some point prior to $t_j$. The degree to which this delay occurs, as well as the between and within subject variability of this delay, is a matter of active investigation. Most generally, we may consider an observation $s_j$ at time $t_j$ to be distributed
$$
s_j \sim  Bern \large[f_{\theta}(t_j - \rho(t_j)) \large],
$$ 
where $\rho(t)$ represents oculomotor delay. As written, we may consider circumstances in which:

\begin{enumerate}
\item $\rho(t)$ is a constant function (including 0)
\item $\rho(t)$ is a random variable, independent of the value of $t_j$
\item $\rho(t)$ is a random variable, dependent on $t_j$ and possibly other aspects of the trial
\end{enumerate}

To differentiate between the underlying data-generating mechanism and what is observed, we let
$$
g_{\theta}(t) = f_{\theta}(t - \rho(t)), 
$$
where $g_{\theta}(t)$ is what is \textit{observed} at time $t$. A saccade planned at $t = 300ms$ with an oculomotor delay of $\rho = 200ms$ will be observed at $t = 500ms$. That is, $g_{\theta}(500) = f_{\theta}(500 - 200) = f_{\theta}(300)$.

At present, it is common under the proportion method to account for this delay via a 200ms shift of the entire constructed proportion curve. We will propose instead a method whereby each saccade may be shifted individually and less homogenously. Reasons and implications for this will be presented in the next section.

We now consider a variety of scenarios for oculomotor delay and the subsequent impacts on the recovery of the underlying fixation curve from the observed data.

There is also this from elsewhere that i didn't delete:

---

The simulation was conducted using a fixed oculomotor delay of $\rho = 200ms$. Although the resulting recovered curve is biased, this bias simply results in a horizontal shift, $g(t) = f(t - \rho)$. This is especially relevant in a situation in which we are interested in comparing the data generating curve between two groups. 

For example, one method of analyzing VWP data (which inspired the `bdots` package) was to determine on which intervals $I = \cup_{k} I_k$ two data generating curves were statistically different. Suppose, for simplicity, that there is an interval $I = [t_1, t_2]$ on which the difference between two curves, $f(t | \theta_1) - f(t|\theta_2)$, is statistically significant. Given that we observe $g_i(t) = f(t - \rho | \theta_i)$, we would simply find that a significant difference occurs at $I + \{\rho\} = [t_1 + \rho, t_2 + \rho]$, a horizontal shift resulting from the oculomotor delay.

In other words, the size of the interval would remain the same, and the relative differences between curves would be preserved under a horizontal shift. 

\section*{Appendix D}

Idk, maybe oculmotor delay? That is kind of a disjoint section of this paper, and i don't think would get more than a mention, really, outside of a published paper. but its relevant for future direcitons, it's relevant for calling things what they are, it's not relevant for considering trace vs empirical results. Here is copy pasted out of the discussion of old paper what I had originally recordered for this:

With regards to the curve described, there are a number of avenues seemingly worthy of investigation. The most pressing of these appears to be methods to minimize the amount of bias present in scenario three, which presents the largest obstacle in the functional recovery of the data generating mechanism. Of special note here is the fact that the particular intervals in which this bias occurs can have a large effect on the overall bias, over and above that introduced by the occulomotor delay. 

For example, consider the plot above in the situation in which there is an unknown random delay (blue curve). We may observe at 500ms the value of the data generating mechanism at 300ms (that is, we observed $g_{\theta}(500) = f_{\theta}(300)$), while $g_{\theta}(500) \approx f_{\theta}(500)$. In other words, the bias over this area is small if we make no correction.

In contrast, an observation at $t = 1000$ results in a highly biased estimate, as $g_{\theta}(1000) \ll f_{\theta}(1000)$. Accordingly, we note that the amount of bias at an observed point is a function of the derivative of the data generating function in a neighborhood of that point. Whether or not this observation proves profitable remains to be seen.

There also seems to be value in finding a way to incorporate the length of fixations into the modeling process. For example, we might consider the impact of weighting each saccade by the duration of its subsequent fixation, as it seems intuitive that saccades resulting in longer fixation periods are more likely initiated by activation rather than a searching pattern.

\section*{Appendix E}

Why no double gauss treatment? simply for the reason that it is unable to find optimal fits in bdots curvefitters. in other words, with double gauss as it relates to the saccade method, there is a shortcoming in implementation rather than anything unique to it theoretically. this is mostly a consequence of \xt{gnls} sensitivity to starting parameters when fitting a function and lack of good starting parameters for dg. As this is a theoretical paper, we limit presentation to those methods that may be implemented successfully on a computer as well.

\end{document}






