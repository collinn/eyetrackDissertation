\documentclass{article}
\title{What, me saccade?}
\date{}

\usepackage{setspace}
\doublespacing

\usepackage[margin=1in]{geometry}
\usepackage{amsmath}
\usepackage{graphicx}
\newcommand{\xt}{\texttt}% 
\usepackage{listings}
\usepackage{datetime}

\usepackage{subfigure}
\usepackage{float}

\bibliographystyle{plain}

\usepackage{color}
\providecommand{\pb}[1]{\textcolor{red}{#1}}
\providecommand{\cn}[1]{\textcolor{blue}{#1}}




\graphicspath{{img/}}
\begin{document}

%https://www.namsu.de/Extra/klassen/latex-article-template.html

\maketitle

Last compiled: \today \  at \currenttime

%\begin{abstract}
abstract
\end{abstract}

\section{Introduction}
intro

\paragraph{Outline}
outline



\begin{abstract}
In 1995 the Visual World Paradigm (VWP) was introduced 
\end{abstract}



\section{Introduction}

[or, you know what, create a new introduction here and let what follows be the first part of next body sentence]

Spoken words create analog signals that are processed by the brain in real time. That is, as spoken word unfolds, a collection of candidate words are considered until the target word is recognized. The degree to which a particular candidate word is being considered is known as activation. An important part of this process involves not only correctly identifying the word but also eliminating competitors. For example, we might consider a discrete unfolding of the word ``elephant" as ``el-e-phant". At the onset of ``el", a listener may activate a cohort of potential resolutions such as ``elephant", ``electricity", or ``elder", all of which may be considered competitors. With the subsequent ``el-e", words consistent with the received signal, such as ``elephant" and ``electricity" remain active competitors, while incompatible words, such as ``elder", are eliminated. Such is a rough description of this process, continuing until the ambiguity is resolved and a single word remains.

%[don't like this next section]


[start more broadly, there are a number of ways to do this, we use activation]

Our interest is in measuring the degree of activation of a target, relative to competitors. Activation, however, is not measured directly, and we instead rely on what can be observed with physiological behavior. And though there are a number of relevant indices (Spivey mouse trials), we concern ourselves here with eye tracking data collected in the context of the Visual World Paradigm (VWP) \cite{tanenhaus1995integration}, an experimental model in which a participant's eye movements are tracked as the respond to spoken language. In a typical VWP experiment, participants are placed in the presence of visual objects (typically presented on a computer screen) and asked to select one in response to spoken language. The location of fixations are measured in real time, with the proportion of fixations towards any potential targeted aggregated across trials

The location of fixations is recorded in real time and aggregated across a series of trials







In the last few years, researchers have begun to reexamine some of the underlying assumptions associated with the VWP, calling into question the validity or interpretation of current methods. We present here a brief history of word recognition in the context of the VWP, along with an examination of contemporary concerns. We address some of these concerns directly, presenting an alternate method for relating eye-tracking data to lexical activation.


This section needs work but it mostly covers the gist of what I am trying to convey, namely we are about to go from history $\rightarrow$ current state of the world $\rightarrow$ proposal and comparison $\rightarrow$ results.

\section{A brief history}
We begin with a brief history to give context to later discussion. In particular, we will consider one of the leading theoretical models in speech perception, TRACE, followed by the introduction of the leading experimental paradigm, the VWP. We examine empirical evidence for the relation between these, and relevant theoretical advancements that have been made. Topics here are presented only briefly and limited to those directly relevant to the present work. For a fuller discussion of the history and uses of VWP. (Or Huettig 2011b?)


\begin{enumerate}
\item VWP by Tannenhaus 1995 \cite{tanenhaus1995integration}
\item VWP + TRACE, Allopenna 1996 (trace aspect no longer relevant, just an aside) \cite{allopenna1998tracking}
\item As far as I can tell, it's Bob's 2010 paper that was among first to \cite{mcmurray2010individual}
\begin{enumerate}
\item Look at individual differences in word recognition (not counting the ortho polynomial fits) (also relevant for the ``group distribution of curves" hypothesis) and
\item Introduce parametric forms to be fit to the data (the assumption we continue to run with), or at very least, introduce ones that are interpretable
\end{enumerate}
\end{enumerate}



\paragraph{Visual World Paradigm} The Visual World Paradigm (VWP) was first introduced in 1995, making the initial link between the mental processes associated with language comprehension and eye movements \cite{tanenhaus1995integration}. A typical experiment in the VWP involves situating a subject in front of a ``visual world", commonly a computer screen today, and asking them to identify and select an object corresponding to a spoken word. The initiation of eye movements and subsequent fixations are recorded as this process unfolds, with the location of the participants' eyes serving as a proxy for which words or images are being considered. This association was first demonstrated by comparing how the mean time to initiate an eye movement to the correct object was mediated by the presence of phonological competitors (``candy" and ``candle", sharing auditory signal at word onset) and situations containing syntactic ambiguity (``Put the apple on the towel in the box" and ``Put the apple \textit{that's} on the towel in the box" in ambiguous scenarios with one or more apples). It is by comparing the trajectory of these mechanics across trials in the presence of auditory or semantic competitors that researchers have used the VWP in their investigation of spoken word recognition.

[``We find that eye movements to objects in the workspace are closely time-locked to referring expression s in the unfolding speech stream, providing a sensitive and nondistruptive measure of spoken language comprehension during continuous speech" \cite{allopenna1998tracking}]


\paragraph{Proportion of fixation} It was against simulated TRACE data that Allopenna (1998) found a tractable way of analyzing eye tracking data. By coding the period of a fixation as a 0 or 1 for each referent and taking the average of fixations towards a referent at each time point, Allopenna was able to create a ``fixation proportion" curve that largely reflected the shape and competitive dynamics of word activation suggested by TRACE, both for the target object, as well as competitors. This also served to establish a simple linking hypothesis, specifically, ``We made the general assumption that the probability of initiating an eye movement to fixate on a target object $o$ at time $t$ is a direct function of the probability that $o$ is the target given the speech input and where the probability of fixating $o$ is determined by the activation level of its lexical entry relative to the activation of other potential targets." Further of note is what this linking hypothesis does not include, namely:

\begin{singlespace}
\begin{enumerate}
\vspace{-3mm}
\item No assumption that scanning patterns in and of themselves reveal underlying cognitive processes
\item No assumption that the fixation location at time $t$ necessarily reveals where attention is directed (only probabilistically related to attention)
\end{enumerate}
\end{singlespace}

Other assumptions included here include that language processing proceeds independent of vision (Magnuson 2019), and that visual objects are not automatically activated. Or, more succinctly, it assumes that fixation proportions over time provide an essentially direct index of lexical activation, whereby the probability of fixating an object increases as the likelihood that it has been referred to increases.


While other linking hypotheses have been presented (Magnuson 2019) \cite{Magnuson2019}, that there is \textit{some} link between the function of fixation proportions and activation has guided the last 25 years of VWP research.



\paragraph{Parametric Methods and Individual Curves} While there have most certainly been advancements to the use of the VWP for speech perception and recognition (and expanded into related domains, such as sentence processing and characterizing language disorders (according to Bob)), we  limit ourselves here to one in particular. In 2010, McMurray et al expanded the domain of the VWP by introducing emphasis on individual differences in participant activation curves. Two aspects of this paper are relevant here. First, although they were not the first to introduce non-linear functions to be fit to observed data, they did introduce a number of important parametric functions in use today, namely the four (or five) parameter logistic and the double-gauss (asymmetrical gauss), the primary benefit being that the parameters of these functions are interpretable, that is, they ``describe readily observable properties." Second, which I suppose was also introduced by Mirman (2008) \cite{Mirman2008} to some degree (though I have not read it yet, just pulling from Bob) is specifying individual subject curves across participants. This has been critical in that:

\begin{singlespace}
\begin{enumerate}
\vspace{-3mm}
\item The parameters of the functions describe interpretable properties
\item This made the idea of distributions of parameters for a particular group a relevant construct
\end{enumerate}
\end{singlespace}

Though not stated directly (given it predates bdots by 8 years), this also served as the impetus for investigating group differences in word activation through the use of bootstrapped differences in time series \cite{oleson2017detecting} and the subsequent development of the \xt{bdots} software in R for analyzing such differences. (A history of exploring differences in group curves can be found in \cite{seedorff2018bdots}).

This brings us to the current day, where the state of things is such that TRACE-validated VWP data is widely used to measure word recognition by collecting data on individual subjects and fitting to them non-linear parametric curves with interpretable parameters. Context in hand, we are now able to introduce some of the main characters of our story, specifically how data in the VWP is understood and used. 



\begin{figure}[h]
\centering
\includegraphics[scale=0.4]{logistic_label.png}
\caption{An illustration of the four-parameter logistic and its associated parameters, introduced as a parametric function for fixations to target objects in McMurray 2010. Can describe the parameters in detail, but should also have the formula itself somewhere to be referenced. (Equation~\ref{eq:logistic})}
\label{fig:logistic_definition}
\end{figure}



\section{Where we are now}

The following section goes into more detail on the specifics

This section includes the finer points of the VWP, eye tracking data, and how allopena's introduction ties in with bob's parametric proposition.

\subsection{Anatomy of Eye Mechanics \cn{[this section needs new name]}}

In the context of eye tracking data and word recognition, there are a few mechanics with which we are concerned. The first of these is activation [which i need to learn a little bit more about first]. Even with the immediacy and (fullness? some word they use to describe dense time series here being better than yes/no response), what we observe with any eye movement is not a direct readout of the underlying activation.  Rather, there is a period of latency between the decision to launch an eye movement and the physiological response, a period known as oculomotor delay. And finally, there are they physical mechanics of the eye movements themselves, the saccade and the fixation which, together, make up a ``look". We will briefly address each of these in the reverse of the order in which they were introduced.



\paragraph{Saccades and fixations:} Rather than acting in a continuous sweeping motion as our perceived vision might suggest, our eyes themselves move about in a series of short, ballistic movements, followed by brief periods of stagnation. These, respectively, are the saccades and fixations. 

Saccades are short, ballistic movements lasting between 20ms-60ms, during which time we are effectively blind. Once in motion, saccades are unable to change trajectory from their intended destination. Following this movement is a period known as a fixation, itself made up of a necessary refraction period (during which time the eye is incapable of movement) followed by a period of voluntary fixation which may include planning time for deciding the destination of the next eye movement; the duration of fixations are typically (some length). It will be convenient to follow previous convention and consider a saccade followed by its adjacent fixation as a single concept called a ``look" \cite{mcmurray2002look}. We take particular care here to note that the beginning of a look, or ``look onset", starts the instance that a previous look ends or, said another way, the instant an eye movement is launched. A visual description of these is provided in Figure~\ref{fig:sac_fix_look}.



\begin{figure}[H]
\centering
\includegraphics[scale=0.25]{sac_fix_look.png}
\caption{redo this image to match anataomy of look image, also for size}
\label{fig:sac_fix_look}
\end{figure}

\paragraph{Oculomotor delay:} While the physiological responses are what we can measure, they are not themselves what we are interested in. Rather, we are interested in determining word activation, itself governing the cognitive mechanism facilitating movements in the eye. Between the decision to launch an eye movement (a cognitive mechanism governed by the activation, next section) and the movement itself is a period known as oculmotor delay. It is typically estimated to take around 200ms to plan and launch an eye movement, and this is usually accounted for by subtracting 200ms from any observed behavior. \cite{viviani1990time}. As oculomotor delay is only roughly estimated to be around 200ms, we suggest that accounting for randomness will be critical in correctly recovering the the cognitive mechanism of interest or at very least in identifying possible sources of bias. How this phenomenon relates to saccades and fixations is demonstrated in Figure~\ref{fig:sac_fix_look_om}.


\begin{figure}[H]
\centering
\includegraphics[scale=0.25]{om_delay2.png}
\caption{redo this image}
\label{fig:sac_fix_look_om}
\end{figure}


\begin{figure}
\centering
\includegraphics[scale=0.5]{labeled_full_diagram.png}
\caption{I want to do this figure again but differently. have saccade be  two bars matching anatomy of look, include refractory period of fixation, noting that that and saccade are identical, followed by period of time of voluntary fixation (theoretically relevant) followed by next CM}
\label{fig:full_diagram_looks}
\end{figure}

\subsection{Activation}

Here tie in idea of activation, though need to be more concise about what we mean than we are currently. Good source for framework being (McClelland and Rumelhard 1981? Rumelhart and McClelland 81 and 82, and mcclelland/elman 86 with trace). They seem to all mention the ``interactive activation framework" which may be worthwhile to elaborate on further. For now, assume that we have adequately stated \textit{what it is}.

While a number of experimental methods are used as real-time indices of lexical access (Spivey mouse trials), we concern ourselves here with the use of eyetracking as it relates to activation as first suggested by \cite{allopenna1998tracking}. Whereas the initial treatment of eyetracking data made no attempt identify or model subject-specific trends, more recent work has made strides in making subject analysis more tractable. Specifically, we adopt the idea that each participant's results can be fit to non-linear functions who's parameters describe clinically relevant properties \cite{mcmurray2010individual}. We will denote this activation function $f$ with parameters $\theta$ as a function in time, giving $f(t|\theta)$

For example, the four parameter logistic function in Figure~\ref{fig:logistic_definition} is often used to model fixations to the target object in the VWP with functional form

\begin{equation} \label{eq:logistic}
f(t|\theta) = \frac{p-b}{1 + \exp \left(\frac{4s}{\text{p}-b} (x - t) \right)} + b.
\end{equation}

Similarly, a six parameter asymmetric Gaussian function,

\begin{equation} \label{eq:dg}
f(t|\theta) = \begin{cases}
\exp \left( \frac{(t - \mu)^2}{-2\sigma_1^2} \right) (p - b_1) + b_1 \quad \text{if } t \leq \mu \\
\exp \left( \frac{(t - \mu)^2}{-2\sigma_2^2} \right) (p - b_2) + b_2 \quad \text{if } t > \mu
\end{cases}
\end{equation}

(I didn't make a nice graph/label for this). 

While both functions are commonly used in the VWP for modeling eye fixations, for simplicity we will limit the primary focus of our discussion, though ultimately our argument is agnostic to the modeling function used, parametric or otherwise. Discussion related to the asymmetric Gauss is treated in the appendix.


\subsection{VWP data}


We now consider how the aforementioned mechanics relate to the visual world paradigm. In a typical instantiation of the VWP, a participant is asked to complete a series of trials, during each of which they are presented with a number of competing images on screen (typically four). A verbal cue is given, and the participants are asked to select the image corresponding to the spoken word. All the while, participants are wearing (generally) a head-mounted eye tracking system recording where on screen they were fixated. 

An individual trial of the VWP may be short, lasting anywhere from 1000ms to 2500ms before the correct image is selected. Prior to selecting the correct image, the participant's eyes scan the environment, considering images as potential candidates to the spoken word. As this process unfolds, a snapshot of the eye is taken at a series of discrete steps (typically every 4ms) indicating where on the screen the participant is fixated. A single trial of the VWP typically contains no more than four to eight total ``looks" before the correct image is clicked, resulting in a paucity of data in any given trial.

To be clear, eye trackers themselves only record $x$ and $y$ coordinates of the eye at any given time, and it is only after the fact that ``psychophysical" attributes are mapped onto the data (saccades, fixations, blinks, etc.,). We adopt the strategy of prior work in discussing eye tracking data in terms of their physiological mapping, as this will be crucial in constructing a physiologically relevant understanding of the problem at hand \cite{mcmurray2002look}.


To create a visual summary of this process aggregated over all of the trials, a la Allopena, a ``proportion of fixations" curve is created, aggregating at each discrete time point the average of indicators of whether or not a participant is fixated on a particular image. A resulting curve is created for each of the competing categories (target, cohort, rhyme, unrelated), creating an empirical estimate of the activation curve, $f(t|\theta)$. See Figure~\ref{fig:bob_diagram_full}. For any subject $i = 1, \dots, n$, across times $t = 0, \dots, T$ and trials $j = 1, \dots, J$, a construction  of this curves can be expressed as:


\begin{equation}\label{eq:sum_proportions}
y_{it} = \frac1J \sum z_{ijt}
\end{equation}
where $z_{ijt}$ is an indicator $\{0, 1\}$ in trial $j$ at time $t$ and such that we have an empirical estimate of the activation curve,
\begin{equation}\label{eq:empir_to_activation}
f(t | \theta_i) \equiv y_{it}.
\end{equation}



For our discussions here, we will call this the proportion of fixation method.


\begin{figure}[H]
\centering
\includegraphics[scale=0.45]{bob_vwp_full.png}
\caption{Stole this from Bob (who apparently stole it from richard aslin), plan on making my own}
\label{fig:bob_diagram_full}
\end{figure}


As each individual trial is only made up of a few ballistic movements, the aggregation across trials allows for these otherwise discrete measurements to more closely represent a continuous curve. Curve fitting methods, such as those employed by \xt{bdots}, are then used to construct estimates of function parameters fitted to this curve.

\section{Where are we going?} 


Having given due consideration to the state of things are they are, we find ourselves in a time of moral reflection, reexamining the underlying relationship between lexical activation (the mechanism of interest) and the physiological behavior we are able to observe (here, specifically eye-tracking. This is referred to in the literature as the linking hypothesis. And while there are a number of competing hypothesis, they each share a collection of implicit assumptions relating what is observed to what is being studied \cite{Magnuson2019}.

The simplest version of a linking hypothesis in the context of the VWP is the ``general assumption that the probability of initiating an eye movement to fixate on a target object $o$ at time $t$ is a direct function of the probability that $o$ is the target given the speech input and where the probability of fixating $o$ is determined by the activation level of its lexical entry relative to the activations of other potential targets (i.e., the other visible objects" \cite{allopenna1998tracking}. It is from this assumption that we justify the relation in Equation~\ref{eq:empir_to_activation}. To a degree, this assumption is shared by most linking hypothesis in that the probabilistic nature of the proportions of fixations is assumed to be related in time to the strength of the underlying activation. Primary differences in linking hypotheses tend to revolve around the particulars of the mechanics involved, including the duration of fixations, eye scanning behavior, the impacts of priming, or the relation between visual  processing acting in conjunction with lexical activation.

We consider a particular meta contribution to this debate presented by McMurray in which he probed the relationship between the observed dynamics in the fixations and the underlying dynamics of activation under a variety of assumptions \cite{mcmurray2022m}. In short, he showed that curves reconstructed using the standard proportion of fixations analysis in the VWP were poor estimates of the underlying system, with the magnitude of bias increasing on the complexity of the mechanisms involved. Though this made no specific claims as to what the underlying mechanics may be, it did demonstrate the inherent difficulty in relating observable behavior to the underlying cognitive process.

An important contribution made there, however, and one that we adopt here is an explicit definition of the underlying activation function. Given the relation in Equation~\ref{eq:empir_to_activation}, it is reasonable to assume that the underlying activation of any of the objects with the VWP could be modeled with a nonlinear function $f(t|\theta)$. The goal of a VWP analysis, then, is the recovery of this underlying function.

From this assumption we propose an alternative model of the relation between the underlying activation and the observed behavior, with a careful delineation of the psycho-physical components of a look in conjunction with its generating behavior. In particular, we consider the cognitive mechanism associated with initiating an eye movement, which is probabilistically associated with lexical activation, the delay between this and the onset of its associated look, and finally how the different components of the look are related to fundamentally different mechanisms. From this and what we ultimately argue is that observed bias in the recovery of the activation curve under the proportion of fixations method can be partitioned into two distinct components:

[i would like to maybe go into more detail here or have a picture idk]

%\begin{singlespace}
\begin{enumerate}
\vspace{-3mm}
\item The first source of bias, which is the primary emphasis of my proposal, is what I call the ``added observation" bias. This involves the fact that in  a standard analysis of VWP data is, the entire duration of a fixation is indicated with a $\{0,1\}$  at any time, $t$, without having observed any behavior associated with the initiation of an eye movement at that time.
\item The second source of bias is ``delayed observation bias". This bias arises from the fact that an eye movement launched at some time $t$ was planned at some time prior. This is primarily a consequence of the  delay
\end{enumerate}
%\end{singlespace}

The first source of bias, the ``added observation" bias, arises singularly from the fact that the destination of a look, which is observed at look onset, has a fundamentally \textit{different} generating mechanism than what determines the duration of a look, never minding such mechanics as the duration of a saccade or the refractory period of a fixation. Nonetheless, a standard analysis of VWP data does not differentiate between the initial onset and the period of subsequent fixation; both are recorded as either $0$ or $1$ according to it's location. A look onset at time $t$ is probabilistically determined by by its lexical activation $f(t|\theta)$ whereas the period of fixation is governed by a separate mechanism altogether. Treating the subsequent fixation as indistinguishable as the effect of not only ``adding" observations to the data, but adding observations that necessarily biased. The result is a distorted estimation of the underlying activation. A depiction of this phenomenon is given in Figure~\ref{fig:folly_of_fixation}. 


\begin{figure}[H]
    \centering
    \subfigure[]{\includegraphics[width=0.45\textwidth]{logistic_a.pdf}} 
    \subfigure[]{\includegraphics[width=0.45\textwidth]{logistic_b.pdf}} 
    \subfigure[]{\includegraphics[width=0.45\textwidth]{logistic_c.pdf}}
    \caption{ \textbf{(a.)} Example of a nonlinear activation curve $f(t|\theta)$ \textbf{(b.)} At some time, $t$, a saccade is launched with its destination probabilistically determined by $f(t|\theta)$ \textbf{(c.)} For a look persisting over $n$ time points, $t+1, \dots, t+n$, we are recording ``observed" data, adding to the proportion of fixations at each time but without having gathered any additional observed data at $f(t+1 | \theta), \dots,f(t+n | \theta)$, thus inflating (or in the case of a monotonically increasing function like the logistic, deflating) the true probability. }
\label{fig:folly_of_fixation}
\end{figure}

The second source of bias is the ``delayed observation" bias. It is well established in the literature that the time it takes to plan and launch a saccade is around 200ms \cite{viviani1990time}, which is typically accounted for by subtracting 200ms off of the observed data. There are two aspects of this that are worth considering further. First, if the mean duration of this oculomotor delay is not 200ms, bias will be observed as the difference between the true time and the 200ms adjustment. And although not bias in the technical sense, there has been no accounting for what effect randomness in this delay has on the recovery of the underlying activation. It will be worthwhile in investigating this as the potential magnitude will determine if this delay is worth considering in any more detail in future research.

---

While we present no immediate solution to the effects of randomness in the delayed observation bias, we argue that the added observation bias can be rectified by using \textit{only} the times observed with look onset in the recovery of the underlying dynamics. We call this the the ``look onset" method, which we explain in more detail.




\paragraph{Look Onset Method:} The look onset method differ in the proportion of fixation method only in determining which observed data should be considered relevant in the estimation of lexical activation. A particularly compelling argument to made in favor of the look onset method, a corollary of the added observation bias, is that it has a readily defensible mathematical description description. A saccade launched at time $t$ (marking the onset of a look) is assumed to be probabilistically determined by its lexical activation (relative to competitors) at time $t$, giving us

\begin{equation} \label{eq:saccade_dist}
s_t \sim Bin(f(t| \theta))
\end{equation}

(it may be that $l_t$ for look onset is better notation, but my concern is that it doesn't capture the ``onset" nature that we are concerned with and may instead suggest the entire saccade + fixation).

The utility of this is evident when tasked with stating the distribution of $y_t$ in Equation~\ref{eq:sum_proportions} as it relates to $f(t|\theta)$, where given the overlap of fixations within a particular trial, it is unclear what relation $y_t$ may have to $y_{t+1}$. 

Two further comments are made about this method here. First, in anticipation of the observation that the look onset method discards relevant information regarding the strength of activation (VOT studies, others from Magnuson review), we acknowledge this and reserve further comment for the discussion. Second given the difference in structure of the observed data, we confirm that the current iteration of \xt{bdots} is capable of fitting nonlinear curves to data both under the proportion of fixation and look onset methods.



\section{Simulations}

Simulations were conducted to replicate the mechanics of a look combined with oculomotor delay, detailed in Figure~\ref{fig:anatomy_of_look}. This section only address Target fixations with a four parameter logistic as given in Equation~\ref{eq:logistic}; simulations according to looks to competitors is treated in the appendix. We will begin by describing the process of simulating a single subject.

\begin{figure}[H]
\centering
\includegraphics[width=\textwidth]{anatomy_of_look.pdf}
\caption{Anatomy of a look -- a key thing to discuss somewhere is the OM delay, refractory period, and planning time. The latter two go in $\gamma$. Worth noting also that while we do need to be able to control for $\rho$, \textit{information} regarding strength of consideration will be in $\gamma$ - refractory period}
\label{fig:anatomy_of_look}
\end{figure}


First, each subject randomly draws a set of parameters $\theta_i$ from an empirically determined distribution based on normal hearing participants in the VWP \cite{FarrisTrimble2014} to construct a subject specific generating curve, $f(t | \theta_i)$.   It is according to this function that the decision to initiate a look at time $t$ will subsequently direct itself to the Target with probability $f(t|\theta_i)$. We then go about simulating trials according to the following method: at some time $t_0$, a subject initiates a look. This look persists for at least a duration of $\gamma$, drawn from a gamma distribution with mean and standard deviation independent of time and previous fixations. At time $t_0+\gamma$, the subject determines the location of its next look, with the next look being directed towards the target with probability $f(t+\gamma | \theta_i)$. The decision to initiate a look is followed by a period of oculomotor delay, $\rho$, during which time the subject remains fixated in the current location. Finally, at time $t_0 + \gamma + \rho$, the subject ends the look initiated at $t_0$ and immediately begins its second look to the location determined at time $t_0 + \gamma$. For the look onset method, the only data recorded are the times of a look onset and their location: in this case, at times $t_0$ and $t_0 + \gamma + \rho$. By contrast, the proportion of fixation method records the object of fixation at 4ms intervals for the entire period of length $\gamma + \rho$. A single trial begins at $t = 0$ and continues constructing looks as described until the total duration of looks exceeds 2000ms. Each subject undergoes 300 trials, and 1,000 subjects are included in each simulation.

Three total simulations were performed to investigate the biases identified in the previous section, each differing only in the random distribution of the oculomotor delay parameter, $\rho$. In the first simulation, we set $\rho = 0$ to remove any oculomotor delay. In this scenario, a look initiated at time $t$ by subject $i$ will be directed towards the target with probability $f(t|\theta_i)$. Doing so removes any potential bias from delayed observation and allows us to identify the effects of the added observation bias in isolation. In the remaining simulations we probe the effects of randomness in oculomotor delay, investigating what effect uncertainty may have in our recovery of the generating function. We did this assigning $\rho$ to follow either a normal or Weibull distribution, each with a mean value of 200ms. As is standard in a VWP analysis, we subtracted 200ms from each observed point prior to fitting the data. Note that a consequence of this is that in these simulations, the bias itself is accurately accounted for by subtracting the correct mean, with the resulting error in the curve fitting process the result of the inherent variability. This does not detract from the argument being made, however, and any true bias in the mean of the oculomotor delay would asymptotically result in a horizontal shift of the observed data according to the direction and magnitude of the bias.

As all of the data could not be individually inspected prior to being included in the analysis, subjects were excluded from consideration if fitted parameters from either the look onset method or the proportion of fixation method resulted in a peak less than the slope, or if the crossover or slope were negative. In the settings in which there was no delay, normally distributed delay, or Weibull distributed delay, 981, 973, and 981 subjects were retained, respectively.

The simulations are performed in R, with the simulation code available on the author's Github page (link?). Simulated data was fit to the four parameter logistic function using \xt{bdots v2.0.0}.


\subsection{No Delay}



\begin{figure}[H]
\centering
    \subfigure[]{\includegraphics[width=0.9\textwidth]{no_delay_par_bias_onset.pdf}} 
    \subfigure[]{\includegraphics[width=0.9\textwidth]{no_delay_par_bias_proportion.pdf}} 
\caption{Parameter bias for no oculmotor delay. }
\label{fig:par_bias_no_delay}
\end{figure}

\begin{figure}[H]
\centering
\includegraphics[width=0.9\textwidth]{rep_curves_no_delay.pdf}
\caption{Representative curves for no oculmotor delay}
\label{fig:rep_curves_no_delay}
\end{figure}



\subsection{Normal Delay}

\begin{figure}[H]
\centering
    \subfigure[]{\includegraphics[width=0.9\textwidth]{normal_delay_par_bias_onset.pdf}} 
    \subfigure[]{\includegraphics[width=0.9\textwidth]{normal_delay_par_bias_proportion.pdf}} 
\caption{Parameter bias for normal OM delay}
\label{fig:par_bias_normal_delay}
\end{figure}

\begin{figure}[H]
\centering
\includegraphics[width=0.9\textwidth]{rep_curves_normal_delay.pdf}
\caption{Representative curves for normal oculomotor delay}
\label{fig:rep_curves_normal_delay}
\end{figure}

\subsection{Weibull Delay}


\begin{figure}[H]
\centering

    \subfigure[]{\includegraphics[width=0.9\textwidth]{weibull_delay_par_bias_onset.pdf}} 
    \subfigure[]{\includegraphics[width=0.9\textwidth]{weibull_delay_par_bias_proportion.pdf}} 
\caption{Parameter bias for weibull OM delay}
\label{fig:par_bias_weibull_delay}
\end{figure}

\begin{figure}[H]
\centering
\includegraphics[width=0.9\textwidth]{rep_curves_weibull_delay.pdf}
\caption{Representative curves for weibull oculmotor delay}
\label{fig:rep_curves_weibull_delay}
\end{figure}





\subsection{Results}

% latex table generated in R 4.2.2 by xtable 1.8-4 package
% Wed Feb  8 15:10:31 2023
\begin{table}[H]
\centering
\begin{tabular}{llrrr}
  \hline
Curve & Delay & 1st Qu. & Median & 3rd Qu. \\ 
  \hline
Look Onset & No Delay & 0.17 & 0.32 & 0.56 \\ 
  Look Onset & Normal Delay & 0.37 & 0.71 & 1.24 \\ 
  Look Onset & Weibull Delay & 1.05 & 2.16 & 4.23 \\ 
  Proportion & No Delay & 8.21 & 11.33 & 16.01 \\ 
  Proportion & Normal Delay & 22.90 & 30.65 & 39.37 \\ 
  Proportion & Weibull Delay & 15.27 & 24.75 & 38.14 \\ 
   \hline
\end{tabular}
\caption{Summary of MISE across simulations. I don't think I necessarily need (or want) all of those summary stats (min/max specifically, cleaner without)}
\label{tab:mise_sims}
\end{table}

Unexplored is \textit{where} the delay occurs also important

some concluding remarks about how terrible the proportion of fixation method is

\section{Discussion}

This is now ``the" discussion

This section needs to be tightened and I have said some things elsewhere. Instead, let this be a general collection of thoughts for now.



I would like to speak a little bit more on the concept of ``information gathering behavior". One of the primary benefits of the proportion method is that it indirectly captures the duration of fixations, with longer times being associated with stronger activation. This also becomes important when differentiating fixations associated with searching patterns (i.e., what images exist on screen?) against those associated with consideration (is this the image I've just heard?). There seems to be a general consensus also that longer fixations correspond to a stronger degree of activation, but a crucially overlooked aspect of this is the implicit assumption that fixation length and activation share a linear relationship. Specifically, insofar as the construction of the fixation curves is considered, a fixation persisting at 20ms after onset (and well within the refraction period) is considered identical to a fixation persisting at 400ms. More likely it seems this would be more of an exponential relationship, with longer fixations offering increasingly more evidence of lexical activation. By separating saccades and fixations at the mathematical level, we are able to construct far more nuanced models (one proposal, for example, might be weighting the saccades by the length of their subsequent fixation, or perhaps constructing a modified activation curve $f_{\theta(t)}(t)$ whereby the parameters themselves can accelerate based on previous information. But this is neither here nor there).

Speaking to the mathematical treatment, there is a wonderful simplicity in letting the saccades themselves follow a specific distribution, namely

\begin{equation} \label{eq:saccade_dist}
s_t \sim Bin(f_{\theta}(t))
\end{equation}
or, with random oculomotor delay $\rho(t)$ (which I haven't really elaborated on as a separate mechanism), 
\begin{equation} \label{eq:saccade_dist_rho}
s_t \sim Bin(f_{\theta}(t-\rho(t)))
\end{equation}
This is in contrast to the fixation method, where the proportion of fixation curves can be described
\begin{equation} \label{eq:single_fix_measure}
y_t = \frac{1}{J} \sum z_{jt}.
\end{equation}
Here, is there a clear distribution for what $y_t$ follows? Under independence it may be the sum of binomials, but then what can be said about the relation of $y_t$ to $y_{t+1}$, given that they may or may not share overlapping fixations from different trials? This is addressed to some degree in Oleson 2017, but this seems more of an ad hoc adjustment to account for this in retrospect. In contrast, the proposed saccade method makes no assumption of trial-level relationship and instead considers all saccades over all trials as binomial samples from the same generating curve in time.

This of course does ignore trial/word/speaker variability, but then perhaps it is time that we shift our language to speaking about a distribution of generating curves for a subject rather than a particular level of activation (note too that this utility is also reflected in the conversation regarding p-values against confidence intervals). 

The arguments presented here has hoped to satisfy two goals, agnostic to the linking hypothesis or functions ultimately decided upon. Foremost is the recognition that saccades and fixations are governed by separate mechanisms, and treating them as such allows for fewer assumptions. For example, reconsider again the quote from Allopenna 1996:

 ``We made the general assumption that the probability of initiating an eye movement ot fixate on a target object $o$ at time $t$ is a direct function of the probability that $o$ is the target given the speech input and where the probability of fixating $o$ is determined by the activation level of its lexical entry relative to the activation of the other potential targets."
 
Under the saccade method, we omit the entirety of ``and where the probability of fixating $o$ is determined by the activation level of its lexical entry relative to the activation of the other potential targets" while still retaining the entirety of the utility in fitting \textit{the same non-linear curves} to less of the data. This decoupling allows the typical time-course utility of the VWP to be used in conjunction with other  methods treating aspects of the fixations separately.

Second to this, we have put a name to two important sources of potential bias in recovering generating curves in such a way as to be generalizable beyond the specifics of the assumptions of the simulation (both here and in McMurray 2022). The first, of course, addresses what was just discussed in the decoupling of saccade and fixation data. The utility of the second comes in that it makes no assumptions as to the source of the delayed observation, removing (possibly) unnecessary specifications between oculomotor delay and general mechanics when the goal is to simply recover the generating function. This may be less relevant when the goal of a study is to specifically address the mechanics of decision making (which itself seems to be difficult to pin down).

In short, what we have hoped to accomplish here is not to drastically change the original assumptions presented in Allopenna (1996) and elaborated upon in Magnuson (2019), but rather to qualify them in statistically sound ways. And really, that is pretty much it. Saccade method is neat, works the same way as the proportion of fixation method, has a more justifiable model while reducing assumptions and allowing room for others.

As a not really conclusion, I am sometimes left to wonder to what degree the proportion of fixation method was a  ``local minimum" is the pursuit of utilizing eye-tracking data. The proportion of fixations created an ostensible curve, prompting McMurray to establish theoretically grounded non-linear functions to model them. These, in turn, where shown to be suitable functions with which to model saccade data over a period of trials. Had saccades lent themselves so naturally to visualizing as the proportion of fixations, perhaps that is where we may have started.

\section{Discussion}

what have we learned?

No new contributions were added to the linking hypothesis, but introduced novel technique for identifying components of look in VWP and making a standard analysis more consistent with the original

Here are really the main takeaways.


\begin{enumerate}
\item We are all revisiting question of linking hypothesis
\item In the process of doing so, Bob identified some critical issues, revealing two distinct sources of bias
\item By introducing saccade method, we remove one source of bias and clearly delineate two separate but likely correlated mechanisms
\item This effectively keeps the assumptions from Allopenna and all of the benefits of constructing a function in time for activation, but also allowing room now for fixations to be used separately in a number of ways (length of fixation, latency to look, total fixations, etc.,)
\end{enumerate}





\section{limitations}

probably good idea to keep running list of these all in one place

\begin{singlespace}
\begin{enumerate}
\item linking hypothesis/cognition curve
\item adding parametric form (necessity for saccade method)
\item Specific results consequence of values chosen and relationship of $\gamma$ to $\rho$ (in size, they are already uncorrelated)
\end{enumerate}
\end{singlespace}

%\bibliographystyle{plain} % We choose the "plain" reference style
%\bibliography{../bib/dissertation} % Entries are in the refs.bib file

\section{appendices}

Here  I am just including more or less random sections that either do not have a definite place yet in the main body of the paper, are part of what might be considered future work, or truly are things that belong in the appendix. Presented in no particular order (commented out, input from other tex files)


%\section*{Appendix D -- TRACE}

[Moving everything TRACE related here because it turns out we don't actually need it]



\paragraph{TRACE } How speech is perceived and understood has been a subject of much debate for a significant portion of psycholinguistics' history. Starting in the 1980s and persisting today, many researchers subscribe to what is known as the connectionist model of speech perception. Briefly, this model posits that speech perception is best understood as a hierarchical dynamical system in which aspects of the model are either self reinforcing or self inhibiting with feed-forward and feedback mechanisms. For example, hearing the phoneme \textbackslash h\textbackslash  \ as in ``hit" will ``feed-foward", cognitively activating words that begin with the \textbackslash h\textbackslash \  sound. These activated words then ``feedback" to the phoneme letter, inhibiting activation for competing phonemes such as \textbackslash b\textbackslash \ or \textbackslash t\textbackslash. In 1986, McClelland and Elman introduced the TRACE\footnote{TRACE doesn't stand for anything -- the name is a reference to ``the trace", a network structure for dynamically processing things in memory} model implementing theoretical considerations into a computer model \cite{elman1985speech}. Maybe useful here to discuss activation, sigmoidal shape, etc., 


\subsection{TRACE -- section}

I have a few issues with this section, and as I have fleshed out my reading and understanding of things, my intention with this section has changed. Originally, my hope was to show that non-linear functions fit with empirical saccade data would be a better match to what is predicted by TRACE than what is found using fixation data. This, however, seems to be the wrong thing to do. There is an apparently magnificent number of ways with which to transform TRACE activation data to probabilities of fixation, Neverminding the fact that the saccade curve is a fundamentally \textit{different} concept/mechanism than the proportion of fixations, calling into question the value of a direct comparison (though I'm not sure that this is really much of an issue, as none of these transformations had mechanics uniquely specific to the properties of fixations).

This general idea is related to an observation made earlier by Allopenna and friends, 

``It is important to note that although the TRACE simulations provided good fits to the behavioral data, the results should be taken as evidence in support of the class of continuous mapping models, rather than support for particular architectural assumptions made by TRACE."

Of course, this finds us in a bit of a circular loop -- Allopenna suggested that the consistency of evidence was used to support the assumptions of TRACE, and here we are suggesting TRACE as evidence for the saccade method. What seems more appropriate, then, is to demonstrate that there \textit{exist} transformations of TRACE data in which \textit{either} the saccade or fixation methods creates a better fit (as measured by $R^2$ or MISE). As such, it makes less sense to use TRACE to show which is ``better", but rather to use TRACE to demonstrate that what is estimated with the saccade method continues to be consistent with the continuous mapping model of lexical activation. Having established theoretical consistency, my argument for the saccade method will rest on what was presented in the previous section, namely the separation of saccades and fixations, and the problems illustrated with the added observation bias. 

As it is, I have a few sections here addressing high level concerns. How it should be precisely organized is up for debate, but the work itself should be largely finished.

Finally, I will note here without much other detail -- the entirety of this next section rests with the empirical and simulated data from McMurray 2010. For the empirical data, only N/TD subjects were used and for TRACE, only the 14 simulations with the default hyper-parameters from jTRACE. The specifics of data processing (less what I mention in relevant sections here) can be cast to the appendices

\subsection{On fitting saccade data}


I will go into more detail later on the precise transformations that I did to arrive at the empirical data from the raw data.  One key thing to note, however, is that rather than using the separate saccade data in the Access DB, I finished cleaning the fixation data and then transformed this to saccade data based on the start time of subsequent fixations. This helped address ambiguities that resulted from deciding which saccades to be included. For example, if we neglected to include any saccades that began before the onset of the audio signal, the first saccades recorded (generally) had a probability of fixating on the target of about 0.25, resulting in an empirical curve with a base parameter much closer to 0.2. Additionally, this first saccade often occured some time after onset, leading to virtually no observed data near $t = 0$. This was addressed by artificially setting the first saccade to have occurred at $t = 0$ with its direction set to match that of the matching fixation at the same time. After making this adjustment, the baseline of the saccade curve matched nearly that of the fixation curve, making the saccade curve more closely match the shape of the fixation curve.

Slightly more of an indulgence was how to treat the end of the saccade curves at each trial. Necessarily in all cases, following the last saccade, recorded fixations were constant until the end of the trial, drastically increasing the ``added observation" bias and resulting in a fixation curve with a peak much closer to 1.

On one hand, this could perhaps have been dealt with by addressing response times and making the appropriate adjustments. Or, far more simply (and with fewer researcher degrees of freedom), I simply added one last saccade to the end of each trial with it's location being that of the last fixation (typically the target object). The rest of the analysis does not depend on this decision in any fashion, and as the results are functionally the same, I elected to use the saccade curve with the inflated data. This most closely matches our expectation of the relation between the fixation and saccade curves and addresses (to some degree) the asymptotic behavior of the saccades which would otherwise be uncollected. A demonstration of these differences is given in Figure~\ref{fig:saccade_inflate}.

\begin{figure}
\centering
\includegraphics{sac_inflate_compare.pdf}
\caption{this is what happens when i inflate saccade with additional saccade at last endpoint}
\label{fig:saccade_inflate}
\end{figure}

\subsection{Transforming TRACE data}


My primary concern with the TRACE data is I am seemingly unable to reconcile it visually with what is presented in the 2010 SLI paper. Specifically, I never achieve a baseline near 0. I tried manipulating the temperature of the luce choice rule (LCR) with both constant factors and sigmoidal shapes with differing parameters; I also tried playing with some of the parameters from the scaling factor function.


Referring back to an email we exchanged 12/14/2022, you (you being theBob) gave me a list of adjustments to make to the scaling factor, including swapping the activation and crossover, as well as expanding the exponential term to include the entire denominator. I did this and confirmed that, as you had, the function goes from 0.0002 at maxact=-0.2 to .739 at maxact = .55. The issue, though, is that this is performed \textit{after} luce choice rule implemented. In that situation, the minimum activation observed is 0.25 rather than -0.2. This made me think that perhaps some other permutation of transformations would result in a curve starting closer to 0 and peaking nearer to .75 (for example, scaling the raw TRACE activations). In my collection of attempts, I never found anything to quite correct for this. It may be a bit much, but I have included plots of the TRACE data related to the target at different points in the transformation process to see how it changes. Maybe something in that will ring at bell. These are included in  Figure~\ref{fig:shades_of_trace}. 


An interesting aside, though -- if we do not make the adjustment to the saccade data where we anchor asymptotic behavior at 0/1, we get a saccade curve bearing less relation to the fixation curve, but with much higher agreement with the set of TRACE curves, in particular with regards to the baseline point and peak. Presumably with some tweaking, it could be made to match even more closely. This phenomenon is illustrated in Figure~\ref{fig:unadjusted_saccade_against_trace}

\begin{figure}[H]
  \centering
  \includegraphics{unadjusted_sac_w_trace.pdf}
  \caption{Plot illustrating how the unadjusted saccade method (without anchoring at asymptotes) both matches more closely with the TRACE predictions (particularly near the baseline) while also taking on a far different shape than the fixation curve. This is in contrast to the Princess Bride simulations in which the distortion was minimal and the saccade curve appeared to be more of a horizontal shift}
  \label{fig:unadjusted_saccade_against_trace}
\end{figure}


\
\begin{figure}[H]
    \centering
    \subfigure[]{\includegraphics[width=0.45\textwidth]{TRACE_test/raw_trace.pdf}} 
    \subfigure[]{\includegraphics[width=0.45\textwidth]{TRACE_test/luce_choice.pdf}} 
    \subfigure[]{\includegraphics[width=0.45\textwidth]{TRACE_test/scaling_factor.pdf}}
    \subfigure[]{\includegraphics[width=0.45\textwidth]{TRACE_test/scaling_times_luce.pdf}}
    \subfigure[]{\includegraphics[width=0.45\textwidth]{TRACE_test/skipping_luce.pdf}}
    \begin{singlespace}
    \caption{(a) This is simply the raw TRACE data across the 14 simulations with standard parameters. (b) Transformation of TRACE activation using LCR with sigmoidal temperature. (c) Scaling factor function built on max activations \textit{after} performing LCR, using peak/baseline values from target object. (d) TRACE activations following LCR transformation and multiplying by scaling factor. This is what I have been using as model prediction of fixations, though note the baseline value being near 0.15. (e) Perhaps unnecessary, this is simply investigating TRACE activation by the scaling factor but without first conducting LCR. Note that none of these seem to have both the correct baseline and peak values} \end{singlespace}
\label{fig:shades_of_trace}
\end{figure}


\subsection{Comparisons}

Here is where I would suggest the consistency of the models. What I show here is a moderated version of this, namely I show that there are two transformations of TRACE (changing the parameters of the sigmoidal function for Luce choice rule, or the temperature directing the rate at which competitors are weeded out) that each match one or the other of the fixation/saccade curves better. As such, neither is superior in any sense, but both are in the realm of consistency.


\begin{figure}[H]
    \centering
    \subfigure[]{\includegraphics[width=0.45\textwidth]{sac_fix_trace_1.pdf}} 
    \subfigure[]{\includegraphics[width=0.45\textwidth]{sac_fix_trace_2.pdf}} 

    \caption{Examples of different temperatures used in LCR and how this effects TRACE activation. In (a), this leads to greater consistency with the saccade curve; in (b), with the fixation curve. This is evidenced also by RMS error values}
\label{fig:shades_of_trace2}
\end{figure}

%\begin{figure}[H]
%\centering
%\includegraphics{sac_fix_trace_compare.pdf}
%\caption{this is just the mean value of the curve parameters. also only includes NH subjects. I feel like confidence intervals would make this chart look messy so might just instead include table of mean integrated squared error using approxfun since trace only has 108 data points and need a function to integrate in R}
%\end{figure}

Presented in Table~\ref{tab:mise_trace} is a summary of the RMS error of the 1000s simulations using both the saccade and fixation methods against two instantiations of TRACE. As we see and corresponding to (a) in Figure~\ref{fig:shades_of_trace2} we have better agreement between the saccade method and trace predictions; this relation is flipped for case (b). 


% latex table generated in R 4.2.2 by xtable 1.8-4 package
% Wed Jan 18 18:41:24 2023
\begin{table}[ht]
\centering
\begin{tabular}{rllrrrrrr}
  \hline
 & Method & TRACE & Min. & 1st Qu. & Median & Mean & 3rd Qu. & Max. \\ 
  \hline
1 & Proportion of Fixation & TRACE1 & 0.1148 & 0.1743 & 0.2181 & 0.2226 & 0.2583 & 0.4407 \\ 
  2 & Look Onset & TRACE1 & 0.0749 & 0.1051 & 0.1396 & 0.1449 & 0.1655 & 0.2933 \\ 
  3 & Proportion of Fixation & TRACE2 & 0.0991 & 0.1270 & 0.1529 & 0.1606 & 0.1712 & 0.3875 \\ 
  4 & Look Onset & TRACE2 & 0.0957 & 0.1404 & 0.1830 & 0.1879 & 0.2275 & 0.3734 \\ 
   \hline
\end{tabular}
\caption{Summary of RMS of two transformations of TRACE against saccade and fixation method}
\label{tab:mise_trace}
\end{table}

As an aside, this also lends itself to the idea of having a \textit{distribution} of curves associated with lexical activation rather than pursuing point estimation.  In some sense, this allows a natural way to account for the observed variability in experimental conditions without having to attempt to model it. Not sure if this is an idea worth elaborating on.



%\section{Recovery of Individual Curves -- Asymmetric Gaussian}

Presented here are the results of the simulations for the recovery of subject-specific curves generated with the asymmetric Gaussian function, the parametric function typically associated with looks to competitors in the VWP. As in the section fitted with the logistic function, simulations include settings in which there is no oculomotor delay, as well as delay that is normally and Weibull distributed. Again, as all fits could not be individual examined, an automated criterion was used to determine which fits were considered  adequate. Here, this stipulated that the estimated sigma parameters be positive and that the height parameter be larger than either of the base parameters. The number of fits retained for the no delay, normal delay, and Weibull delay were  855, 786, and 816, respectively.


\subsection{Results}

As might be expected, the more complicated mean structure provided by the asymmetric Gaussian led to a generalized increase in the difficulty of recovery for both the look onset and proportion methods, relative to those generated with a logistic mean structure. However, we do still find that in the case of no delay, given in Figure~\ref{fig:dg_rep_curves_no_delay}, that the recovery of individual parameters is still unbiased and, as with the logistic, the location parameter (here, mu), is right shifted.

The results for the median integrated squared error are given in Table~\ref{tab:dg_mise_sims}. We again see results similar to those with the logistic in that the look onset method outperforms the proportion of fixation method in all cases. 

\begin{figure}[H]
\centering
\includegraphics[width=0.9\textwidth]{dg_rep_and_diff_no_delay.pdf}
\caption{Summary of simulation results in the recovery of subject-specific curves generated by the asymmetric Gauss with no oculomotor delay}
\label{fig:dg_rep_curves_no_delay}
\end{figure}

\begin{figure}[H]
\centering
\includegraphics[width=0.9\textwidth]{dg_rep_and_diff_no_delay.pdf}
\caption{Summary of simulation results in the recovery of subject-specific curves generated by the asymmetric Gauss with normally distributed oculomotor delay}
\label{fig:dg_rep_curves_normal_delay}
\end{figure}


\begin{figure}[H]
\centering
\includegraphics[width=0.9\textwidth]{dg_rep_and_diff_weibull_delay.pdf}
\caption{Summary of simulation results in the recovery of subject-specific curves generated by the asymmetric Gauss with Weibull distributed oculomotor delay}
\label{fig:dg_rep_curves_weibull_delay}
\end{figure}




\begin{table}[ht]
\centering
\begin{tabular}{llrrr}
  \hline
Curve & Delay & 1st Qu. & Median & 3rd Qu. \\ 
  \hline
Look Onset & No Delay & 0.22 & 0.36 & 0.63 \\ 
  Look Onset & Normal Delay & 0.38 & 0.70 & 1.15 \\ 
  Look Onset & Weibull Delay & 0.52 & 0.84 & 1.39 \\ 
  Proportion & No Delay & 0.75 & 1.29 & 2.08 \\ 
  Proportion & Normal Delay & 1.38 & 2.44 & 3.96 \\ 
  Proportion & Weibull Delay & 1.00 & 1.98 & 3.43 \\ 
   \hline
\end{tabular}
\caption{Median integrated squared error for recovery of individual curves generated with asymmetric Gaussian}
\label{tab:dg_mise_sims}
\end{table}

\subsection{$R^2$ instead of MISE for Recovery of Individual Curves}

Here, we provide an alternative summary of the recovery of subject specific curves fit with both the logistic and asymmetric Gauss. 


\subsubsection{Logistic}

\begin{table}[H]
\centering
\begin{tabular}{llrrr}
  \hline
Curve & Delay & 1st Qu. & Median & 3rd Qu. \\ 
  \hline
Look Onset & No Delay & 1.00 & 1.00 & 1.00 \\ 
  Look Onset & Normal Delay & 0.99 & 1.00 & 1.00 \\ 
  Look Onset & Weibull Delay & 0.98 & 0.99 & 0.99 \\ 
  Proportion & No Delay & 0.92 & 0.94 & 0.95 \\ 
  Proportion & Normal Delay & 0.80 & 0.83 & 0.86 \\ 
  Proportion & Weibull Delay & 0.80 & 0.86 & 0.91 \\ 
   \hline
\end{tabular}
\caption{$R^2$ for Logistic}
\label{tab:r2_logistic_sims}
\end{table}

\subsubsection{Asymmetric Gaussian}

\begin{table}[H]
\centering
\begin{tabular}{llrrr}
  \hline
Curve & Delay & 1st Qu. & Median & 3rd Qu. \\ 
  \hline
Look Onset & No Delay & 0.80 & 0.91 & 0.95 \\ 
  Look Onset & Normal Delay & 0.63 & 0.82 & 0.91 \\ 
  Look Onset & Weibull Delay & 0.57 & 0.77 & 0.87 \\ 
  Proportion & No Delay & 0.48 & 0.65 & 0.75 \\ 
  Proportion & Normal Delay & 0.10 & 0.33 & 0.52 \\ 
  Proportion & Weibull Delay & 0.20 & 0.46 & 0.64 \\ 
   \hline
\end{tabular}
\caption{$R^2$ for Asymmetric Gaussian}
\label{tab:r2_dg_sims}
\end{table}


\bibliography{../bib/dissertation}

\end{document}






